\documentclass{article}

\usepackage{amssymb,amsmath}

\usepackage{hyperref}
\usepackage[nameinlink]{cleveref}
\usepackage{enumerate}
\usepackage{tikz-cd}
\usepackage{graphicx}

\usepackage{mathrsfs}
%% Fancy fonts --- feel free to remove! %%
\usepackage{Baskervaldx}
\usepackage{mathpazo}


\crefname{section}{Section}{Sections}
\crefname{equation}{}{}

%% Theorem environments %%

\usepackage{amsthm}

  \theoremstyle{plain}

  \newtheorem{innercustomproposition}{Proposition}
  \crefname{innercustomproposition}{Proposition}{Propositions}
  \newenvironment{proposition}[1]
    {\renewcommand\theinnercustomproposition{#1}\innercustomproposition}
    {\endinnercustomproposition}

  \newtheorem{innercustomlemma}{Lemma}
  \crefname{innercustomlemma}{Lemma}{Lemmas}
  \newenvironment{lemma}[1]
    {\renewcommand\theinnercustomlemma{#1}\innercustomlemma}
    {\endinnercustomlemma}

  \newtheorem{innercustomcorollary}{Corollary}
  \crefname{innercustomcorollary}{Corollary}{Corollaries}
  \newenvironment{corollary}[1]
    {\renewcommand\theinnercustomcorollary{#1}\innercustomcorollary}
    {\endinnercustomcorollary}

  \newtheorem{theorem}{Theorem}


  \theoremstyle{definition}

  \newtheorem*{remark}{Remark}

  \newtheorem{innercustomdefinition}{Definition}
  \crefname{innercustomdefinition}{Definition}{Definitions}
  \newenvironment{definition}[1]
    {\renewcommand\theinnercustomdefinition{#1}\innercustomdefinition}
    {\endinnercustomdefinition}


%% Shortcuts %%

\newcommand{\sh}{\mathscr}
\newcommand{\cat}{\mathcal}
\newcommand{\HH}{\mathrm{H}}

\renewcommand{\geq}{\geqslant}
\renewcommand{\leq}{\leqslant}

\DeclareMathOperator{\ann}{ann}
\DeclareMathOperator{\supp}{supp}
\DeclareMathOperator{\rank}{rank}

\newcommand{\todo}{\textbf{ !TODO! }}
\newcommand{\oldpage}[1]{\marginpar{\footnotesize$\Big\vert$ \textit{p.~#1}}}


%% Document %%

\usepackage{embedall}
\begin{document}

\renewcommand{\abstractname}{Translator's note.}

\title{Divisors in algebraic geometry}
\author{C.S. Seshadri}
\date{}
\maketitle

\begin{abstract}
  \renewcommand*{\thefootnote}{\fnsymbol{footnote}}
  \emph{This text is one of a series\footnote{\url{https://github.com/thosgood/translations}} of translations of various papers into English.}
  \emph{The translator takes full responsibility for any errors introduced in the passage from one language to another, and claims no rights to any of the mathematical content herein.}
  
  \emph{What follows is a translation (last updated \today) of the French paper:}

  \medskip\noindent
  \textsc{Seshadri, C. S.} Diviseurs en géométrie algébrique. \emph{Séminaire Claude Chevalley}, Volume~\textbf{4} (1958-1959), Talk no.~4, 9~p. {\footnotesize\url{http://www.numdam.org/item/SCC_1958-1959__4__A4_0/}}
\end{abstract}

\setcounter{footnote}{0}

\tableofcontents


%% Content %%

\bigskip\bigskip
\oldpage{4-01}
In the first part of this exposé, we will prove a theorem of Serre on complete varieties \cite{6}, following the methods of Grothendieck \cite{4}.
The second part is dedicated to generalities on divisors.
In the literature, we often call the divisors studied here ``locally principal'' divisors.

The algebraic spaces considered here are defined over an algebraically closed field $K$.
By ``variety'', we mean an irreducible algebraic space.
If $X$ is an algebraic space, we denote by $\sh{O}(X)$, $\sh{R}(X)$, etc. (or simply $\sh{O}$, $\sh{R}$, etc.) the sheaf of local rings, of regular functions, etc. on $X$ (to define $\sh{R}(X)$ we assume that $X$ is a variety).
By ``coherent sheaf'' on $X$, we mean a coherent sheaf of $\sh{O}$-modules on $X$.


\section{Preliminaries}
\label{section1}

\cite{4,5,6}
\medskip

If $M$ is a module over an integral ring $A$ (commutative and with $1$), then we say that an element $m\in M$ is a \emph{torsion element} if there exists some non-zero $a\in A$ such that $a\cdot m=0$.
We say that $M$ is a \emph{torsion module} (resp. \emph{torsion-free module}) if every element of $M$ is a torsion element (resp. if $M\neq0$ and no non-zero element of $M$ is a torsion element).
The torsion elements of $M$ form a torsion submodule of $M$ (denoted by $T(M)$);
if $M\neq0$, then $M/T(M)$ is a torsion-free module.
If $M$ is a torsion module of finite type over $A$, then the ideal $\ann M$ of $A$ (the ideal of $A$ given by the elements $a\in A$ such that $aM=0$) is non-zero.

Let $X$ be an algebraic space and $\sh{F}$ a sheaf of $\sh{O}$-modules on $X$.
We define $\supp\sh{F}$ to be the set of points $x\in X$ such that $\sh{F}_x\neq0$.
If $\sh{F}$ is coherent, then $\supp\sh{F}$ is a closed subset of $X$.
If $X$ is affine, then $\supp\sh{F}$ is the set defined by the ideal $\ann\HH^0(X,\sh{F})$ of the affine algebra $\HH^0(X,\sh{O})$, where $\HH^0(X,\sh{F})$ is considered as a module over $\HH^0(X,\sh{O})$.

A sheaf $\sh{F}$ of $\sh{O}$-modules on a \emph{variety} $X$ is said to be a \emph{torsion sheaf} (resp. \emph{torsion-free sheaf}) if, for every $x\in X$, the module $\sh{F}_x$ over the ring $\sh{O}_x$ is a torsion module (resp. torsion-free module).

\oldpage{4-02}
\begin{proposition}{1}
\label{proposition1}
  If $\sh{F}$ is a coherent sheaf on a variety $X$, then there exists a coherent subsheaf $T(\sh{F})$ of $\sh{F}$ (and only one) such that $(T(\sh{F}))_x = T(\sh{F}_x)$.
\end{proposition}

\begin{proof}
  The uniqueness is trivial.
  The exists is a consequence of the fact that, if $X$ is affine, then $T(\sh{F}_x)$ is given by localisation of the module $T(\HH^0*(X,\sh{F}))$ with respect to the maximal ideal of $\HH^0(X,\sh{O})$ that defines $x$.
\end{proof}

\begin{corollary}
  If $\sh{F}\neq0$ \todo then $\sh{F}/T(\sh{F})$ is a torsion-free coherent sheaf.
\end{corollary}

\begin{proposition}{2}
\label{proposition2}
  If $\sh{F}$ is a coherent sheaf on the variety $X$, then $\supp\sh{F}\neq X$ if and only if $\sh{F}$ is a torsion sheaf.
\end{proposition}

\begin{proof}
  This is a trivial consequence of the fact that, if $U$ is an affine open subset, then $\supp\sh{F}\cap U$ is defined by the ideal $\ann\HH^0(U,\sh{F})$ of $\HH^0(U,\sh{O})$, where $\HH^0(U,\sh{F})$ is considered as a module over $\HH^0(U,\sh{O})$.
\end{proof}

\begin{proposition}{3}
\label{proposition3}
  If $\sh{F}$ is a torsion-free coherent sheaf on a variety $X$, with $\sh{F}\subset\sh{R}^n$, then there exists a coherent sheaf $\sh{I}\neq0$ of ideals of $\sh{O}$ such that $\sh{I}\cdot\sh{F}\subset\sh{O}^n$.
\end{proposition}

\begin{proof}
  Let $\sh{I}_x$ be the ideal $[\sh{O}_X^n:\sh{F}_x]$ of $\sh{O}_x$, i.e. the ideal of elements $i_x$ of $\sh{O}_x$ such that $i_x\sh{F}_x\subset\sh{O}_x^n$.
  Since $\sh{F}_x$ is of finite type over $\sh{O}_x$, we know that $\sh{I}_x\neq0$.
  If we take an affine open subset $U$ of $X$, then we can prove that $\sh{I}_x$ is given by localisation of the ideal $[\HH^0(U,\sh{O}^n):\HH^0(U,\sh{F})]$ of $\HH^0(U,\sh{O})$ by the maximal ideal of $\HH^0(U,\sh{O})$ that defines $x$.
  Thus $\{\sh{I}_x\}_{x\in X}$ defines a coherent sheaf $\sh{I}$ of ideals of $\sh{O}$ such that $\sh{I}\cdot\sh{F}\subset\sh{O}^n$.
\end{proof}

Let $\sh{F}$ be a torsion-free coherent sheaf on a variety $X$.
Then the canonical homomorphism $\sh{F}\to\sh{F}\otimes_{\sh{O}}\sh{R}$ is injective.
The sheaves $\sh{R}$ and $\sh{F}\otimes_{\sh{O}}\sh{R}$ are locally constant sheaves, and thus constant (\cite[page~229]{5}).
We can then identify $\sh{F}\otimes_{\sh{O}}\sh{R}$ with a vector space of finite dimension over $\sh{R}$ (we identify the field of rational functions with the sheaf $\sh{R}$ since $\sh{R}$ is constant).
We call this dimension the \emph{rank of $\sh{F}$}, and we can then consider $\sh{F}$ as a subsheaf of $\sh{R}^n$, where $n=\rank\sh{F}$.

\begin{proposition}{4}
\label{proposition4}
  Under the same hypotheses as in \cref{proposition3}, there exists a coherent sheaf $\sh{I}\neq0$\todo of ideals of $\sh{O}$ such that $\sh{I}\cdot\sh{F}\subset\sh{O}^n$, where $n=\rank\sh{F}$;
  then $\sh{O}^n/\sh{I}\cdot\sh{F}$ and $\sh{F}/\sh{I}\cdot\sh{F}$ are torsion sheaves.
\end{proposition}

\begin{proof}
  The proof is immediate.
\end{proof}

\oldpage{4-03}
If $Y$ is a closed subset of an algebraic space $X$, then we denote by $\sh{I}_Y$ the coherent sheaf of ideals of $\sh{O}$ defined by $Y$.

\begin{proposition}{5}
\label{proposition5}
  Let $Y$ be a closed subset of an algebraic space $X$, and $\sh{F}$ a coherent sheaf on $X$, with $\supp\sh{F}\subset Y$;
  then there exists an integer $k$ such that $\sh{I}_Y^k\sh{F}=0$.
\end{proposition}

\begin{proof}
  We can reduce to the case where $X$ is affine, since there exists a finite cover of $X$ by affine opens.
  In this case, the hypothesis implies that the set defined by the ideal $\ann\HH^0(X,\sh{F})$ is contained in $Y$.
  This implies, as is well known, that $\ann\HH^0(X,\sh{F})\supset\sh{I}_Y^k$.
\end{proof}

\begin{proposition}{6}
\label{proposition6}
  Let $\sh{F}$ be a coherent sheaf of fractional ideals on a variety $X$ (i.e. a coherent subsheaf of $\sh{R}$) such that, for every $x$ outside of a closed subset $Y$ of $X$, $\sh{F}_x$ is an ideal of $\sh{O}_x$.
  Then there exists an integer $k$ such that $\sh{I}_Y^k\cdot\sh{F}\subset\sh{O}$.
\end{proposition}

\begin{proof}
  By \cref{proposition3} and the hypothesis, there exists a coherent sheaf $\sh{J}$ of ideals of $\sh{O}$ such that $\sh{J}_x=\sh{O}_x$ if $x\not\in Y$, and such that $\sh{J}\cdot\sh{F}\subset\sh{O}$.
  Thus $\supp(\sh{O}/sh{J})\subset Y$, and, by \cref{proposition5}, there exists an integer $k$ such that $\sh{I}_Y^k(\sh{O}/\sh{J})=0$ \todo.
  This implies that $\sh{I}_Y^k\subset\sh{J}$.
\end{proof}


\section{D\'{e}vissage theorem}
\label{section2}

Let $\cat{C}$ be an abelian category, and $\cat{C}'$ a subcategory of objects of $\cat{C}$.
We say that $\cat{C}'$ is \emph{left exact in $\cat{C}$} if
\begin{enumerate}
  \item every subobject of an object of $\cat{C}'$ is in $\cat{C}'$;
  \item for every exact sequence $0\to\sh{A}'\to\sh{A}\to\sh{A}''\to0$ in $\cat{C}$, the object $\sh{A}$ is in $\cat{C}'$ if the other two objects are in $\cat{C}'$.
  \footnote{The axioms here that define a left-exact subcategory are slightly stronger than those of Grothendieck \cite{4}.}
\end{enumerate}

Let $X$ be an algebraic space.
We denote by $\cat{C}(X)$ the abelian category of coherent sheaves on $X$.
If $Y$ is a closed subset of $X$, then a coherent sheaf on $Y$ has a canonical extension to a coherent sheaf on $X$ (extending by $0$ outside of $Y$), and so we can consider $\cat{C}(Y)$ as a subcategory of $\cat{C}(X)$.
With this notation, we have the following theorem:

\oldpage{4-04}
\begin{theorem}[D\'{e}vissage]
  Let $\cat{D}$ be a left-exact subcategory of $\cat{C}(X)$ that has the following property:
  for every closed irreducible subset $Y$ of $X$, there exists a coherent sheaf $\sh{M}_Y$ of $\cat{C}(Y)$ that belongs to $\cat{D}$, and that is torsion-free as a sheaf on $Y$.
  Then $\cat{D}=\cat{C}(X)$.
\end{theorem}

\begin{proof}
  The proof works by induction on the dimension of $X$.
  If $\dim X=0$, then $X$ consists of a finite number of points $P_1,\ldots,P_r$, and a coherent sheaf on $X$ can be identified with a system $\{N_i\}_{i=1,\ldots,r}$, where $N_i$ is a vector space of finite dimension over $K$.
  Thus the sheaf $\sh{M}_{P_i}$ on $P_i$ that we have, by hypothesis, is a vector space of finite dimension over $K$.
  By the axioms of a left-exact subcategory, it is trivial to show that every system $\{N_i\}_{i=1,\ldots,r}$, where $N_i$ is a vector space of finite dimension over $K$, considered as a coherent sheaf on $X$, belongs to $\cat{D}$.

  Now
\end{proof}


%% Bibliography %%

\nocite{*}
\bibliographystyle{acm}

\begin{thebibliography}{10}

  \bibitem{1}
  {\sc Cartier, P.}
  \newblock Diviseurs et d\'{e}rivations en g\'{e}om\'{e}trie alg\'{e}brique.
  \newblock {({\em Th\`{e}se Sc. math. Paris.} 1958)}
  \newblock (To appear in {\em Bull. Soc. math. France}).

  \bibitem{2}
  {\sc Chevalley, C.}
  \newblock Fondements de la G\'{e}om\'{e}trie alg\'{e}brique
  \newblock Paris, Secr\'{e}tariat math\'{e}matique, 1958, multigraphed.
  \newblock (Class taught at the Sorbonne in 1957--58).

  \bibitem{3}
  {\sc Godement, R.}
  \newblock Propri\'{e}t\'{e}s analytiques des localit\'{e}s
  \newblock In {\em S\'{e}minarie Cartan-Chevalley}, vol.~8, 1955--56.
  \newblock (Talk number 19).

  \bibitem{4}
  {\sc Grothendieck, A.}
  \newblock Sur les faisceaux alg\'{e}briques et les faisceaux analytiques
    coh\'{e}rents.
  \newblock In {\em S\'{e}minaire H. Cartan}, vol.~9.
  \newblock (Talk number 2).

  \bibitem{5}
  {\sc Serre, J.-P.}
  \newblock Faisceaux alg\'{e}briques coh\'{e}rents.
  \newblock {\em Ann. Math. (2)\/}, vol.~61 (1955), 197--279.

  \bibitem{6}
  {\sc Serre, J.-P.}
  \newblock Sur la cohomologie des vari\'{e}t\'{e}s alg\'{e}briques.
  \newblock {\em J. Math. pures et appl. (9)\/}, vol.~36 (1957), 1--16.

  \bibitem{7}
  {\sc Weil, A.}
  \newblock Fibre spaces in algebraic geometry
  \newblock (Notes taken by A. Wallace, 1952)
  \newblock Chicago, University of Chicago, 1955.

\end{thebibliography}

\end{document}
