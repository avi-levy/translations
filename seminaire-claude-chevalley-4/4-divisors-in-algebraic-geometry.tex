\documentclass{article}

\title{Divisors in algebraic geometry}
\author{C.S. Seshadri}
\date{}

\usepackage{amssymb,amsmath}

\usepackage{hyperref}
\usepackage[nameinlink]{cleveref}
\usepackage{enumerate}

\usepackage{mathrsfs}
%% Fancy fonts --- feel free to remove! %%
\usepackage{Baskervaldx}
\usepackage{mathpazo}


\usepackage{fancyhdr}
\usepackage{lastpage}
\usepackage{xstring}
\makeatletter
\ifx\pdfmdfivesum\undefined
  \let\pdfmdfivesum\mdfivesum
\fi
\edef\filesum{\pdfmdfivesum file {\jobname}}
\pagestyle{fancy}
\makeatletter
\let\runauthor\@author
\let\runtitle\@title
\makeatother
\fancyhf{}
\lhead{\footnotesize\runtitle}
\rhead{\footnotesize Version: \texttt{\StrMid{\filesum}{1}{8}}}
\cfoot{\small\thepage\ of \pageref*{LastPage}}


\crefname{section}{Section}{Sections}
\crefname{equation}{}{}

%% Theorem environments %%

\usepackage{amsthm}

  \theoremstyle{plain}

  \newtheorem{innercustomproposition}{Proposition}
  \crefname{innercustomproposition}{Proposition}{Propositions}
  \newenvironment{proposition}[1]
    {\renewcommand\theinnercustomproposition{#1}\innercustomproposition}
    {\endinnercustomproposition}

  \newtheorem*{theorem}{Theorem}
  \newtheorem*{cor}{Corollary}


  \theoremstyle{definition}

  \newtheorem*{remark}{Remark}


%% Shortcuts %%

\newcommand{\sh}{\mathscr}
\newcommand{\cat}{\mathcal}
\newcommand{\HH}{\mathrm{H}}

\renewcommand{\geq}{\geqslant}
\renewcommand{\leq}{\leqslant}
\renewcommand{\Im}{\operatorname{Im}}
\renewcommand{\div}{\operatorname{div}}

\DeclareMathOperator{\ann}{ann}
\DeclareMathOperator{\supp}{supp}
\DeclareMathOperator{\rank}{rank}
\DeclareMathOperator{\Ker}{Ker}
\DeclareMathOperator{\ord}{ord}

\newcommand{\todo}{\textbf{ !TODO! }}
\newcommand{\oldpage}[1]{\marginpar{\footnotesize$\Big\vert$ \textit{p.~#1}}}


%% Document %%

\usepackage{embedall}
\begin{document}

\maketitle
\thispagestyle{fancy}

\renewcommand{\abstractname}{Translator's note.}

\begin{abstract}
  \renewcommand*{\thefootnote}{\fnsymbol{footnote}}
  \emph{This text is one of a series\footnote{\url{https://github.com/thosgood/translations}} of translations of various papers into English.}
  \emph{The translator takes full responsibility for any errors introduced in the passage from one language to another, and claims no rights to any of the mathematical content herein.}
  
  \emph{What follows is a translation (last updated \today) of the French paper:}

  \medskip\noindent
  \textsc{Seshadri, C. S.}
  ``Diviseurs en g\'{e}om\'{e}trie alg\'{e}brique''.
  \emph{S\'{e}minaire Claude Chevalley}, Volume~\textbf{4} (1958-1959), Talk no.~4, 9~p.
  {\footnotesize\url{http://www.numdam.org/item/SCC_1958-1959__4__A4_0/}}
\end{abstract}

\setcounter{footnote}{0}

\tableofcontents
\bigskip


%% Content %%

\oldpage{4-01}
In the first part of this expos\'{e}, we will prove a theorem of Serre on complete varieties \cite{6}, following the methods of Grothendieck \cite{4}.
The second part is dedicated to generalities on divisors.
In the literature, we often call the divisors studied here ``locally principal'' divisors.

The algebraic spaces considered here are defined over an algebraically closed field $K$.
By ``variety'', we mean an irreducible algebraic space.
If $X$ is an algebraic space, we denote by $\sh{O}(X)$, $\sh{R}(X)$, etc. (or simply $\sh{O}$, $\sh{R}$, etc.) the sheaf of local rings, of regular functions, etc. on $X$ (to define $\sh{R}(X)$ we assume that $X$ is a variety).
By ``coherent sheaf'' on $X$, we mean a coherent sheaf of $\sh{O}$-modules on $X$.


\section{Preliminaries}
\label{section1}

\cite{4,5,6}
\medskip

If $M$ is a module over an integral ring $A$ (commutative and with $1$), then we say that an element $m\in M$ is a \emph{torsion element} if there exists some non-zero $a\in A$ such that $a\cdot m=0$.
We say that $M$ is a \emph{torsion module} (resp. \emph{torsion-free module}) if every element of $M$ is a torsion element (resp. if $M\neq0$ and no non-zero element of $M$ is a torsion element).
The torsion elements of $M$ form a torsion submodule of $M$ (denoted by $T(M)$);
if $M\neq0$, then $M/T(M)$ is a torsion-free module.
If $M$ is a torsion module of finite type over $A$, then the ideal $\ann M$ of $A$ (the ideal of $A$ given by the elements $a\in A$ such that $aM=0$) is non-zero.

Let $X$ be an algebraic space and $\sh{F}$ a sheaf of $\sh{O}$-modules on $X$.
We define $\supp\sh{F}$ to be the set of points $x\in X$ such that $\sh{F}_x\neq0$.
If $\sh{F}$ is coherent, then $\supp\sh{F}$ is a closed subset of $X$.
If $X$ is affine, then $\supp\sh{F}$ is the set defined by the ideal $\ann\HH^0(X,\sh{F})$ of the affine algebra $\HH^0(X,\sh{O})$, where $\HH^0(X,\sh{F})$ is considered as a module over $\HH^0(X,\sh{O})$.

A sheaf $\sh{F}$ of $\sh{O}$-modules on a \emph{variety} $X$ is said to be a \emph{torsion sheaf} (resp. \emph{torsion-free sheaf}) if, for every $x\in X$, the module $\sh{F}_x$ over the ring $\sh{O}_x$ is a torsion module (resp. torsion-free module).

\oldpage{4-02}
\begin{proposition}{1}
\label{proposition1}
  If $\sh{F}$ is a coherent sheaf on a variety $X$, then there exists a coherent subsheaf $T(\sh{F})$ of $\sh{F}$ (and only one) such that $(T(\sh{F}))_x = T(\sh{F}_x)$.
\end{proposition}

\begin{proof}
  The uniqueness is trivial.
  The exists is a consequence of the fact that, if $X$ is affine, then $T(\sh{F}_x)$ is given by localisation of the module $T(\HH^0*(X,\sh{F}))$ with respect to the maximal ideal of $\HH^0(X,\sh{O})$ that defines $x$.
\end{proof}

\begin{cor}
  \renewcommand*{\thefootnote}{\fnsymbol{footnote}}
  If $\sh{F}\neq0$ then $\sh{F}/T(\sh{F})$ is a torsion-free coherent sheaf.
  \footnote{\emph{[Translator.] The condition that $\sh{F}\neq0$ is unnecessary, but we include it here since it is in the original. Note that the zero sheaf is indeed a torsion-free sheaf, otherwise any coherent torsion sheaf $\sh{F}$ provides a counterexample to this corollary.}}
\end{cor}

\begin{proposition}{2}
\label{proposition2}
  If $\sh{F}$ is a coherent sheaf on the variety $X$, then $\supp\sh{F}\neq X$ if and only if $\sh{F}$ is a torsion sheaf.
\end{proposition}

\begin{proof}
  This is a trivial consequence of the fact that, if $U$ is an affine open subset, then $\supp\sh{F}\cap U$ is defined by the ideal $\ann\HH^0(U,\sh{F})$ of $\HH^0(U,\sh{O})$, where $\HH^0(U,\sh{F})$ is considered as a module over $\HH^0(U,\sh{O})$.
\end{proof}

\begin{proposition}{3}
\label{proposition3}
  If $\sh{F}$ is a torsion-free coherent sheaf on a variety $X$, with $\sh{F}\subset\sh{R}^n$, then there exists a coherent sheaf $\sh{I}\neq0$ of ideals of $\sh{O}$ such that $\sh{I}\cdot\sh{F}\subset\sh{O}^n$.
\end{proposition}

\begin{proof}
  Let $\sh{I}_x$ be the ideal $[\sh{O}_X^n:\sh{F}_x]$ of $\sh{O}_x$, i.e. the ideal of elements $i_x$ of $\sh{O}_x$ such that $i_x\sh{F}_x\subset\sh{O}_x^n$.
  Since $\sh{F}_x$ is of finite type over $\sh{O}_x$, we know that $\sh{I}_x\neq0$.
  If we take an affine open subset $U$ of $X$, then we can prove that $\sh{I}_x$ is given by localisation of the ideal $[\HH^0(U,\sh{O}^n):\HH^0(U,\sh{F})]$ of $\HH^0(U,\sh{O})$ by the maximal ideal of $\HH^0(U,\sh{O})$ that defines $x$.
  Thus $\{\sh{I}_x\}_{x\in X}$ defines a coherent sheaf $\sh{I}$ of ideals of $\sh{O}$ such that $\sh{I}\cdot\sh{F}\subset\sh{O}^n$.
\end{proof}

Let $\sh{F}$ be a torsion-free coherent sheaf on a variety $X$.
Then the canonical homomorphism $\sh{F}\to\sh{F}\otimes_{\sh{O}}\sh{R}$ is injective.
The sheaves $\sh{R}$ and $\sh{F}\otimes_{\sh{O}}\sh{R}$ are locally constant sheaves, and thus constant (\cite[page~229]{5}).
We can then identify $\sh{F}\otimes_{\sh{O}}\sh{R}$ with a vector space of finite dimension over $\sh{R}$ (we identify the field of rational functions with the sheaf $\sh{R}$ since $\sh{R}$ is constant).
We call this dimension the \emph{rank of $\sh{F}$}, and we can then consider $\sh{F}$ as a subsheaf of $\sh{R}^n$, where $n=\rank\sh{F}$.

\begin{proposition}{4}
\label{proposition4}
  Under the same hypotheses as in \cref{proposition3}, there exists a coherent sheaf $\sh{I}\neq0$ of ideals of $\sh{O}$ such that $\sh{I}\cdot\sh{F}\subset\sh{O}^n$, where $n=\rank\sh{F}$;
  then $\sh{O}^n/(\sh{I}\cdot\sh{F})$ and $\sh{F}/(\sh{I}\cdot\sh{F})$ are torsion sheaves.
\end{proposition}

\begin{proof}
  The proof is immediate.
\end{proof}

\oldpage{4-03}
If $Y$ is a closed subset of an algebraic space $X$, then we denote by $\sh{I}_Y$ the coherent sheaf of ideals of $\sh{O}$ defined by $Y$.

\begin{proposition}{5}
\label{proposition5}
  Let $Y$ be a closed subset of an algebraic space $X$, and $\sh{F}$ a coherent sheaf on $X$, with $\supp\sh{F}\subset Y$;
  then there exists an integer $k$ such that $\sh{I}_Y^k\sh{F}=0$.
\end{proposition}

\begin{proof}
  We can reduce to the case where $X$ is affine, since there exists a finite cover of $X$ by affine opens.
  In this case, the hypothesis implies that the set defined by the ideal $\ann\HH^0(X,\sh{F})$ is contained in $Y$.
  This implies, as is well known, that $\ann\HH^0(X,\sh{F})\supset\sh{I}_Y^k$.
\end{proof}

\begin{proposition}{6}
\label{proposition6}
  Let $\sh{F}$ be a coherent sheaf of fractional ideals on a variety $X$ (i.e. a coherent subsheaf of $\sh{R}$) such that, for every $x$ outside of a closed subset $Y$ of $X$, $\sh{F}_x$ is an ideal of $\sh{O}_x$.
  Then there exists an integer $k$ such that $\sh{I}_Y^k\cdot\sh{F}\subset\sh{O}$.
\end{proposition}

\begin{proof}
  By \cref{proposition3} and the hypothesis, there exists a coherent sheaf $\sh{J}$ of ideals of $\sh{O}$ such that $\sh{J}_x=\sh{O}_x$ if $x\not\in Y$, and such that $\sh{J}\cdot\sh{F}\subset\sh{O}$.
  Thus $\supp(\sh{O}/\sh{J})\subset Y$, and, by \cref{proposition5}, there exists an integer $k$ such that $\sh{I}_Y^k(\sh{O}/\sh{J})=0$.
  This implies that $\sh{I}_Y^k\subset\sh{J}$.
\end{proof}


\section{D\'{e}vissage theorem}
\label{section2}

Let $\cat{C}$ be an abelian category, and $\cat{C}'$ a subcategory of objects of $\cat{C}$.
We say that $\cat{C}'$ is \emph{left exact in $\cat{C}$} if
\begin{enumerate}
  \item every subobject of an object of $\cat{C}'$ is in $\cat{C}'$;
  \item for every exact sequence $0\to\sh{A}'\to\sh{A}\to\sh{A}''\to0$ in $\cat{C}$, the object $\sh{A}$ is in $\cat{C}'$ if the other two objects are in $\cat{C}'$.
  \footnote{The axioms here that define a left-exact subcategory are slightly stronger than those of Grothendieck \cite{4}.}
\end{enumerate}

Let $X$ be an algebraic space.
We denote by $\cat{C}(X)$ the abelian category of coherent sheaves on $X$.
If $Y$ is a closed subset of $X$, then a coherent sheaf on $Y$ has a canonical extension to a coherent sheaf on $X$ (extending by $0$ outside of $Y$), and so we can consider $\cat{C}(Y)$ as a subcategory of $\cat{C}(X)$.
With this notation, we have the following theorem:

\oldpage{4-04}
\begin{theorem}[D\'{e}vissage]
\label{theorem-devissage}
  Let $\cat{D}$ be a left-exact subcategory of $\cat{C}(X)$ that has the following property:
  for every closed irreducible subset $Y$ of $X$, there exists a coherent sheaf $\sh{M}_Y$ of $\cat{C}(Y)$ that belongs to $\cat{D}$, and that is torsion-free as a sheaf on $Y$.
  Then $\cat{D}=\cat{C}(X)$.
\end{theorem}

\begin{proof}
  The proof works by induction on the dimension of $X$.
  If $\dim X=0$, then $X$ consists of a finite number of points $P_1,\ldots,P_r$, and a coherent sheaf on $X$ can be identified with a system $\{N_i\}_{i=1,\ldots,r}$, where $N_i$ is a vector space of finite dimension over $K$.
  Thus the sheaf $\sh{M}_{P_i}$ on $P_i$ that we have, by hypothesis, is a vector space of finite dimension over $K$.
  By the axioms of a left-exact subcategory, it is trivial to show that every system $\{N_i\}_{i=1,\ldots,r}$, where $N_i$ is a vector space of finite dimension over $K$, considered as a coherent sheaf on $X$, belongs to $\cat{D}$.

  Now assume that we have proven the theorem for all dimensions $\leq (n-1)$.
  Let $\dim X=n$.
  Let $Y$ be a closed subset of $X$ such that $\dim Y\leq(n-1)$.
  We can easily show that $\cat{D}\cap\cat{C}(Y)$ is a left-exact subcategory of $\cat{C}(Y)$ that satisfies the hypotheses of the theorem.
  So, by the induction hypothesis, $\cat{D}\supset\cat{C}(Y)$.

  We will now prove that, if $\sh{F}$ is a coherent sheaf on $X$ with $\supp\sh{F}=Y$, then $\sh{F}\in\cat{D}$.
  If $\sh{I}_Y\cdot\sh{F}=0$, then $\sh{F}\in\cat{C}(Y)$, and, by the above, $\sh{F}\in\cat{D}$.
  No matter what, by \cref{proposition5}, there exists an integer $k\geq1$ such that $\sh{I}_Y^k\sh{F}=0$.
  We will complete the proof by induction on $k$.
  Suppose that that claim has been proven for every coherent sheaf $\sh{G}$ on $X$ such that $\sh{I}_Y^{k-1}\sh{G}=0$.
  For $\sh{F}$, we have an exact sequence
  \[
    0 \to \sh{I}_Y\cdot\sh{F} \to \sh{F} \to \sh{F}/(\sh{I}_Y\cdot\sh{F}) \to 0.
  \]
  The sheaf $\sh{I}_Y\sh{F}$ is annihilated by $\sh{I}_Y^{k-1}$, and the sheaf $\sh{F}/(\sh{I}_Y\sh{F})$ is annihilated by $\sh{I}_Y$.
  Thus $\sh{I}_Y\sh{F}$ and $\sh{F}/(\sh{I}_Y\sh{F})$ belong to $\cat{D}$.
  This implies that $\sh{F}\in\cat{D}$.

  Suppose that $X$ is a variety, and that $\sh{F}$ is a torsion-free sheaf on $X$.
  We can consider $\sh{F}$ as a coherent subsheaf of $\sh{R}^n$, where $n=\rank\sh{F}$, and, by \cref{proposition4}, there then exists a coherent sheaf of ideals $\sh{I}$ such that $\sh{I}\cdot\sh{F}\subset\sh{O}^n$, and such that the sheaves $\sh{F}/(\sh{I}\sh{F})$ and $\sh{O}^n/(\sh{I}\sh{F})$ are torsion sheaves.
  Since $\sh{F}/(\sh{I}\sh{F})$ is a torsion sheaf, $\sh{F}/(\sh{I}\sh{F})\in\cat{D}$;
  thus $\sh{F}\in\cat{D}$ if and only if $\sh{I}\sh{F}\in\cat{D}$.
  Analogously, $\sh{I}\sh{F}\in\cat{D}$ if and only if $\sh{O}^n\in\cat{D}$, and, by the axioms of an exact subcategory, if and only if $\sh{O}\in\cat{D}$.
  Thus $\sh{F}\in\cat{D}$ if and only if
\oldpage{4-05}
  $\sh{O}\in\cat{D}$.
  If we repeat the same argument for the torsion-free sheaf $\sh{M}_X$, which we have by hypothesis, then we see that $\sh{O}\in\cat{D}$, which implies that $\sh{F}\in\cat{D}$.

  Suppose again that $X$ is a variety, but now that $\sh{F}$ is an arbitrary coherent sheaf.
  We will show that $\sh{F}\in\cat{D}$.
  We can assume that $\sh{F}\neq0$, and we then have
  \[
    0 \to T(\sh{F}) \to \sh{F} \to \sh{F}/T(\sh{F}) \to 0
  \]
  where $T(\sh{F})$ is a torsion sheaf, and $\sh{F}/T(\sh{F})$ is a torsion-free sheaf.
  By \cref{proposition2}, $\supp T(\sh{F})\neq X$, and, since $X$ is a variety, $\dim\supp T(\sh{F})<\dim T(X)$.
  We then have, by the induction hypothesis, that $T(\sh{F})\in\cat{D}$, and we have just proven that $\sh{F}/T(\sh{F})\in\cat{D}$.
  Thus $\sh{F}\in\cat{D}$.

  Now let $X$ be an arbitrary algebraic space, and $X_1,\ldots,X_p$ its irreducible components.
  If $\sh{F}$ is a coherent sheaf on $X$, then $\sh{F}/(\sh{I}_{X_i}\sh{F})$ can be identified with a sheaf on the variety $X_i$ (where $\sh{I}_{X_i}$ is the sheaf of ideals of $\sh{O}(X)$ determined by $X_i$), and, by the above, $\sh{F}/(\sh{I}_{X_i}\sh{F})\in\cat{D}$.
  Thus the sheaf $\sh{G}=\sum_{i=1}^p\sh{F}/(\sh{I}_{X_i}\sh{F})$ belongs to $\cat{D}$.
  We have a canonical homomorphism $\varphi\colon\sh{F}\to\sh{G}$.
  The image of $\varphi$ is a coherent subsheaf of $\sh{G}$, and so the image of $\varphi$ belongs to $\cat{D}$.

  We know that $\supp\Ker\varphi\subset\bigcup_{i\neq j}X_i\cap X_j$, and so $\dim\supp\Ker\varphi<\dim X$, and, by the induction hypothesis, $\Ker\varphi\in\cat{D}$.
  Thus $\sh{F}\in\cat{D}$, and the theorem is proven.
\end{proof}

\begin{cor}[Serre's Theorem]
  If $\sh{F}$ is a coherent sheaf on a complete algebraic space $X$, then $\HH^0(X,\sh{F})$ is a vector space of finite dimension over $K$.
\end{cor}

\begin{proof}
  We take $\cat{D}$ to be the category of all coherent sheaves $\sh{F}$ on $X$ such that $\HH^0(X,\sh{F})$ is of finite dimension over $K$.
  We can prove that $\cat{D}$ is a left-exact subcategory of $\cat{C}(X)$.
  Also, we know that, if $Y$ is an irreducible closed subset of $X$, then $Y$ is a complete variety.
  Thus the coherent sheaf $\sh{O}(Y)$ on $Y$ is a torsion-free sheaf with the property that $\HH^0(Y,\sh{O}(Y))\cong K$, and so $\HH^0(X,\sh{O}(Y))=\HH^0(Y,\sh{O}(Y))$ is of finite dimension over $K$ (we denote also by $\sh{O}(Y)$ the canonical extension of $\sh{O}(Y)$ to $X$).
  By \hyperref[theorem-devissage]{the theorem}, the corollary is proven.
\end{proof}


\section{Divisors (Generalities)}
\label{section3}

\oldpage{4-06}
Let $X$ be an algebraic variety, and $\sh{R}^\times(X)$ and $\sh{O}^\times(X)$ (or simply $\sh{R}^\times$ and $\sh{O}^\times$) the constant sheaf on $X$ of non-zero rational functions and the sheaf on $X$ of invertible regular functions (respectively).
The sheaves $\sh{R}^\times$ and $\sh{O}^\times$, endowed with their multiplicative structure, are sheaves of abelian groups.

A \emph{divisor} $D$ on $X$ is a section of the quotient sheaf $\sh{R}^\times/\sh{O}^\times$.
An element of $\sh{R}^\times$ that is a representative of the value $D(x)$ of $D$ at $x$ is called a \emph{definition function of $D$ at $x$}.
More generally, a function $f\in\sh{R}^\times$ is called a \emph{definition function of $D$ in an open subset $U$} if, for all $x\in U$, $f$ is a representative of $D(x)$; then $f$ is determined up to an invertible regular function on $U$.
Since we can locally lift a section of $\sh{R}^\times/\sh{O}^\times$ to a section of $\sh{R}^\times$, a divisor $D$ is determined by the following data: a cover $\{U_i\}$ by open subsets, and non-zero rational functions $f_i$ on $U_i$ such that, on $U_i\cap U_j$, $f_{ij}=f_i/f_j$ is an invertible regular function.
We have that $f_{ij}f_{jk}f_{ki}=1$ on $U_i\cap U_j\cap U_k$, and, as is well known, this allows us to construct a locally trivial fibre bundle with $K^\times$ as the structure group;
it is easy to see that this fibre bundle is determined up to equivalence \cite{7}.
We also know that the coherent sheaves of fractional ideals (i.e. the coherent subsheaves of $\sh{R}$) that are generated by $f_i$ and $f_j$ (respectively) agree on $U_i\cap U_j$, and do not depend on the choice of definition functions of $D$ in $U_i$ and $U_j$.
This implies that the divisor $D$ canonically determines a coherent sheaf of locally principal fractional ideals.
We can easily see that the converse is true, and this gives us an equivalent definition of a divisor \cite{1}.

A divisor $D$ is said to be \emph{positive} if, for each $x\in X$, $D(x)\in\sh{O}_x/\sh{O}_x^\times$ (i.e. if all the definition functions of $D$ at $x$ are regular functions in $x$).

Since $\sh{R}^\times/\sh{O}^\times$ is a sheaf of abelian groups on $X$, there is a canonical structure of an abelian group on the set of divisors on $X$;
this group is called the \emph{group of divisors on $X$}.
The composition law in this group is written additively, and the identity element in this group is thus called the \emph{zero divisor}, and is denoted by $(0)$.

If $f$ is a non-zero rational function on $X$, then it defines a divisor $\div f$ by the data $(\div f)(x)=\Im f\subseteq\sh{R}^\times/\sh{O}_x^\times$.
The divisors
\oldpage{4-07}
obtained in this way are called \emph{principal divisors}, and form a subgroup of the group of divisors on $X$;
the quotient group is called the \emph{group of classes of divisors on $X$}.
Two divisors $D_1$ and $D_2$ are said to be equivalent if they are equivalent module the group of principal divisors;
we write $D_1\sim D_2$.
We have seen that a divisor defines, up to equivalence, a locally trivial algebraic fibre bundle with structure group $K^\times$.
On the other hand, it is easy to see that a locally trivial algebraic fibre bundle with $K^\times$ as its structure group defines, up to equivalence, a divisor \cite{7}.
Thus the group of classes of divisors on $X$ is equal to $\HH^1(X,\sh{O}^\times)$, the group of classes of equivalent algebraic fibre bundles with $K^\times$ as their structure group.

We can define, in an analogous way, an \emph{additive divisor} on a variety $X$ as a section of the sheaf $\sh{R}/\sh{O}$ (the divisors defined above are called \emph{multiplicative divisors}, or simply \emph{divisors}).
The additive divisors form an abelian group, and even a vector space over $K$.
An additive divisor is determined by the following data: a cover $\{U_i\}$ of $X$ by open subsets, and rational functions $f_i$ on $U_i$ such that $f_{ij}=f_i-f_j$ is a regular function on $U_i\cap U_j$.
We can define, as for (multiplicative) divisors, the notions of definition functions of an additive divisor, equivalence between two additive divisors, etc.
We find, for example, that $\HH^1(X,\sh{O})$ is equal to the group of classes of additive divisors on $X$.

Let $D$ be a (multiplicative) divisor on $X$.
We define $\supp D$ to be the set of points $x\in X$ such that $D(x)$ is not the identity element in $\sh{R}^\times/\sh{O}_x^\times$, i.e. such that every definition function of $D$ at $x$ is either not defined at $x$ or takes the value $0$ at $x$.

\begin{proposition}{7}
\label{proposition7}
  The support of a divisor $D$ on a variety $X$ is a closed subset $\neq X$ of $X$, and $D=0$ if and only if the support is empty.
\end{proposition}

\begin{proof}
  The latter claim is trivial.
  For the former, we prove that the set $E$ of points $x\in X$ such that every definition function of $D$ at $x$ belongs to $\sh{O}_x^\times$ is a non-empty open subset;
  indeed, if we take a definition function $g$ of $D$ at $x$, then it is also a definition function of $D$ in an open subset $U$ that contains $x$.
  By hypothesis, if $x\in E$, then $g$ is regular at $x$ and $g(x)\neq0$, and we can choose $U$ such that $g$ is regular and invertible on $U$, which proves that $E$ is open.
\end{proof}

\oldpage{4-08}
\begin{proposition}{8}
\label{proposition8}
  If $D$ is a divisor on a normal variety $X$, then $\supp D$ is a union of hypersurfaces (i.e. of closed subvarieties of codimension $1$).
\end{proposition}

\begin{proof}
  If $f$ is a function on a normal variety $Y$, we know that, if $f$ is not defined at $x\in Y$, then $x$ belongs to a variety of poles or zeros of $f$ (i.e. to an irreducible component of the closure of the set of points $x\in Y$ such that $f(x)\in\{0,\infty\}$).
  So, if we take $f$ to be a definition function for $D$ in some open subset $U\subset X$, then $\supp D\cap U$ is the union of the pole and zero varieties of $f$ in $U$, and we know that these varieties are of codimension $1$ (\cite[chapitre~III]{2}).
\end{proof}

\begin{remark}
  If $X$ is not normal, then the support of a divisor $D$ on $X$ is not necessarily of codimension $1$.
  It is easy to define an affine variety $X$ of dimension $>1$ that is normal everywhere except at a single point $x_0$ (for example, the point $(a,ab,b^2,b^3)$ in the four-dimensional space $K^4$).
  There exists a function $u$ that is everywhere defined on $X$, and which is entire on the local ring of $x_0$, but which is not contained in this ring;
  by adding, if necessary, a constant, we can assume that $x_0$ is not a zero of $u$.
  There is then an open neighbourhood $X'$ of $x_0$ such that the divisor of the function induced by $u$ on $X'$ has support equal to the single point $x_0$.
\end{remark}

Suppose that $X$ is a normal variety, and $D$ is a divisor on $X$.
Let $S$ be a hypersurface of $X$.
If $f$ is a definition function of $D$ at $x\in S$, then the order of $f$ on $S$ (\cite{2}) does not depend on the choice of $f$, nor on $x\in S$.
We denote this integer by $\ord_S D$.
It is easy to see that $\ord_S D=0$ if and only if $S\not\subset\supp D$.
If we now take the formal combination $C=\sum_S(\ord_S D)\cdot S$, where $S$ runs over the set of all hypersurfaces of $X$, then $C$ is a cycle of codimension $1$, and we call it the \emph{associated cycle} of the divisor $D$.

\begin{proposition}{9}
\label{proposition9}
  Let $X$ be a normal variety.
  The map that sends a divisor $D$ to its associated cycle of codimension~$1$ is an injective homomorphism from the group of divisors on $X$ to the group of cycles of codimension~$1$.
\end{proposition}

\begin{proof}
  The proof is trivial.
\end{proof}

\begin{proposition}{10}
\label{proposition10}
  If $X$ is further a non-singular variety, then the homomorphism that sends a divisor to its associated cycle is bijective.
\end{proposition}

\oldpage{4-09}
\begin{proof}
  It suffices to show that, for every hypersurface $S$, there exists a divisor $D$ such that the cycle $1\cdot S$ is the cycle associated to $D$.
  Since $X$ is non-singular, for every $x\in X$, the local ring $\sh{O}_x$ is factorial \cite{3};
  thus, for every $x\in S$, $S$ is defined by one single equation in a neighbourhood of $x$.
  So there exists a cover $\{U_i\}_{i=1,\ldots,p}$ of $S$ by open subsets $U_i$ of $X$, and, for each $i$, a regular function $f_i$ on $U_i$ that is non-zero outside of $U_i\cap S$ in $U_i$ with $\ord_S f_i=1$.
  It then follows that $f_i/f_j$ is an invertible regular function in $U_i\cap U_j$.
  Now take the cover $\{U_i\}_{i=0,\ldots,p}$, where $U_0=CS$, and take $f_0=1$.
  IT is easy to see that the divisor $D$ for which $f_i$ is a definition function of $D$ on $U_i$ is such that the cycle associated to $D$ is $1\cdot S$.
  So the proposition is proven.
\end{proof}

\begin{remark}
  \hyperref[proposition10]{The proposition} is not necessarily true if $X$ is non-singular.
  For example, for the cone $xy-zw=0$ in $K^4$, the cycle defined by $x=z=0$ is not a cycle associated to any divisor.
\end{remark}


%% Bibliography %%

\nocite{*}
\bibliographystyle{acm}

\begin{thebibliography}{10}

  \bibitem{1}
  {\sc Cartier, P.}
  \newblock Diviseurs et d\'{e}rivations en g\'{e}om\'{e}trie alg\'{e}brique.
  \newblock {({\em Th\`{e}se Sc. math. Paris.} 1958)}
  \newblock (To appear in {\em Bull. Soc. math. France}).

  \bibitem{2}
  {\sc Chevalley, C.}
  \newblock Fondements de la G\'{e}om\'{e}trie alg\'{e}brique
  \newblock Paris, Secr\'{e}tariat math\'{e}matique, 1958, multigraphed.
  \newblock (Class taught at the Sorbonne in 1957--58).

  \bibitem{3}
  {\sc Godement, R.}
  \newblock Propri\'{e}t\'{e}s analytiques des localit\'{e}s
  \newblock In {\em S\'{e}minarie Cartan-Chevalley}, vol.~8, 1955--56.
  \newblock (Talk number 19).

  \bibitem{4}
  {\sc Grothendieck, A.}
  \newblock Sur les faisceaux alg\'{e}briques et les faisceaux analytiques
    coh\'{e}rents.
  \newblock In {\em S\'{e}minaire H. Cartan}, vol.~9.
  \newblock (Talk number 2).

  \bibitem{5}
  {\sc Serre, J.-P.}
  \newblock Faisceaux alg\'{e}briques coh\'{e}rents.
  \newblock {\em Ann. Math. (2)\/}, vol.~61 (1955), 197--279.

  \bibitem{6}
  {\sc Serre, J.-P.}
  \newblock Sur la cohomologie des vari\'{e}t\'{e}s alg\'{e}briques.
  \newblock {\em J. Math. pures et appl. (9)\/}, vol.~36 (1957), 1--16.

  \bibitem{7}
  {\sc Weil, A.}
  \newblock Fibre spaces in algebraic geometry
  \newblock (Notes taken by A. Wallace, 1952)
  \newblock Chicago, University of Chicago, 1955.

\end{thebibliography}

\end{document}
