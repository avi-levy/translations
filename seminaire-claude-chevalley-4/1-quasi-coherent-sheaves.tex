\documentclass{article}

\usepackage{amssymb,amsmath}

\usepackage{hyperref}
\usepackage[nameinlink]{cleveref}
\usepackage{enumerate}

\usepackage{mathrsfs}
%% Fancy fonts --- feel free to remove! %%
\usepackage{Baskervaldx}
\usepackage{mathpazo}


\crefname{section}{Section}{Sections}
\crefname{equation}{}{}

%% Theorem environments %%

\usepackage{amsthm}

  \theoremstyle{plain}

  \newtheorem{innercustomtheorem}{Theorem}
  \crefname{innercustomtheorem}{Theorem}{Theorems}
  \newenvironment{theorem}[1]
    {\renewcommand\theinnercustomtheorem{#1}\innercustomtheorem}
    {\endinnercustomtheorem}

  \newtheorem{innercustomcorollary}{Corollary}
  \crefname{innercustomcorollary}{Corollary}{Corollaries}
  \newenvironment{corollary}[1]
    {\renewcommand\theinnercustomcorollary{#1}\innercustomcorollary}
    {\endinnercustomcorollary}

  \newtheorem*{proposition}{Proposition}


%% Shortcuts %%

\newcommand{\sh}{\mathscr}
\newcommand{\cat}{\mathcal}

\renewcommand{\geq}{\geqslant}
\renewcommand{\leq}{\leqslant}

\newcommand{\todo}{\textbf{ !TODO! }}
\newcommand{\oldpage}[1]{\marginpar{\footnotesize$\Big\vert$ \textit{p.~#1}}}


%% Document %%

\usepackage{embedall}
\begin{document}

\renewcommand{\abstractname}{Translator's note.}

\title{Quasi-coherent sheaves}
\author{P. Gabriel}
\date{}
\maketitle

\begin{abstract}
  \renewcommand*{\thefootnote}{\fnsymbol{footnote}}
  \emph{This text is one of a series\footnote{\url{https://github.com/thosgood/translations}} of translations of various papers into English.}
  \emph{The translator takes full responsibility for any errors introduced in the passage from one language to another, and claims no rights to any of the mathematical content herein.}
  
  \emph{What follows is a translation (last updated \today) of the French paper:}

  \medskip\noindent
  \textsc{Gabriel, P.}
  ``Faisceaux quasi-coh\'{e}rents''.
  \emph{S\'{e}minaire Claude Chevalley}, Volume~\textbf{4} (1958-1959), Talk no.~1, 12~p.
  {\footnotesize\url{http://www.numdam.org/item/SCC_1958-1959__4__A1_0/}}
\end{abstract}

\setcounter{footnote}{0}

\tableofcontents


%% Content %%

\bigskip\bigskip
Lorem ipsum dolor sit amet, consectetur adipisicing elit, sed do eiusmod
tempor incididunt ut labore et dolore magna aliqua. Ut enim ad minim veniam,
quis nostrud exercitation ullamco laboris nisi ut aliquip ex ea commodo
consequat. Duis aute irure dolor in reprehenderit in voluptate velit esse
cillum dolore eu fugiat nulla pariatur. Excepteur sint occaecat cupidatat non
proident, sunt in culpa qui officia deserunt mollit anim id est laborum.


%% Bibliography %%

\nocite{*}
\bibliographystyle{acm}
\begin{thebibliography}{10}

  \bibitem{1}
  {\sc Cartan, H.}
  \newblock Vari\'{e}t\'{e}s alg\'{e}briques affines.
  \newblock {\em S\'{e}minaire Cartan-Chevalley, Volume~8} (1955/56), Talk no.~3.

  \bibitem{2}
  {\sc Godement, R.}
  \newblock ``Topologie alg\'{e}brique et th\'{e}orie des faisceaux''.
  \newblock {Paris, Hermann.}
  \newblock {\em Act. scient. et ind. 1252} (1958)

  \bibitem{3}
  {\sc Krull, W.}
  \newblock Jacobsonsche Rings, Hilbertscher Nullstellensatz, Dimensionstheorie.
  \newblock {\em Math. Z. 54} (1951), pp.~354--387

  \bibitem{4}
  {\sc Serre, J.-P.}
  \newblock Faisceaux alg\'{e}briques coh\'{e}rents.
  \newblock {\em Ann. Math. 61\/} (1955), pp.~197--279.

\end{thebibliography}

\end{document}
