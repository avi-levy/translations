\documentclass{article}

\usepackage{amssymb,amsmath}

\usepackage{hyperref}
\usepackage[nameinlink]{cleveref}
\usepackage{enumerate}

\usepackage{mathrsfs}
%% Fancy fonts --- feel free to remove! %%
\usepackage{Baskervaldx}
\usepackage{mathpazo}


\crefname{section}{Section}{Sections}
\crefname{equation}{}{}

%% Theorem environments %%

\usepackage{amsthm}

  \theoremstyle{plain}

  \newtheorem{innercustomtheorem}{Theorem}
  \crefname{innercustomtheorem}{Theorem}{Theorems}
  \newenvironment{theorem}[1]
    {\renewcommand\theinnercustomtheorem{#1}\innercustomtheorem}
    {\endinnercustomtheorem}

  \newtheorem{innercustomcorollary}{Corollary}
  \crefname{innercustomcorollary}{Corollary}{Corollaries}
  \newenvironment{corollary}[1]
    {\renewcommand\theinnercustomcorollary{#1}\innercustomcorollary}
    {\endinnercustomcorollary}

  \newtheorem*{proposition}{Proposition}


%% Shortcuts %%

\newcommand{\sh}{\mathscr}
\newcommand{\cat}{\mathcal}

\renewcommand{\geq}{\geqslant}
\renewcommand{\leq}{\leqslant}

\newcommand{\todo}{\textbf{ !TODO! }}
\newcommand{\oldpage}[1]{\marginpar{\footnotesize$\Big\vert$ \textit{p.~#1}}}


%% Document %%

\usepackage{embedall}
\begin{document}

\renewcommand{\abstractname}{Translator's note.}

\title{Quasi-coherent sheaves}
\author{P. Gabriel}
\date{}
\maketitle

\begin{abstract}
  \renewcommand*{\thefootnote}{\fnsymbol{footnote}}
  \emph{This text is one of a series\footnote{\url{https://github.com/thosgood/translations}} of translations of various papers into English.}
  \emph{The translator takes full responsibility for any errors introduced in the passage from one language to another, and claims no rights to any of the mathematical content herein.}
  
  \emph{What follows is a translation (last updated \today) of the French paper:}

  \medskip\noindent
  \textsc{Gabriel, P.}
  ``Faisceaux quasi-coh\'{e}rents''.
  \emph{S\'{e}minaire Claude Chevalley}, Volume~\textbf{4} (1958-1959), Talk no.~1, 12~p.
  {\footnotesize\url{http://www.numdam.org/item/SCC_1958-1959__4__A1_0/}}
\end{abstract}

\setcounter{footnote}{0}

\tableofcontents


%% Content %%

\bigskip\bigskip
We assume knowledge of the definitions and elementary properties of sheaves of modules on a topological space, i.e. \cite[chapitre~I, \S1; chapitre~II, \S\S1--2]{2}.
We define a presheaf $\sh{P}$ on a base $\sh{B}$ of open subsets of a topological space $X$, with values in a category $\cat{C}$, to be the following data:
\begin{enumerate}[(a)]
  \item for every open subset $U$ in $\sh{B}$, an object $\sh{P}(U)$ of $\cat{C}$, that we may also denote by $\Gamma(U,\sh{P})$;
  \item for every pair $(U,V)$ of open subsets $U$ in $\sh{B}$ such that $U\subset V$, a morphism $\rho_{UV}\colon\sh{P}(V)\to\sh{P}(U)$.
    The morphism $\rho_{UV}$ will be called the restriction of $V$ to $U$.
    We further suppose that, if $U\subset V\subset W$ are open subsets in $\sh{B}$, then $\rho_{UV}\circ\rho_{VW}=\rho_{UW}$.
\end{enumerate}

The construction of the sheaf $\widetilde{\sh{P}}$ associated to a presheaf $\sh{P}$ can be easily generalised to the case of \todo


%% Bibliography %%

\nocite{*}
\bibliographystyle{acm}
\begin{thebibliography}{10}

  \bibitem{1}
  {\sc Cartan, H.}
  \newblock Vari\'{e}t\'{e}s alg\'{e}briques affines.
  \newblock {\em S\'{e}minaire Cartan-Chevalley, Volume~8} (1955/56), Talk no.~3.

  \bibitem{2}
  {\sc Godement, R.}
  \newblock ``Topologie alg\'{e}brique et th\'{e}orie des faisceaux''.
  \newblock {Paris, Hermann.}
  \newblock {\em Act. scient. et ind. 1252} (1958)

  \bibitem{3}
  {\sc Krull, W.}
  \newblock Jacobsonsche Rings, Hilbertscher Nullstellensatz, Dimensionstheorie.
  \newblock {\em Math. Z. 54} (1951), pp.~354--387

  \bibitem{4}
  {\sc Serre, J.-P.}
  \newblock Faisceaux alg\'{e}briques coh\'{e}rents.
  \newblock {\em Ann. Math. 61\/} (1955), pp.~197--279.

\end{thebibliography}

\end{document}
