\documentclass{article}

\usepackage{amssymb,amsmath}

\usepackage{hyperref}
\usepackage[nameinlink]{cleveref}
\usepackage{enumerate}
\usepackage{tikz-cd}

\usepackage{mathrsfs}
\usepackage[frak=euler]{mathalfa}
%% Fancy fonts --- feel free to remove! %%
\usepackage{Baskervaldx}
\usepackage{mathpazo}


\crefname{section}{Section}{Sections}
\crefname{equation}{}{}

%% Theorem environments %%

\usepackage{amsthm}

  \theoremstyle{plain}

  \newtheorem{innercustomtheorem}{Theorem}
  \crefname{innercustomtheorem}{Theorem}{Theorems}
  \newenvironment{theorem}[1]
    {\renewcommand\theinnercustomtheorem{#1}\innercustomtheorem}
    {\endinnercustomtheorem}

  \newtheorem{innercustomcorollary}{Corollary}
  \crefname{innercustomcorollary}{Corollary}{Corollaries}
  \newenvironment{corollary}[1]
    {\renewcommand\theinnercustomcorollary{#1}\innercustomcorollary}
    {\endinnercustomcorollary}

  \newtheorem*{proposition}{Proposition}


  \theoremstyle{definition}

  \newtheorem*{remark}{Remark}


%% Shortcuts %%

\newcommand{\sh}{\mathscr}
\newcommand{\cat}{\mathcal}

\renewcommand{\geq}{\geqslant}
\renewcommand{\leq}{\leqslant}

\DeclareMathOperator{\HH}{H}
\DeclareMathOperator{\Hom}{Hom}
\DeclareMathOperator{\Ker}{Ker}
\DeclareMathOperator{\Coker}{Coker}

\newcommand{\todo}{\textbf{ !TODO! }}
\newcommand{\oldpage}[1]{\marginpar{\footnotesize$\Big\vert$ \textit{p.~#1}}}


%% Document %%

\usepackage{embedall}
\begin{document}

\renewcommand{\abstractname}{Translator's note.}

\title{Quasi-coherent sheaves}
\author{P. Gabriel}
\date{}
\maketitle

\begin{abstract}
  \renewcommand*{\thefootnote}{\fnsymbol{footnote}}
  \emph{This text is one of a series\footnote{\url{https://github.com/thosgood/translations}} of translations of various papers into English.}
  \emph{The translator takes full responsibility for any errors introduced in the passage from one language to another, and claims no rights to any of the mathematical content herein.}
  
  \emph{What follows is a translation (last updated \today) of the French paper:}

  \medskip\noindent
  \textsc{Gabriel, P.}
  ``Faisceaux quasi-coh\'{e}rents''.
  \emph{S\'{e}minaire Claude Chevalley}, Volume~\textbf{4} (1958-1959), Talk no.~1, 12~p.
  {\footnotesize\url{http://www.numdam.org/item/SCC_1958-1959__4__A1_0/}}
\end{abstract}

\setcounter{footnote}{0}

\tableofcontents


%% Content %%

\bigskip\bigskip
\oldpage{1-01}
We assume prior knowledge of the definitions and elementary properties of sheaves of modules on a topological space, i.e. \cite[chapitre~I, \S1; chapitre~II, \S\S1--2]{2}.
We define a presheaf $\sh{P}$ on a base $\mathscr{B}$ of open subsets of a topological space $X$, with values in a category $\cat{C}$, to be the following data:
\begin{enumerate}[(a)]
  \item for every open subset $U$ in $\mathscr{B}$, an object $\sh{P}(U)$ of $\cat{C}$, that we may also denote by $\Gamma(U,\sh{P})$;
  \item for every pair $(U,V)$ of open subsets $U$ in $\mathscr{B}$ such that $U\subset V$, a morphism $\rho_{UV}\colon\sh{P}(V)\to\sh{P}(U)$.
    The morphism $\rho_{UV}$ will be called the restriction of $V$ to $U$.
    We further suppose that, if $U\subset V\subset W$ are open subsets in $\mathscr{B}$, then $\rho_{UV}\circ\rho_{VW}=\rho_{UW}$.
\end{enumerate}

The construction of the sheaf $\widetilde{\sh{P}}$ associated to a presheaf $\sh{P}$ can be easily generalised to the case of presheaves on a base of open subsets.
If $\mathfrak{U}=(U_i)_{i\in I}$ is a cover of the open subset $U\in\mathscr{B}$, where $U_i\in\mathscr{B}$, and $U_i\cap U_j\in\mathscr{B}$, and if $\sh{P}$ is a presheaf of rings (resp. of modules) on the base $\mathscr{B}$, then we write $\HH^0(\mathfrak{U},\sh{P})$ to mean the subring (resp. submodule) of $\prod_{i\in I}\sh{P}(U_i)$ given by the $(X_i)_{i\in I}$ such that $\rho_{U_i\cap U_j,U_i}(X_i) = \rho_{U_i\cap U_j,U_j}(X_j)$ for every pair $i,j\in I$.
We then have canonical maps:
\[
  \sh{P}(U) \to \HH^0(\mathfrak{U},\sh{P}) \to \widetilde{\sh{P}}(U);
\]
if these maps are injective for every cover $\mathfrak{U}$ satisfying the conditions above, then $\widetilde{\sh{P}}(U)$ is the union of the $\HH^0(\mathfrak{U},\sh{P})$.


\section{Preliminaries on localisation}
\label{section1}

Let $A$ be a unital commutative ring.
A submonoid $S$ of the multiplicative monoid of $A$ (i.e. a non-empty subset of $A$ such that, if it contains $s$ and $t$, then it contains $s\cdot t$) is said to be \emph{complete} if it satisfies the following condition:
if $s\cdot t\in S$, then $s\in S$ and $t\in S$.
To every multiplicative submonoid
\oldpage{1-02}
$S$, we associate a complete monoid $\widetilde{S}$ in the following way:
$s\in\widetilde{S}$ if and only if there exists some $t$ in $A$ such that $s\cdot t\in S$.
The prime ideals that meet $S$ also meet $\widetilde{S}$, and vice versa.
Furthermore, the complement of $\widetilde{S}$ in $A$ is a union of prime ideals (the ideals that are maximal amongst those that do not meet $\widetilde{S}$ are prime).

If now $M$ denotes a unital $A$-module, and $S$ a multiplicative submonoid of $A$, then we denote by $M_S$ the following abelian group:

The set $M_S$ is the quotient of $M\times S$ by the equivalence relation
\[
  (m,s) = (n,t) \iff \exists r\in S\mbox{ such that }r(mt-ns)=0.
\]

The addition in $M\times S$ is defined by $(m,s)+(n,t)=(mt+ns,st)$;
our equivalence relation is compatible with the addition, whence we get the structure of an abelian group on $M_S$.
We denote by $m/s$ the class of $(m,s)$ in $M_S$.

We have a bilinear map $A_S\otimes_\mathbb{Z}M_S\to M_S$ defined by passing to quotients from the map $((a,s),(m,t))\mapsto(am,st)$.
Taking $M=A$, we see that $A_S$ is a ring and, more generally, $M_S$ is an $A_S$-module.
The map $\psi\colon m\mapsto m/1$ from $M$ to $M_S$ is compatible with the ring homomorphism $\varphi\colon a\mapsto a/1$ from $A$ to $A_S$ (i.e. it is a homomorphism of abelian groups such that $\psi(a\cdot m)=\varphi(a)\cdot\psi(m)$).

We can furthermore easily prove the following claims:
the correspondence $M\mapsto M_S$ is functorial (in the evident way);
the functor $M\mapsto M_S$ is exact (i.e. if $0\to M\to M'\to M''\to 0$ is an exact sequence of $A$-modules, then $0\to M_S\to M'_S\to M''_S\to 0$ is an exact sequence of $A$-modules) and commutes with inductive limits (i.e. if $(M_i)_{i\in I}$ is an inductive system of $A$-modules, and $L$ an inductive limit of this system, then the $(M_i)_S$ form an inductive system of $A_S$-modules, and the homomorphisms $(M_i)_S\to L_S$ induced by the $M_i\to L$ define $L_S$ as an inductive limit of the system $((M_i)_S)$); the canonical map $M\otimes_A A_S\to M_S$ is bijective.

If $S$ and $T$ are submonoids, then the abelian groups $(M_S)_T$, $(M_T)_S$, and $M_{S\cdot T}$ are all canonically isomorphic to one another, where $S\cdot T$ denotes the monoid given by products $s\cdot t$ with $s\in S$ and $t\in T$.
Identifying the rings $(A_S)_T$, $(A_T)_S$, and $A_{S\cdot T}$, which are themselves all canonically isomorphic to one another, the isomorphisms between $(M_S)_T$, $(M_T)_S$, and $M_{S\cdot T}$ are isomorphisms of modules.

\oldpage{1-03}
If $S$ is contained in a submonoid $S'$ of the multiplicative monoid of $A$, then there is a canonical map from $M_S$ to $M_{S'}$ that sends each element $m_s$ of $M_S$ (where $m\in M$ and $s\in S$) to the element of $M_{S'}$ denoted by the same symbol.
The map $A_S\to A_{S'}$ is a ring homomorphism;
the map $M_S\to M_{S'}$ is that which corresponds to the map $M\otimes_A A_S\to M\otimes_A A_{S'}$, induced by $A_S\to A_{S'}$.
If $S'=\widetilde{S}$ is the smallest complete monoid containing $S$, then the map $M_S\to M_{\widetilde{S}}$ is bijective.
Finally, if $S$ is the union of an increasingly-ordered filtered family of monoids $S_i$, then $M_S$ can be identified with $\varinjlim M_{S_i}$.


\section{The prime spectrum of a commutative ring}
\label{section2}

Let $A$ be a unital commutative ring, and $V(A)$ the set of prime ideals of $A$.
If $\mathfrak{a}$ is an ideal of $A$, then we denote by $W(\mathfrak{a})$ the set of prime ideals that contain $\mathfrak{a}$, and $U(\mathfrak{a})=V(A)\setminus W(\mathfrak{a})$.

Then
\begin{gather*}
  W(\mathfrak{a}) \subset V(A), \quad U(\mathfrak{a})\subset V(A),
\\W(\sum\mathfrak{a}_i) = \bigcap_i W(\mathfrak{a}_i),
\quad W(\mathfrak{a}\cap\mathfrak{b}) = W(\mathfrak{a}\cdot\mathfrak{b}) = W(\mathfrak{a})\cup W(\mathfrak{b}),
\\U(\sum\mathfrak{a}_i) = \bigcup_i U(\mathfrak{a}_i),
\quad U(\mathfrak{a}\cap\mathfrak{b}) = U(\mathfrak{a}\cdot\mathfrak{b}) = U(\mathfrak{a})\cap U(\mathfrak{b}).
\end{gather*}

The set $U(\mathfrak{a})$ increases with $\mathfrak{a}$ and $W(\mathfrak{a})$ decreases when $\mathfrak{a}$ increases.
Finally, $W(\mathfrak{a})=W(\mathfrak{b})$ if and only if every element of $\mathfrak{a}$ has a power in $\mathfrak{b}$, and vice versa:
indeed, to say that $W(\mathfrak{a})\subset W(\mathfrak{b})$ is to say that every prime ideal that contains $\mathfrak{a}$ also contains $\mathfrak{b}$, i.e. that $\mathfrak{b}$ is contained in the intersection of the prime ideals containing $\mathfrak{a}$, and it is classical that this latter intersection consists of the elements that have a power in $\mathfrak{a}$.

The sets $U(\mathfrak{a})$ are the open subsets of a topology on $V(A)$, and, endowed with this topology, $V(A)$ is called the \emph{prime spectrum of $A$}.

If the ideal $\mathfrak{a}$ is generated by the $(f_i)$, then $U(\mathfrak{a})$ is the union of the $U((f_i))=U_{f_i}$;
since $U_f\cap U_g=U_{fg}$, the $U_f$ thus form a base of open subsets when $f$ runs over the elements of $A$.
Every open subset of the form $U_f$ is said to be \emph{special}.
Every special open subset is \emph{quasi-compact}:
indeed, if $U_f$ is the union of some $U(\mathfrak{a}_i)$, then $U_f=\bigcup_i U(\mathfrak{a}_i)=U(\sum\mathfrak{a}_i)$, and a power $f^n$ of $f$ belongs to $\sum_i\mathfrak{a}_i$, and thus to the sum of a finite number of the $\mathfrak{a}_i$.

In particular, $V=U_i$ is quasi-compact.
Finally, if the ring $A$ is \emph{Noetherian}, then every increasing sequence of ideals stabilises, and the same is true for every
\oldpage{1-04}
increasing sequence of open subsets.
The prime spectrum is thus a \emph{Zariski topological space}.


\section{Quasi-coherent sheaves on $V(A)$}
\label{section3}

Now let $M$ be a unital module over the ring $A$.
We are going to define a \emph{presheaf $\mathfrak{M}^*$ on the base of special open subsets of $V(A)$}.
For this, we will associate, to each $f\in A$, the multiplicatively-stable system $S_f$ given by the complement in $A$ of the union of the prime ideals of $U_f$.
The system $S_f$ is the complete multiplicatively-stable system generated by $f$.
We then have the equivalence:
\[
  S_f \subset S_g \mapsto U_f \supset U_g.
\]

With this in mind, we will define the presheaf $\mathfrak{M}^*$ by the formulas $\mathfrak{M}^*(U_f)=M_{S_f}$, which we will also denote by $\mathfrak{M}_f$;
if $U_f\supset U_g$, then the restriction map $\rho\colon\mathfrak{M}^*(U_f)\to\mathfrak{M}^*(U_g)$ is the canonical map from $\mathfrak{M}_{S_f}$ to $\mathfrak{M}_{S_g}$.
These definitions clearly do not depend on the elements $f$ and $g$ that define $U_f$ and $U_g$;
furthermore, the axioms of a presheaf are satisfied, thanks to the properties of localisation.

We will denote by $\mathfrak{M}$ the sheaf associated to the presheaf $\mathfrak{M}^*$.
Since the $A_f$ are rings, and the $M_f$ are $A_f$-modules, $\mathfrak{A}^*$ is a presheaf of rings, $\mathfrak{M}^*$ a presheaf of $\mathfrak{A}^*$-modules, $\mathfrak{A}$ a sheaf of rings, and $\mathfrak{M}$ a sheaf of $\mathfrak{A}$-modules.
Furthermore, the functors that send $M$ to $\mathfrak{M}^*$ and $\mathfrak{M}$ are exact, since the functor that sends $M$ to the $M_f$ is exact.
We define an \emph{algebraic sheaf} on $V(A)$ to be any sheaf of modules over the sheaf of rings $\mathfrak{A}$.
We define a \emph{quasi-coherent sheaf} on $V(A)$ to be any sheaf isomorphic to a sheaf of the form $\mathfrak{A}(M)$ for some $M$.
We will first be interested in the sections of such a sheaf over a special open subset $U_f$.

For this, we note that the prime ideals of $A_f$ are the images under the map $A\mapsto A_f$ of the prime ideals of $A$ that do not contain $f$.
Furthermore, the canonical application thus defined from the spectrum $V(A_f)$ of $A_f$ to $V(A)$ is a homomorphism from $V(A_f)$ to $U_f$.
Finally, the sheaf on $V(A_f)$ associated to the module $M_f$ can be identified with the restriction of $\mathfrak{M}$ to $U_f$.

More generally, if $S$ is a multiplicatively-stable system (that we can assume to be complete) of $A$, then let $E_S$ be the topological subspace of $V(A)$ consisting of the prime ideals that do not meet $S$.
The canonical map $A\mapsto A_S$ induces a homomorphism from $V(A_S)$ to $E_S$;
if $S$ is generated by the $f_i$,
\oldpage{1-05}
then $E_S$ is the intersection of the $U_{f_i}$, and, conversely, every intersection of special open subsets of $V(A)$ is of the form $E_S$.
We will see that the restriction of $\mathfrak{M}$ to the subspace $E_S$ can be identified with the sheaf $\mathfrak{M}_S$ associated to the $A_S$-module $M_S$.

For every prime ideal $\mathfrak{p}$, we denote by $A_\mathfrak{p}$ and $M_\mathfrak{p}$ the ring and the module obtained by localisation of the multiplicative system $S=A\setminus\mathfrak{p}$.

\begin{theorem}{1}
\label{theroem1}
  \begin{enumerate}[(a)]
    \item If $\mathfrak{p}$ is a point of $V(A)$, then the localised ring $\mathfrak{A}_\mathfrak{p}$ (the fibre of the sheaf $\mathfrak{A}$ over $\mathfrak{p}$) is canonically isomorphic to $A_\mathfrak{p}$;
      Identifying $\mathfrak{A}_\mathfrak{p}$ with $A_\mathfrak{p}$, the localised module $\mathfrak{M}_\mathfrak{p}$ of the sheaf $\mathfrak{M}$ at $\mathfrak{p}$ is canonically isomorphic to $M_\mathfrak{p}$, and thus to $M\otimes_A A_\mathfrak{p}$.
    \item The canonical map from $M_f=\mathfrak{M}^*(U_f)$ to $\mathfrak{M}(U_f)$ is an isomorphism.
  \end{enumerate}
\end{theorem}

\begin{proof}
  The first claim of (a) follows from the fact that
  \[
    \mathfrak{A}_\mathfrak{p}
    = \varinjlim_{U_f\ni\mathfrak{p}} \mathfrak{A}^*(U_f)
    = \varinjlim_{f\not\in\mathfrak{p}} A_f
    = A_\mathfrak{p};
  \]
  the second claim of (a) can be proven in the same way.

  We now show that the map $M_f\to\mathfrak{M}(U_f)$ is \emph{injective}:
  Since $U_f=V(A_f)$, we can always assume that $f=1$ and $U_f=V(A)$.
  The kernel of the map $M\to\mathfrak{M}(V)$ then consists of the $m$ such that $m\otimes_A1_{A_\mathfrak{p}}=0$ for every prime ideal $\mathfrak{p}$, i.e. the $m$ whose annihilator is not contained in any prime ideal:
  the kernel is thus zero.

  Similarly, the map $M_f\to\mathfrak{M}(U_f)$ is \emph{surjective}:
  it suffices to prove this in the case where $f=1$ and $U_f=V$.
  By the above, $\mathfrak{M}(V)=\Gamma(\mathfrak{M})$ is the union of the $\HH^0(\mathfrak{U},\mathfrak{M}^*)$ where $\mathfrak{U}$ runs over the finite covers of $V$ by special open subsets;
  it thus suffices to prove that the map $M\to\HH^0(\mathfrak{U},\mathfrak{M})$ is surjective for every finite cover by special open subsets.

  For this, let $(f_i)_{i=1,\ldots,p}$ be elements of $A$ such that $V=\bigcup U_{f_i}$, and let $X_i$ be the elements of $M_{f_i}$ such that $X_i$ and $X_j$ have the same image in $M_{f_i\cdot f_j}$;
  even if we replace $f_i$ by one of its powers, we can still suppose that $X_i$ is of the form $m_i/f_i$, where $m_i\in M$.
  Since $X_i$ and $X_j$ have the same image in $M_{f_i\cdot f_j}$,
  \[
    \frac{m_i f_j - m_j f_i}{f_i f_j} = 0 \in M_{f_i\cdot f_j}.
  \]
\oldpage{1-06}
  In other words,
  \[
    (f_i\cdot f_j)^r(m_if_j\cdot m_jf_i) = 0 \in M\mbox{ for $r$ large enough.}
  \]

  Also, the $U_{f_i}=U_{f_i^{r+1}}$ cover $V$, i.e. the $f_i^{r+1}$ generate $A$, i.e. we have a relation of the form $1=\sum_{i=1}^p a_i f_i^{r+1}$, where $a_i\in A$.
  It follows that, if $m=\sum_i m_ia_if_i^{r+1}$, then
  \[
    f_j^{r+1}
    = \sum_i a_im_if_i^rf_j^{r+1}
    = \sum_i a_im_jf_j^rf_i^{r+1}
    = m_jf_j^r
  \]
  and, in $M_{f_j}$, we have the equality $X_j=m_j/f_j=m/1$ for all $j$.
\end{proof}

\begin{corollary}{1}
\label{corollary1}
  If $S$ is a multiplicatively-stable subset of $A$, then the restriction of the sheaf $\mathfrak{M}$ to $E_S$ can be identified with the sheaf $\mathfrak{M}_S$ on $V(A_S)$ associated to $M_S$.
\end{corollary}

\begin{proof}
  Let $f$ be an element of $A$.
  The open subset $E_S\cap U_f$ of $E_S$ can then be identified with $E_{S\cdot S_f}$, and, consequently,
  \[
    \mathfrak{M}_S(E_S\cap U_f)
    = M_{S\cdot S_f}
    = (M_S)_f
    = \varinjlim_{U_g\supset E_S}M_{f\cdot g}.
  \]
  The natural maps $M_{f\cdot g}=\mathfrak{M}(U_{fg})\to\mathfrak{M}(E_S\cap U_f)$ thus extend to give a map
  \[
    \mathfrak{M}_S(E_S\cap U_f) \to \mathfrak{M}(E_S\cap U_f)
  \]
  and induce a morphism of sheaves:
  \[
    \mathfrak{M}_S \to \mathfrak{M}|E_S.
  \]

  But if $\mathfrak{p}\in E_S$, then the fibre of $\mathfrak{M}_S$ over $\mathfrak{p}$ is exactly $(M_S)_\mathfrak{p}=M_\mathfrak{p}$, and the above map induces an isomorphism over each $\mathfrak{p}$:
  the map is thus an isomorphism of sheaves.
\end{proof}

\oldpage{1-07}
\begin{corollary}{2}
\label{corollary2}
  $M_S$ is the module of sections of $\mathfrak{M}$ over $E_S$.
\end{corollary}

\begin{corollary}{3}
\label{corollary3}
  If $M$ and $N$ are $A$-modules, then the map $\varphi\colon\Hom(M,N)\to\Hom(\mathfrak{M},\sh{N})$ defined by the functor $M\mapsto\mathfrak{M}$ is bijective.
\end{corollary}

\begin{proof}
  We will exhibit an inverse map $\psi$.
  For this, let $\Gamma$ be the functor that sends any sheaf of $\mathfrak{A}$-modules to its module of sections over $V(A)$.
  The functor $\Gamma$ defines a map from $\Hom(\mathfrak{M},\sh{N})$ to $\Hom(\Gamma(\mathfrak{M}),\Gamma(\sh{N}))$, and thus, since this latter group is isomorphic to $\Hom(M,N)$ by an explicit isomorphism described above, a map $\psi$ from $\Hom(\mathfrak{M},\sh{N})$ to $\Hom(M,N)$.
  Then $\varphi$ and $\psi$ are inverse to one another.
\end{proof}

It follows from \hyperref[corollary3]{this corollary}, along with the exactness of the functor $M\mapsto\mathfrak{M}$, that, if $u\colon\sh{F}\to\sh{G}$ is a morphism of quasi-coherent sheaves, then $\Ker u$ and $\Coker u$ are quasi-coherent sheaves.

\begin{theorem}{2}
\label{theorem2}
  If $\sh{F}$ is an algebraic sheaf on $V(A)$, then the following are equivalent:
  \begin{enumerate}[(a)]
    \item $\sh{F}$ is quasi-coherent.
    \item For every special open subset $U_f$, the natural map
      \[
        \Gamma(V,\sh{F})\otimes_A A_f \to \Gamma(U_f,\sh{F})
      \]
      is bijective.
    \item For every point $\mathfrak{p}$ of $V$, the natural map
      \[
        \Gamma(V,\sh{F})\otimes_A A_\mathfrak{p} \to \sh{F}_\mathfrak{p}
      \]
      is bijective.
    \item $\sh{F}$ is locally isomorphic to a quasi-coherent sheaf, i.e. every point $\mathfrak{p}$ of $V$ has a special neighbourhood $U_f$ such that $\sh{F}|U_f$ is a coherent sheaf.
  \end{enumerate}
\end{theorem}

\begin{proof}
  We have already seen that (a)$\implies$(b), (a)$\implies$(c), and (a)$\implies$(d).

  We will first show that (b)$\implies$(a):
  Let $M=\Gamma(V,\sh{F})$.
  Then, for every affine open subset $U_f$ we have restrictions: $M\to\Gamma(U_f,\sh{F})$, and since $\Gamma(U_f,\sh{F})$ is an $A_f$-module, it follows that we have a map:
\oldpage{1-08}
  \[
    M_f = M\otimes_A A_f \to \Gamma(U_f,\sh{F}).
  \]

  These maps induce a morphism of sheaves: $\mathfrak{M}\to\sh{F}$.

  This morphism will be bijective if the induced maps $\mathfrak{M}(U_f)\to\sh{F}(U_f)$ are bijective (and so (b)$\implies$(a)) or if the induced maps $\mathfrak{M}_\mathfrak{p}\to\sh{F}_\mathfrak{p}$ are bijective (and so (c)$\implies$(a)).

  It remains only to show that (d)$\implies$(a).
  The proof that follows is due to Grothendieck.

  We note first of all that the map $A\to A_f$ endows every $A_f$-module with an $A$-module structure, and thus associates, to every quasi-coherent sheaf $\sh{F}$ on $U_f$, a quasi-coherent sheaf $\overline{\sh{F}}$ on $V$ (we will return later to this operation).
  Furthermore, if $U_g$ is an affine open subset of $V$, then:
  \[
    \Gamma(U_g,\overline{\sh{F}})
    = \Gamma(V,\overline{\sh{F}})_g
    = \Gamma(U_f,\sh{F})_g
    = \Gamma(U_f\cap U_g,\sh{F}).
  \]

  So take some algebraic sheaf $\sh{G}$ on $V$ such that there exists a cover of $V$ by special open subsets $U_{f_i}$ with the condition that $\sh{G}|U_{f_i}$ is quasi-coherent.
  We will now show that $\sh{G}$ is quasi-coherent.

  Since $\sh{G}$ is a sheaf, we have, for all $U_g$, an exact sequence of the form:
  \[
    0
    \to \Gamma(U_g,\sh{G})
    \to \prod_i \Gamma(U_g\cap U_i,\sh{G})
    \to \prod_{i<j} \Gamma(U_g\cap U_i\cap U_j,\sh{G})
  \]
  or even
  \[
    0
    \to \Gamma(U_g,\sh{G})
    \to \prod_i \Gamma)U_g,\overline{\sh{G}|U_i})
    \to \prod_{i\leq j} \Gamma(U_g,\overline{\sh{G}|U_i\cap U_j}).
  \]

  The two latter terms of the exact sequence are the sections over $U_g$ of the sheaves associated to the modules
  \[
    M = \prod_i \Gamma(U_i,\sh{G}|U_i)
    \mbox{ and }
    N = \prod_{i<j} \Gamma(U_i\cap U_j,\sh{G}).
  \]
  Letting $g$ vary, we obtain, by taking the limit, an ``exact sequence of sheaves'': $0\to\sh{G}\to\mathfrak{M}\to\mathfrak{N}$.
  The sheaf $\sh{G}$ is thus the kernel of a morphism of quasi-coherent sheaves: $\sh{G}$ is quasi-coherent.
\end{proof}


\section{Coherent sheaves on $V(A)$}
\label{section4}

\oldpage{1-09}
From now on we will suppose that \emph{$A$ is a Noetherian ring}.
Then it is well known that the category of Noetherian $A$-modules agrees with that of $A$-modules of finite type.
We define a \emph{coherent sheaf on $V(A)$} to be any sheaf $\sh{F}$ that is isomorphic to a sheaf of the type $\mathfrak{M}$, where $M$ is an $A$-module of finite type.
The sheaf $\sh{F}$ is thus quasi-coherent, and it is a sheaf of finite type
(a sheaf $\sh{F}$ of $\mathfrak{A}$-modules on a topological space $X$ is said to be \emph{of finite type} if every point of $X$ admits a neighbourhood $U$ along with a surjection $\mathfrak{A}^p|U\to\sh{F}|U\to0$, where $\mathfrak{A}^p$ is sheaf given by the direct sum of $p$ sheaves, each isomorphic to $\mathfrak{A}$.
Such a surjection induces a map
\[
  \Gamma(U,\mathfrak{A}^p) = \Gamma(U,\mathfrak{A})^p \to \Gamma(U,\sh{F})
\]
and defines $p$ surjections from $\sh{F}$ to $U$, i.e. the images of the basis vectors of $\Gamma(U,\mathfrak{A})^p$.
Conversely, the data of $p$ sections of $\sh{F}$ over $U$ defines a map $A^p|U\to\sh{F}|U$).

\begin{theorem}{3}
\label{theorem3}
  The following are equivalent (if $A$ is Noetherian and if $\sh{F}$ is an algebraic sheaf on $V(A)$):
  \begin{enumerate}[(a)]
    \item $\sh{F}$ is a coherent algebraic sheaf on $V(A)$.
    \item $\sh{F}$ is quasi-coherent and of finite type.
    \item $\sh{F}$ is of finite type, and, for every open subset $U$ and every morphism $\varphi\colon\sh{G}\to\sh{F}|U$, where $\sh{G}$ is a sheaf of finite type on $U$, the kernel $\Ker\varphi$ is of finite type on $U$.
  \end{enumerate}
\end{theorem}

\begin{proof}
  We already know that (a)$\implies$(b).
  Conversely, if $\sh{F}$ is quasi-coherent and of finite type, then there exists a cover $(U_i)_{i\in I}$ of $V(A)$ by special open subsets such that $\Gamma(U_{f_i},\sh{F})$ is an $A_{f_i}$-module of finite type.
  Since $V(A)$ is quasi-compact, we can always assume that this cover is finite, and that $\sh{F}$ is of the form $\mathfrak{M}$:
  we then need to prove that $M$ is an $A$-module of finite type.
  But, for all $i$, there exists a finite number of $m_{i_k}\in M$ that generate $M_{f_i}$, and, if $N$ denotes the submodule generated by all the $m_{i_k}$, then $N$ is an $A$-module of finite type, and clearly $\mathfrak{N}=\mathfrak{M}$, whence $N=M$.

  \bigskip
  Now for (c)$\implies$(b):
  If $\sh{F}$ is of finite type, then every point admits a special neighbourhood $U$ over which we have a surjection:
\oldpage{1-10}
  \[
    \mathfrak{A}^p|U \to \sh{F}|U \to 0.
  \]

  Since the kernel of this surjection is also of finite type, we in fact have an exact sequence (provided that the special open subset $U$ is small enough):
  \[
    \mathfrak{A}^q|U \to \mathfrak{A}^p|U \to \sh{F}|U \to 0.
  \]

  The sheaf $\sh{F}$ is thus, on every small enough special open subset, the cokernel of a morphism of quasi-coherent sheaves:
  so $\sh{F}$ is locally quasi-coherent, and is thus quasi-coherent.

  \bigskip
  Finally, (a)$\implies$(c):
  If $\sh{F}$ is coherent, then $\sh{F}$ is of finite type.
  Furthermore, if $\varphi\colon\sh{G}\to\sh{F}|U$ is a morphism, then we can always assume that $U$ is a special open subset that is small enough such that we have a surjection $\psi\colon\mathfrak{A}^p|U\to\sh{G}\to0$.

  We then have the following diagram:
  \[
    \begin{tikzcd}
      & \mathfrak{A}^p|U \dar[swap,"\psi"] \drar["\chi"] &
    \\\Ker\varphi \rar & \sh{G} \rar[swap,"\varphi"] \dar & \sh{F}|U
    \\ & 0 &
    \end{tikzcd}
  \]
  where $\chi=\varphi\circ\psi$.

  The sheaves $\mathfrak{A}^p|U$ and $\sh{F}|U$ are coherent on $U$, and $\chi$ is thus induced by a morphism $\chi'\colon\mathfrak{A}^p(U)\to\sh{F}(U)$.
  This ``resolution'' of $\Gamma(U,\sh{F})=\sh{F}(U)$ extends to a resolution
  \[
    \Gamma(U,A)^q \to \Gamma(U,A)^p \to \Gamma(U,\sh{F}).
  \]

  This latter exact sequence allows us to complete the diagram and show that, on $U$, $\Ker\varphi$ is the quotient of a sheaf of finite type:
  \[
    \begin{tikzcd}
      \mathfrak{A}^q|U \rar \dar & \mathfrak{A}^p|U \rar["\chi"] \dar[swap,"\psi"] & \sh{F}|U \dar
    \\\Ker\varphi \rar \dar & \sh{G} \rar[swap,"\varphi"] \dar & \sh{F}|U \uar
    \\ 0 & 0 &
    \end{tikzcd}
  \]
\end{proof}

\begin{remark}
\label{remark-theorem3}
\oldpage{1-11}
  Claim (c) gives a characterisation of coherent sheaves, which does not require the base of special open subsets of $V(A)$.
  In fact, if $V$ is an arbitrary topological space, and $\mathfrak{A}$ a sheaf of rings on $V$, then the sheaves of $\mathfrak{A}$-modules that satisfy (c) are the objects of an abelian subcategory of the category of sheaves of $\mathfrak{A}$-modules (see Serre \cite{4}).
  In the case considered here, the quasi-coherent sheaves are the inductive limits of coherent sheaves.
\end{remark}


\section{The maximal spectrum of a Jacobson ring}
\label{section5}

We are going to apply the above to algebraic geometry.
For this, we denote by $\Omega(A)$ the \emph{maximal spectrum of $A$}, which is defined to be the topological subspace of $V(A)$ consisting of the maximal ideals of $A$.
We further suppose that $A$ is a \emph{Jacobson ring}, i.e. that every prime ideal is the intersection of maximal ideals (see \cite{1} and \cite{3}).
The following proposition then holds true:

\begin{proposition}
  The following are equivalent:
  \begin{enumerate}[(a)]
    \item $A$ is a Jacobson ring.
    \item The correspondence $U\mapsto U\cap\Omega(A)$ between open subsets of $V(A)$ and of $\Omega(A)$ is bijective.
    \item The correspondence $W\mapsto W\cap\Omega(A)$ between closed subsets of $V(A)$ and of $\Omega(A)$ is bijective.
  \end{enumerate}
\end{proposition}

\begin{proof}
  The equivalence of (b) and (c) is trivial.
  On the other hand, the closed subsets of $V$ correspond to the ideals of $A$ that are intersections of prime ideals.
  The closed subsets of $\Omega$ correspond to the ideals of $A$ that are intersections of maximal ideals.
  The two families of ideals agree if and only if $A$ is a Jacobson ring.
\end{proof}

Under this last hypothesis, the spaces $V(A)$ and $\Omega(A)$ thus have the same lattice of open subsets.
It clearly follows that the sheaves on $V$ and on $\Omega$ ``are in bijective correspondence''.
In particular, we define a \emph{special open subset of $\Omega$} to be the restriction of any special open subset of $V$, and a \emph{quasi-coherent} (resp. \emph{coherent}) \emph{algebraic sheaf on $\Omega$} to be the restriction of any quasi-coherent (resp. coherent) algebraic sheaf on $V$.
The properties of these sheaves on $V$ extend (up to evident modifications) to their restrictions to $\Omega$.

In particular, every algebra of finite type over a field is a Jacobson ring.
If $k$ is an algebraically closed field, $V$ an affine algebraic set over $k$, and $A$ its coordinate ring, then $V$ is homeomorphic to the maximal spectrum
\oldpage{1-12}
$\Omega(A)$ of $A$.
The sheaf of rings $\mathfrak{A}|\Omega(A)$ is called the \emph{sheaf of germs of regular functions on $V$}.

Similarly, a sheaf of modules over this sheaf of rings is said to be \emph{algebraic} (resp. \emph{quasi-coherent algebraic}, resp. \emph{coherent algebraic}) if it is the restriction of a sheaf on $V(A)$ with the named property.


\section{Quasi-coherent sheaves on an algebraic variety}
\label{section6}

More generally, if $X$ is an arbitrary algebraic set over $k$, then we define a \emph{quasi-coherent} (resp. \emph{coherent}) \emph{algebraic sheaf on $X$} to be any sheaf $\sh{F}$ such that there exists a cover of $X$ by affine open subsets $U_i$ such that $\sh{F}|U_i$ is quasi-coherent (resp. coherent).
By \hyperref[remark-theorem3]{the remark} that follows \cref{theorem3}, this is equivalent to saying that, for every affine open subset $U$, the sheaf $\sh{F}|U$ is quasi-coherent (resp. coherent).

In particular, the sheaf of rings $\mathfrak{A}$ that is such that, for every affine open subset $U$, $\mathfrak{A}(U)$ is the ring of regular functions on $U$ (with the evident restrictions), is a coherent sheaf.
We call it the \emph{sheaf of germs of regular functions}.
Every sheaf of modules over $\mathfrak{A}$ is said to be \emph{algebraic}.


%% Bibliography %%

\nocite{*}
\bibliographystyle{acm}
\begin{thebibliography}{10}

  \bibitem{1}
  {\sc Cartan, H.}
  \newblock Vari\'{e}t\'{e}s alg\'{e}briques affines.
  \newblock {\em S\'{e}minaire Cartan-Chevalley, Volume~8} (1955/56), Talk no.~3.

  \bibitem{2}
  {\sc Godement, R.}
  \newblock ``Topologie alg\'{e}brique et th\'{e}orie des faisceaux''.
  \newblock {Paris, Hermann.}
  \newblock {\em Act. scient. et ind. 1252} (1958)

  \bibitem{3}
  {\sc Krull, W.}
  \newblock Jacobsonsche Rings, Hilbertscher Nullstellensatz, Dimensionstheorie.
  \newblock {\em Math. Z. 54} (1951), pp.~354--387

  \bibitem{4}
  {\sc Serre, J.-P.}
  \newblock Faisceaux alg\'{e}briques coh\'{e}rents.
  \newblock {\em Ann. Math. 61\/} (1955), pp.~197--279.

\end{thebibliography}

\end{document}
