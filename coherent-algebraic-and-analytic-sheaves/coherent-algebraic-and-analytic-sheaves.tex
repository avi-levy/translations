\documentclass[10pt]{article}

\usepackage{amssymb,amsmath}
\usepackage{hyperref}
\usepackage{cleveref}
\usepackage{enumerate}
\usepackage{mathrsfs}
%% Fancy fonts --- feel free to remove! %%
\usepackage{ebgaramond-maths}


%% Theorem environments %%

\usepackage{amsthm}

\theoremstyle{plain}

\newtheorem{innercustomproposition}{Proposition}
\crefname{innercustomproposition}{Proposition}{Propositions}
\newenvironment{proposition}[1]
  {\renewcommand\theinnercustomproposition{#1}\innercustomproposition}
  {\endinnercustomproposition}

\newtheorem{innercustomlemma}{Lemma}
\crefname{innercustomlemma}{Lemma}{Lemmas}
\newenvironment{lemma}[1]
  {\renewcommand\theinnercustomlemma{#1}\innercustomlemma}
  {\endinnercustomlemma}

\newtheorem{innercustomcorollary}{Corollary}
\crefname{innercustomcorollary}{Corollary}{Corollaries}
\newenvironment{corollary}[1]
  {\renewcommand\theinnercustomcorollary{#1}\innercustomcorollary}
  {\endinnercustomcorollary}


\theoremstyle{definition}
\newtheorem*{remark}{Remark}
\newtheorem*{definition}{Definition}


%% Shortcuts %%

\newcommand{\sh}{\mathscr}
\usepackage{aurical}
\newcommand{\shHom}{\sh{H}\textup{\kern-2.2pt{\Fontauri\slshape om}}}

\renewcommand{\geq}{\geqslant}
\renewcommand{\leq}{\leqslant}

\newcommand{\todo}{\textbf{ !TODO! }}
\newcommand{\oldpage}[1]{\marginpar{\textit{p.~#1}}}


%% Document %%

\usepackage{embedall}
\begin{document}

\renewcommand{\abstractname}{Translator's note.}

\title{On coherent algebraic and analytic sheaves}
\author{A. Grothendieck}
\date{4\textsuperscript{th} and 11\textsuperscript{th} of February, 1957}
\maketitle

\begin{abstract}
  What follows is a translation of the French paper:

  \smallskip\noindent
  \textsc{Grothendieck, A.}. Sur les faisceaux algébriques et les faisceaux analytiques cohérents. \emph{Séminaire Henri Cartan}, Volume 9 (1956-1957), Talk no.~2, pp.~1–16. {\footnotesize\url{http://www.numdam.org/item/SHC_1956-1957__9__A2_0/}}

  \smallskip
  This translation is one of a series: {\footnotesize\url{https://github.com/thosgood/translations}.}
\end{abstract}

\tableofcontents


%% Content %%

\bigskip
\oldpage{2-01}
The aim of this expose is to generalise certain theorems of Serre.
It makes fundamental use of the techniques of Serre \cite{1,2,3}.


\section{Generalities on coherent algebraic sheaves}

Let $X$ be a topological space endowed with a sheaf of rings $\sh{O}$.
A sheaf of $\sh{O}$-modules $A$ (or simply an $\sh{O}$-module) is said to be \emph{of finite type} if, on every small-enough open subset, it is isomorphic to a quotient of $\sh{O}^n$ (for some finite integer $n\geq0$), and \emph{coherent} if it is of finite type and if, for every homomorphism $\sh{O}^m\to A$ on an open subset $U$ of $X$, the kernel is of finite type.
If $0\to A'\to A\to A''\to0$ is an exact sequence of $\sh{O}$-modules, and if two of the modules are coherent, then so too is the third;
the kernel, cokernel, image, and coimage of a homomorphism of coherent $\sh{O}$-modules is a coherent $\sh{O}$-module.
If $A$ and $B$ are coherent $\sh{O}$-modules, then so too is the sheaf $\shHom_\sh{O}(A,B)$ of germs of homomorphisms from $A$ to $B$.
If $\sh{O}$ itself is coherent, then coherent $\sh{O}$-modules are exactly the $\sh{O}$-modules that, on small-enough open subsets, are isomorphic to the cokernel of some homomorphism $\sh{O}^m\to\sh{O}^n$.
For all of this, and other elementary properties, see \cite[chapitre~1, paragraphe~2]{1}.


%% Bibliography %%

\nocite{*}
\bibliographystyle{acm}
\bibliography{\jobname}

\end{document}
