\documentclass{report}

\title{Regular singular differential equations}
\author{Pierre Deligne}
\date{}

\usepackage{amssymb,amsmath}

\usepackage{hyperref}
\usepackage[nameinlink]{cleveref}
\usepackage{enumerate}

\usepackage{mathrsfs}
%% Fancy fonts --- feel free to remove! %%
\usepackage{Baskervaldx}
\usepackage{mathpazo}


\usepackage{fancyhdr}
\usepackage{lastpage}
\usepackage{xstring}
\makeatletter
\ifx\pdfmdfivesum\undefined
  \let\pdfmdfivesum\mdfivesum
\fi
\edef\filesum{\pdfmdfivesum file {\jobname}}
% \pagestyle{fancy}
% \fancyhf{}
% \lhead{\footnotesize\thechapter\thesection}
% \rhead{\footnotesize Version: \texttt{\StrMid{\filesum}{1}{8}}}
% \cfoot{\small\thepage\ of \pageref*{LastPage}}


\crefname{section}{Section}{Sections}
\crefname{equation}{}{}


%% Shortcuts %%

\newcommand{\sh}{\mathscr}
\newcommand{\cat}{\mathcal}

\renewcommand{\geq}{\geqslant}
\renewcommand{\leq}{\leqslant}

\DeclareMathOperator{\Spec}{Spec}

\newcommand{\todo}{\textbf{ !TODO! }}
\newcommand{\oldpage}[1]{\marginpar{\footnotesize$\Big\vert$ \textit{p.~#1}}}


%% Document %%

\usepackage{embedall}
\begin{document}

\maketitle

\renewcommand{\abstractname}{Translator's note.}

\begin{abstract}
  \renewcommand*{\thefootnote}{\fnsymbol{footnote}}
  \emph{This text is one of a series\footnote{\url{https://github.com/thosgood/translations}} of translations of various papers into English.}
  \emph{The translator takes full responsibility for any errors introduced in the passage from one language to another, and claims no rights to any of the mathematical content herein.}
  
  \emph{What follows is a translation (last updated \today) of the French book:}

  \medskip\noindent
  \textsc{Deligne, P.}
  \emph{Equations Diff\'{e}rentielles \`{a} Points Singuliers R\'{e}guliers.}
  Springer-Verlag, Lecture Notes in Mathematics \textbf{163} (1970).
  {\footnotesize\url{https://publications.ias.edu/node/355}}
\end{abstract}

\setcounter{footnote}{0}

\tableofcontents


%% Content %%


\setcounter{chapter}{-1}

\chapter{Introduction}
\label{0}

\oldpage{1}
If $X$ is a (non-singular) complex analytic manifold, then there is an equivalence between the notions of
\begin{enumerate}[a)]
  \item local systems of complex vectors on $X$; and
  \item vector bundles on $X$ endowed with an integrable connection.
\end{enumerate}

The latter of these two notions can be adapted in an evident way to the case where $X$ is a non-singular algebraic variety over a field $k$ (which we will take here to be of characteristic $0$).
However, general algebraic vector bundles with integrable connections are pathological (see \cref{II.6.19});
we only obtain a reasonable theory if we impose a ``regularity'' condition at infinity.
By a theorem of Griffiths \cite{8}, this condition is automatically satisfied for ``Gauss-Manin connections'' (see \cref{II.7}).
In dimension one, this is closely linked to the idea of regular singular points of a differential equation (see \cref{I.4} and \cref{II.1}).

In Chapter~I, we explain the different forms that the notion of an integrable connection can take.
In Chapter~II, we prove the fundamental facts concerning regular connections.
In Chapter~III, we translate certain results that we have obtained into the language of Nilsson class functions, and, as an application of the regularity theorem (\cref{II.7}), we explain the proof by Brieskorn \cite{5} of the monodromy theorem.

These notes came from the non-crystalline part of a seminar given at Harvard during the autumn of 1969, under the title: ``Regular singular differential equations and crystalline cohomology''.

I thank the assistants of this seminar, who had to be subjected to often unclear expos\'{e}s, and who allowed me to bring numerous simplifications.

I also thank N.~Katz, with whom I had numerous and useful conversations, and to whom are due the principal results of section~\cref{II.1}.


\section*{Notations and terminology}

\oldpage{2}
Within a single chapter, the references follow the decimal system.
A reference to a different chapter (resp. to the current introduction) is preceded by the Roman numeral of the chapter (resp. by 0).

We will use the following definitions:
\begin{enumerate}[({0.}1)]
  \item \emph{analytic space}:
    the analytic spaces are complex and of locally-finite dimension.
    They are assumed to be $\sigma$-compact, but not necessarily separated.
  \item \emph{multiform function}:
    a synonym for multivalued function --- see a precise definition in \cref{I.6.2}.
  \item \emph{immersion}:
    following the tradition of algebraic geometers, immersion is a synonym for ``embedding''.
  \item \emph{smooth}:
    a morphism $f\colon X\to S$ of analytic spaces is smooth if, locally on $X$, it is isomorphic to the projection from $D^n\times S$ to $S$, where $D^n$ is an open polydisc.
  \item \emph{locally paracompact}:
    a topological space is locally paracompact if every point has a paracompact neighbourhood (and thus a fundamental system of paracompact neighbourhoods).
  \item non-singular (or smooth) \emph{complex algebraic variety}:
    a smooth scheme of finite type over $\Spec(\mathbb{C})$.
  \item (complex) \emph{analytic manifold}:
    a non-singular (or smooth) analytic space.
  \item \emph{covering}:
    following the tradition of topologists, a covering is a continuous map $f\colon X\to Y$ such that every point $y\in Y$ has a neighbourhood $V$ such that $f|V$ is isomorphic to the projection from $F\times V$ to $V$, where $F$ is discrete.
\end{enumerate}


\renewcommand{\thechapter}{\Roman{chapter}}

\chapter{Dictionary}
\label{I}

\oldpage{3}
In this chapter, we explain the relations between various aspects and various uses of the notion of ``local systems of complex vectors''.
The equivalence between the points of view considered has been well known for a long time.

The ``crystalline'' point of view has not been considered;
see \cite{4,10}.


\section{Local systems and the fundamental group}
\label{I.1}

% \begin{definition}{1.1}
% \label{I.1.1}
%   Let $X$ be a topological space.
%   A \emph{local complex system} on $X$ is a sheaf of complex vectors on $X$ that, locally on $X$, is isomorphic to a constant sheaf $\mathbb{C}^n$ (n$\in\mathbb{N}$).
% \end{definition}

\todo\textbf{SORT OUT HEADERS/FOOTERS}


%% Bibliography %%

\nocite{*}
\bibliographystyle{acm}

\end{document}
