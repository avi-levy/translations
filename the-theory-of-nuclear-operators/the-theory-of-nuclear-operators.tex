\documentclass[11pt]{article}

\usepackage{amssymb,amsmath}

\usepackage{hyperref}
\usepackage[nameinlink]{cleveref}
\usepackage{enumerate}
\usepackage{tikz-cd}
\usepackage{graphicx}

\usepackage{mathrsfs}
%% Fancy fonts --- feel free to remove! %%
% \usepackage{ebgaramond-maths}
% \usepackage{Baskervaldx}
\usepackage{boisik}


\crefname{section}{Section}{Sections}
\crefname{equation}{}{}

%% Theorem environments %%

\usepackage{amsthm}

  \theoremstyle{plain}

  \newtheorem{innercustomtheorem}{Theorem}
  \crefname{innercustomtheorem}{Theorem}{Theorems}
  \newenvironment{theorem}[1]
    {\renewcommand\theinnercustomtheorem{#1}\innercustomtheorem}
    {\endinnercustomtheorem}

  \newtheorem{innercustomproposition}{Proposition}
  \crefname{innercustomproposition}{Proposition}{Propositions}
  \newenvironment{proposition}[1]
    {\renewcommand\theinnercustomproposition{#1}\innercustomproposition}
    {\endinnercustomproposition}

  \newtheorem{innercustomlemma}{Lemma}
  \crefname{innercustomlemma}{Lemma}{Lemmas}
  \newenvironment{lemma}[1]
    {\renewcommand\theinnercustomlemma{#1}\innercustomlemma}
    {\endinnercustomlemma}

  \newtheorem{innercustomcorollary}{Corollary}
  \crefname{innercustomcorollary}{Corollary}{Corollaries}
  \newenvironment{corollary}[1]
    {\renewcommand\theinnercustomcorollary{#1}\innercustomcorollary}
    {\endinnercustomcorollary}


  \theoremstyle{definition}

  \newtheorem*{remark}{Remark}

  \newtheorem{innercustomdefinition}{Definition}
  \crefname{innercustomdefinition}{Definition}{Definitions}
  \newenvironment{definition}[1]
    {\renewcommand\theinnercustomdefinition{#1}\innercustomdefinition}
    {\endinnercustomdefinition}


%% Shortcuts %%

\newcommand{\sh}{\mathscr}
\newcommand{\cat}{\mathcal}
\newcommand{\BB}{\mathcal{B}}
\newcommand{\projotimes}{\widehat{\otimes}_\pi}
\newcommand{\tr}{\operatorname{Tr}}

\renewcommand{\geq}{\geqslant}
\renewcommand{\leq}{\leqslant}

\newcommand{\todo}{\textbf{ !TODO! }}
\newcommand{\oldpage}[1]{\marginpar{\footnotesize$\Big\vert$ \textit{p.~#1}}}


%% Document %%

\usepackage{embedall}
\begin{document}

\renewcommand{\abstractname}{Translator's note.}

\title{The theory of nuclear operators}
\author{L. Schwartz}
\date{17\textsuperscript{th} of February, 1954}
\maketitle

\begin{abstract}
  \emph{This text is one of a series\footnote{\url{https://github.com/thosgood/translations}} of translations of various papers into English.}
  \emph{What follows is a translation (last updated \today) of the French paper:}

  \medskip\noindent
  \textsc{Schwartz, L}. La théorie des opérateurs nucléaires. \emph{Séminaire Schwartz}, Volume~1 (1953-1954), Talk no.~12, 7~p. {\footnotesize\url{http://www.numdam.org/item/SLS_1953-1954__1__A13_0/}}
\end{abstract}

\tableofcontents


%% Content %%

\section{The trace}
\label{section1}

\oldpage{1}

Let $E$ be a \emph{Banach} space, whose dual we will call $E'$.
We know, by definition, that there exists a bijective and isometric correspondence between the space $\BB(E,E')$ of continuous bilinear forms on $E\times E'$ and the dual $E\projotimes E'$.
To the canonical bilinear form $(x,x')\mapsto\langle x,x'\rangle$ thus corresponds a continuous linear form on $E\projotimes E'$ that we call ``the trace'', and that we denote by $\tr$.
If



%% Bibliography %%

\nocite{*}
\bibliographystyle{acm}
\bibliography{\jobname}

\end{document}
