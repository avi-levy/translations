\documentclass{article}

\usepackage{amssymb,amsmath}

\usepackage{hyperref}
\usepackage[nameinlink]{cleveref}
\usepackage{enumerate}
\usepackage{tikz-cd}
\usepackage{graphicx}

\usepackage{mathrsfs}
%% Fancy fonts --- feel free to remove! %%
\usepackage{ebgaramond-maths}
\usepackage{mathpazo}


\crefname{section}{Section}{Sections}
\crefname{equation}{}{}

%% Theorem environments %%

\usepackage{amsthm}

  \theoremstyle{plain}

  \newtheorem{innercustomtheorem}{Theorem}
  \crefname{innercustomtheorem}{Theorem}{Theorems}
  \newenvironment{theorem}[1]
    {\renewcommand\theinnercustomtheorem{#1}\innercustomtheorem}
    {\endinnercustomtheorem}

  \newtheorem{innercustomproposition}{Proposition}
  \crefname{innercustomproposition}{Proposition}{Propositions}
  \newenvironment{proposition}[1]
    {\renewcommand\theinnercustomproposition{#1}\innercustomproposition}
    {\endinnercustomproposition}

  \newtheorem{innercustomlemma}{Lemma}
  \crefname{innercustomlemma}{Lemma}{Lemmas}
  \newenvironment{lemma}[1]
    {\renewcommand\theinnercustomlemma{#1}\innercustomlemma}
    {\endinnercustomlemma}

  \newtheorem{innercustomcorollary}{Corollary}
  \crefname{innercustomcorollary}{Corollary}{Corollaries}
  \newenvironment{corollary}[1]
    {\renewcommand\theinnercustomcorollary{#1}\innercustomcorollary}
    {\endinnercustomcorollary}


  \theoremstyle{definition}

  \newtheorem*{remark}{Remark}

  \newtheorem{innercustomdefinition}{Definition}
  \crefname{innercustomdefinition}{Definition}{Definitions}
  \newenvironment{definition}[1]
    {\renewcommand\theinnercustomdefinition{#1}\innercustomdefinition}
    {\endinnercustomdefinition}


%% Shortcuts %%

\newcommand{\sh}{\mathscr}
\newcommand{\cat}{\mathcal}
\newcommand{\BB}{\mathcal{B}}
\newcommand{\LL}{\mathcal{L}}
\newcommand{\projotimes}{{\otimes}_\pi}
\newcommand{\injotimes}{{\otimes}_\varepsilon}
\newcommand{\cprojotimes}{\widehat{\otimes}_\pi}
\newcommand{\cinjotimes}{\widehat{\otimes}_\varepsilon}
\newcommand{\cotimes}{\widehat{\otimes}}
\newcommand{\tr}{\operatorname{Tr}}

\renewcommand{\geq}{\geqslant}
\renewcommand{\leq}{\leqslant}
\renewcommand{\rangle}{\rotatebox[origin=c]{180}{$\langle$}}

\newcommand{\todo}{\textbf{ !TODO! }}
\newcommand{\oldpage}[1]{\marginpar{\footnotesize$\Big\vert$ \textit{p.~#1}}}


%% Document %%

\usepackage{embedall}
\begin{document}

\renewcommand{\abstractname}{Translator's note.}

\title{The theory of nuclear operators}
\author{L. Schwartz}
\date{17\textsuperscript{th} of February, 1954}
\maketitle

\begin{abstract}
  \emph{This text is one of a series\footnote{\url{https://github.com/thosgood/translations}} of translations of various papers into English.}
  \emph{What follows is a translation (last updated \today) of the French paper:}

  \medskip\noindent
  \textsc{Schwartz, L}. La théorie des opérateurs nucléaires. \emph{Séminaire Schwartz}, Volume~1 (1953-1954), Talk no.~12, 7~p. {\footnotesize\url{http://www.numdam.org/item/SLS_1953-1954__1__A13_0/}}
\end{abstract}

\tableofcontents


%% Content %%

\section{The trace}
\label{section1}

\oldpage{1}

Let $E$ be a \emph{Banach} space, whose dual we will call $E'$.
We know, by definition, that there exists a bijective and isometric correspondence between the space $\BB(E,E')$ of continuous bilinear forms on $E\times E'$ and the dual $E\cprojotimes E'$.
To the canonical bilinear form $(x,x')\mapsto\langle x,x'\rangle$ thus corresponds a continuous linear form on $E\cprojotimes E'$ that we call ``the trace'', and that we denote by $\tr$.
If $u=\sum_v x_v\otimes y'_v$ then, by definition, $\tr(u)=\sum_v\langle x_v,y'_v\rangle$.
The trace form is of norm $1$.
Furthermore, every $u\in E\cprojotimes E'$ can be written in the form $u=\sum_{n\geq0}x_n\otimes y'_n$ with $\sum_{n\geq0}\|x_n\|\|y'_n\|$ finite, and so the series $\sum_{n\geq0}\langle x_n,y'_n\rangle$ converges absolutely, and, since the trace is continuous, we have that
\[
  \tr(u) = \sum_{n\geq0} \langle x_n, y'_n \rangle.
\]

To justify the name ``trace'', recall that we can identify $E\otimes E'$ with the space of endomorphisms of finite rank of $E$, and that, if $E$ is of finite dimensions, then the trace form agrees with the usual trace of operators.

There exists a canonical continuous map $E'\cprojotimes E\to\LL(E;E)$.
If we do not know whether or not it is bijective, we can only speak of the trace of an element of $E'\cprojotimes E$, and not the trace of the image of the operator in $\LL(E;E)$.

Recall as well that there exists an isomorphism $S$ (for symmetry) between $E\otimes E'$ and $E'\otimes E$, defined by
\[
  S\colon \sum_v x_v\otimes y'_v \mapsto \sum_v y'_v\otimes x_v
\]
for $x_v\in E$ and $y'_v\in E'$.

If we identify $E\otimes E'$ with the space of maps of finite rank from $E$ to $E$, and $E'\otimes E\subset E'\otimes(E')'$ with a space of transformations of $E'$,
\oldpage{2}
then the map $S$ corresponds to the transposition of operators.
Thanks to $S$, the trace is also defined on $E'\cprojotimes E$.
We can thus understand the duality between $E\cprojotimes F$ and $\BB(E,F)$ by means of the trace: let $A\in\BB(E,F)\subset\LL(E;F')$.
If $1$ is the identity in $F$, then $A\otimes1$ sends $E\cprojotimes F$ to $F'\cprojotimes F$.
So if $u\in E\cprojotimes F$, then we can take the trace of $(A\otimes1)(u)\in\LL(F;F')$, and we have
\[
\label{equation1}
  \langle u,A \rangle = \tr((A\otimes1)(u)).
  \tag{1}
\]
Indeed, both sides of the equation (for fixed $A$) are continuous linear forms in $u$, and are equal for $u=x\otimes y$.


\section{The map $E'\cprojotimes F\to\LL_b(E_\tau;F)$ for $E$ and $F$ locally convex}
\label{section2}

The subscript ${}_b$ denotes the uniform convergence topology on bounded subsets of a space of linear maps.

Let $E$ and $F$ be arbitrary locally convex separated spaces.
Elements of $E'\otimes F$ correspond to continuous linear maps of finite rank from $E$ to $F$.
So $E'\otimes F\subset\LL(E_\tau;F)$, since the latter is the space of weakly continuous maps (see Exposé~8, §1).

\begin{proposition}{1}
\label{proposition1}
  The topology induced on $E'\otimes F$ by $\LL_b(E_\tau;F)$ is identical to the topology of $E'_b\injotimes F$.
\end{proposition}

\begin{proof}
  The topology of $E'_b\injotimes F$ is, by definition, the topology induced on $E'\otimes F$ by $\LL_\varepsilon((E'')_\tau;F)$.
  But an equicontinuous subset of $E''$ is the  polar of a neighbourhood of $0$ in $E'$, which is itself the polar of a bounded subset of $E$, and thus (by the bipolar theorem) is the weakly closed convex balanced hull of a bounded subset of $E$.
  But, in a \todo!!!!!!-topology, we can replace the sets of \todo!!!!!! by their closed convex balanced hull.
  Thus $\LL_\varepsilon((E'')_\tau;F)$ and $\LL_b(E_\tau;F)$ induce the same topology on $E'\otimes F$.
\end{proof}

\begin{corollary}{1}
\label{corollary-1}
  If $E$ and $F$ are complete, then there exists a continuous map $\varphi$ from $E'\projotimes F$ to $\LL_b(E_\tau;F)$ that extends the identity on $E'\otimes F$.
\end{corollary}

\begin{proof}
  Indeed, the $\pi$-topology being finer than the $\varepsilon$-topology, there exists a canonical map $E'\cprojotimes F\to E'\cinjotimes F$ which we can compose with the map $E'\cinjotimes F\to \LL_b(E_\tau;F)$.
\end{proof}

\section{Definition of nuclear maps — the case of Banach spaces}
\label{section3}

\oldpage{3}
\emph{From now on, the only tensor product that we will consider is the $\pi$-product; thus $E\cotimes F$ means $E\cprojotimes F$.}

\begin{definition}{1}
\label{definition1}
  If $E$ and $F$ are Banach spaces, then we write $L^1(E;F)$ to denote the subspace $\varphi(E'\cotimes F)$ of $\LL(E;F)$.
  The elements of $L^1(E;F)$ are called \emph{nuclear} (or \emph{Fredholm}) operators.
  Note that $L^1(E;F)$ is a \emph{quotient space} of $E\cotimes F$.
  The quotient norm of the $\pi$-norm will be called the \emph{trace norm}, or the \emph{nuclear norm}, denoted by $\|\cdot\|_1$ or $\|\cdot\|_{\tr}$.
\end{definition}

We do not know a case where $\varphi$ is not bijective, but we do not know how to prove this in general.

Since


\end{document}
