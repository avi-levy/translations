\documentclass{article}

\usepackage{amssymb,amsmath}

\usepackage{hyperref}
\usepackage[nameinlink]{cleveref}
\usepackage{enumerate}

\usepackage{mathrsfs}
%% Fancy fonts --- feel free to remove! %%
\usepackage{Baskervaldx}
\usepackage{mathpazo}


\crefname{section}{Section}{Sections}
\crefname{equation}{}{}


%% Theorem environments %%

\usepackage{amsthm}

  \theoremstyle{plain}

  \newtheorem{innercustomtheorem}{Theorem}
  \crefname{innercustomtheorem}{Theorem}{Theorems}
  \newenvironment{theorem}[1]
    {\renewcommand\theinnercustomtheorem{#1}\innercustomtheorem}
    {\endinnercustomtheorem}

  \newtheorem{innercustomproposition}{Proposition}
  \crefname{innercustomproposition}{Proposition}{Propositions}
  \newenvironment{proposition}[1]
    {\renewcommand\theinnercustomproposition{#1}\innercustomproposition}
    {\endinnercustomproposition}

  \newtheorem{innercustomlemma}{Lemma}
  \crefname{innercustomlemma}{Lemma}{Lemmas}
  \newenvironment{lemma}[1]
    {\renewcommand\theinnercustomlemma{#1}\innercustomlemma}
    {\endinnercustomlemma}

  \newtheorem*{corollary}{Corollary}


  \theoremstyle{definition}

  \newtheorem*{remark}{Remark}

  \newtheorem{innercustomdefinition}{Definition}
  \crefname{innercustomdefinition}{Definition}{Definitions}
  \newenvironment{definition}[1]
    {\renewcommand\theinnercustomdefinition{#1}\innercustomdefinition}
    {\endinnercustomdefinition}


%% Shortcuts %%

\newcommand{\sh}{\mathscr}
\newcommand{\cat}{\mathcal}
\newcommand{\PP}{\mathbf{P}}

\renewcommand{\geq}{\geqslant}
\renewcommand{\leq}{\leqslant}

\DeclareMathOperator{\HH}{H}
\DeclareMathOperator{\rank}{rank}
\DeclareMathOperator{\cl}{cl}

\newcommand{\todo}{\textbf{ !TODO! }}
\newcommand{\oldpage}[1]{\marginpar{\footnotesize$\Big\vert$ \textit{p.~#1}}}


%% Document %%

\usepackage{embedall}
\begin{document}

\renewcommand{\abstractname}{Translator's note.}

\title{The theory of Chern classes}
\author{A. Grothendieck}
\date{}
\maketitle

\begin{abstract}
  \renewcommand*{\thefootnote}{\fnsymbol{footnote}}
  \emph{This text is one of a series\footnote{\url{https://github.com/thosgood/translations}} of translations of various papers into English.}
  \emph{The translator takes full responsibility for any errors introduced in the passage from one language to another, and claims no rights to any of the mathematical content herein.}
  
  \emph{What follows is a translation (last updated \today) of the French paper:}

  \medskip\noindent
  \textsc{Grothendieck, A.}
  ``La th\'{e}orie des classes de Chern''.
  \emph{Bulletin de la Soci\'{e}t\'{e} Math\'{e}matique de France}, Volume~\textbf{86} (1958) , pp.~137-154.
  \textsc{doi}: \href{https://www.doi.org/10.24033/bsmf.1501}{10.24033/bsmf.1501}.
\end{abstract}

\setcounter{footnote}{0}

\tableofcontents


%% Content %%

\bigskip\bigskip


\section*{Introduction}

\oldpage{137}
In this appendix, we will develop an axiomatic theory of Chern classes that will allow us, in particular, to define the Chern classes of an algebraic vector bundle $E$ on a non-singular quasi-projective algebraic variety $X$ as elements of the Chow ring $A(X)$ of $X$, i.e. as classes of cycles under rational equivalence.
This expos\'{e} is inspired by the book of Hirzebruch on one hand (where the essential \emph{formal properties} characterising a theory of Chern classes was brought to light), and by an idea of Chern \cite{2} that consists of using the multiplicative structure of the ring of cycle classes on the bundle of projective spaces $P(E)$ associated to $E$, to reach an effective \emph{construction} of Chern classes.
We note that the exposition given here also applies to other settings than algebraic geometry, and recovers, for example, an entirely elementary theory of Chern classes for complex vector bundles on topological manifold (and, from this, on any space for which the classification theorem of principal bundles with a structure group via a ``classifying space'' holds true).
Similarly, we will obtain, for a complex-analytic vector bundle $E$ on a (non-singular) complex-analytic manifold $X$, Chern classes
\[
  c_p(E) \in \HH^p(X,\Omega_X^p),
\]
where $\Omega_X^p$ is the sheaf of germs of holomorphic differential forms of degree $p$ on $X$.
[And it is certainly easy to prove that this definition agrees with that given recently by Atiyah \cite{1}, and that it is linked to the topological definition of Chern classes via the spectral sequence linking $\HH^p(X,\Omega_X^q)$ and $\HH^\bullet(X,\mathbb{C})$.]
Similarly, the theory of
\oldpage{138}
Stiefel-Whitney classes in cohomology mod~$2$ fits into the framework that we will describe here.

It appears that a satisfying theory of Chern classes in algebraic geometry has been given, for the first time, by W.L.~Chow (unpublished), using the Grassmannian.
The main aim of the current paper has been to eliminate the Grassmannian from the theory.
I have already shown \cite{4} how the theory of Chern classes allows us to \emph{recover} the structure of $A(X)$ when $X$ is a Grassmannian.


\section{Notation}
\label{section1}

In order to not expose ourselves to the complications arising from the theory of intersections, we will limit ourselves in what follows to considering only \emph{non-singular} topological spaces.
The base field $k$ will be fixed once and for all, and to better understand the ideas, the reader can assume it to be algebraically closed.
All the bundles, subvarieties, morphisms, etc. that we consider in what follows will be defined over $k$.

If $X$ is an algebraic space, and $E$ a vector bundle on $X$, then we denote by $\mathbb{P}(E)$ the associated projective bundle.
The fibre $\mathbb{P}(E)_x$ of $\mathbb{P}(E)$ at a point $x\in X$ is thus the projective space associated to the vector space $E_x$, and so a point of $\mathbb{P}(E)_x$ over a point $x\in X$ is exactly a homogeneous line in $E_x$.
Let $f\colon\mathbb{P}(E)\to X$ be the projection of the bundle;
we will consider the inverse image of $E$ under $f$, which is the vector bundle $f^{-1}(E)$ on $\mathbb{P}(E)$.
There is a canonical rank-$1$ subbundle of $f^{-1}(E)$, whose fibre at a point $d$ of $\mathbb{P}(E)$ (over a point $x\in X$) is the line $d$ in $E_x=f^{-1}(E)_d$.
The dual bundle of this subbundle of $f^{-1}(E)$ is denoted $L_E$, and we thus have the inclusion
\[
  \check{L}_E \subset f^{-1}(E).
\]

Let $p$ be the rank of $E$ (assumed to be constant, which is always the case if $X$ is connected).
Then $E^{(1)}=f^{-1}(E)/L_E$ is a vector bundle of rank $p-1$ on $X^{(1)}=\mathbb{P}(E)$, and we can thus construct $X^{(2)}=\mathbb{P}(E^{(1)})$ and the analogous bundle $E^{(2)}=(E^{(1)})^{(1)}$ of rank $p-2$ on $X^{(2)}$.
We thus inductively construct manifolds $X^{(i)}$ and vector bundles $E^{(i)}$ of rank $p-i$ on $X^{(i)}$ ($1\leq i\leq p$), where $X^{(i)}$ is the bundle $\mathbb{P}(E^{(i-1)})$ on $X^{(i-1)}$.
We define a \emph{flag of length $i$} in a vector space $V$ to be an increasing sequence $(V_j)_{0\leq j\leq i}$ of vector subspaces $V_j$, with $\dim V_j=j$.
Then $X^{(i)}$ can also be understood as the \emph{bundle on $X$ of flags of length $i$} in $E$, and if $f^{(i)}$ is the projection from $X^{(i)}$ to $X$, then we directly define, as in the definition of $L_E$, an increasing sequence of subbundles $(V_j)_{0\leq j\leq i}$ of $E_i=(f^{(i)})^{(-1)}(E)$, with $\rank(V_j)=j$, and the quotient of $E_i$ by $V_i$ being exactly the vector bundle $E^{(i)}$.
In particular, $X^{(p)}$ is the \emph{flag manifold} (of maximum length $p$) $D(E)$ of $E$, which thus appears as a ``multiple extension'' of $X$ by fibrations in projective spaces associated to vector bundles\todo;
the inverse image $E_p$ of $E$ in $X^{(p)}=D(E)$ is further \emph{completely split}.
By this, we mean that this rank-$p$ vector bundle is endowed with a sequence $(V_i)_{0\leq i\leq p}$ of vector subbundles,
\oldpage{139}
with $\rank(V_i)=i$.
Then the $V_i/V_{i-1}$ ($1\leq i\leq p$) are vector bundles of rank~$1$, and are called the \emph{factors} of the given splitting.

If $X$ is an algebraic space, then we denote by $\PP(X)$ the group of isomorphism classes of rank-$1$ vector bundles on $X$ (the composition law of the group being given by the tensor product of bundles).
If $L$ is such a rank-$1$ vector bundle, then we denote by $\cl_X(L)$ the element of $\PP(X)$ that it defines.
We thus have
\begin{gather*}
  \cl_X(L\otimes L') = \cl_X(L) + \cl_X(L')
\\\cl_X(\check{L}) = -\cl_X(L).
\end{gather*}

If $f\colon X\to Y$ is a morphism, then the formula
\[
  f^*(\cl_X(L)) = \cl_X(f^{-1}(L))
\]
defines a homomorphism $f^*$ from $\PP(Y)$ to $\PP(X)$.
In this way, $\PP(X)$ can be considered as a \emph{contravariant functor} in $X$.

With $f\colon X\to Y$ still a morphism, let $F$ be a rank-$p$ vector bundle on $Y$, and set $E=f^{-1}(F)$.
This is a rank-$p$ vector bundle on $X$, and we have a canonical isomorphism $\mathbb{P}(E)=f^{-1}(\mathbb{P}(F))$, whence a natural morphism
\[
  \overline{f}\colon \mathbb{P}(E) \to \mathbb{P}(F)
  \quad\mbox{[$E=f^{-1}(F)$].}
\]

With this, we can immediately verify that \emph{$L_e$ is canonically isomorphic to the inverse image $\overline{f}^{-1}(L_F)$}.
We thus have the formula
\[
  \cl(L_E) = \overline{f}^*(\cl(L_F)).
\]

Let $E$ be a rank-$p$ vector bundle on $X$, and $s$ a regular section of $E$.
This is then a morphism from $X$ to $E$, and even an isomorphism from $X$ to a closed subspace of $E$ of codimension~$p$.
In particular, the image of $X$ under the zero section is a closed non-singular subspace $X'$ of $E$ of codimension~$p$.
Evidently, the inverse image $s^{-1}(X')$ is exactly the set of zeros of $s$.
For the \emph{cycle} $s^{-1}(X')$ to be defined, it is necessary and sufficient for the set of zeros of $s$ to be everywhere empty, or of codimension~$p$ in $X$.
In this case, we can then spaek of the \emph{cycle of zeros} of the section $s$.
Recall also that the morphism $s$ is said to be \emph{transversal} to the subvariety $X'$ of $X$ if, at every point of the inverse image of $X'$ under $s$, the tangent map to $s$ is surjective mod the tangent space to $X'$.
In this case, $s^{-1}(X')$ is a non-singular algebraic subspace of $X$ that is everywhere of codimension~$p$, and all its components are of multiplicity~$1$ in the cycle of zeros of $s$.
We will say, for brevity, that the section $s$ is \emph{transversal to the zero section}.
To express this property by a calculation, since it is local on $X$, we can assume that $E$ is the trivial bundle $X\times k^p$, so that $s$ is defined by the data of $p$ regular functions $(f_1,\ldots,f_p)$ on $X$.
For $s$ to be transversal to the zero section, it is necessary and sufficient for the functions $f_1,\ldots,f_p$ to give a regular system of parameters of $\sh{O}_x$ at every point $x$.


\section{The functor $A(X)$}
\label{section2}

\oldpage{140}

In what follows, suppose that we have a category $\cat{V}$ of non-singular algebraic spaces (the morphisms in this category being the morphisms of algebraic spaces).
The only condition that we impose on $\cat{V}$ is that
\begin{enumerate}[({V}1)]
  \item\label{axiomV1}
    If $X\in\cat{V}$, and if $E$ is a vector bundle on $X$, then $\mathbb{P}(E)\in\cat{V}$.
\end{enumerate}

Suppose further that we have the following data:
\begin{enumerate}[a.]
  \item\label{dataa}
    a contravariant functor $X\mapsto A(X)$ from $\cat{V}$ to the category of unital anticommutative graded rings;
  \item\label{datab}
    a functorial homomorphism $p_X\colon\PP(X)\to A^2(X)$ (for $X\in\cat{V}$);
  \item\label{datac}
    for all $X\in\cat{V}$, and for every closed algebraic subspace $Y$ of $X$, of constant codimension~$p$ in $X$, with $Y\in\cat{V}$, a group homomorphism
    \[
      i_*\colon A(Y)\to A(X)
    \]
    (where $i$ denotes the injection $Y\to X$) that raises the degree by $2p$.
\end{enumerate}

If $f\colon X\to Y$ is a morphism in $\cat{V}$, then the corresponding homomorphism from $A(Y)$ to $A(X)$ is denoted by $f^*$.
The unit element of $A(X)$ will be denoted by $1_X$, and if $X,Y\in\cat{V}$, and if $Y$ is a closed subspace of $X$ of constant codimension~$p$, then we define $p_X(Y)=i_*(1_Y)$, where $i$ is (again) the injection morphism from $Y$ into $X$.

With this, we suppose that the following conditions are satisfied:
\begin{enumerate}[({A}1)]
  \item\label{axiomA1}
    Let $X\in\cat{V}$, and let $E$ be a rank-$p$ vector bundle on $X$, $\mathbb{P}(E)$ the associated projective bundle, and $\xi_E$ the element of $A(\mathbb{P}(E))$ defined by
    \[
      \xi_E = p(\cl(L_E)).
    \]
    We can think of $A(\mathbb{P}(E))$ as a left module over $A(X)$ via the homomorphism $f^*\colon A(X)\to A(\mathbb{P}(E))$ associated to the projection $f\colon\mathbb{P}(E)\to X$.
    Then the elements
    \[
      (\xi_E)^i
      \quad\mbox{(for $0\leq i\leq p-1$)}
    \]
    form a basis of $A(\mathbb{P}(E))$ over $A(X)$.
  \item\label{axiomA2}
    Let $X\in\cat{V}$, and let $L$ be a rank-$1$ vector bundle on $X$, and $s$ a regular section of $L$ that is transversal to the zero section, and such that the space $Y$ of zeros of $s$ belongs to $\cat{V}$.
    Then
    \[
      p_X(Y) = p_X(\cl_X(L)).
    \]
  \item\label{axiomA3}
    Let $X,Y,Z\in\cat{V}$, with $Z\subset Y\subset X$, and let $i$ and $j$ be the injection morphisms $Z\to Y$ and $Y\to X$ (respectively).
    Then $(ji)_* = j_*i_*$.
  \item\label{axiomA4}
    Let $X,Y\in\cat{V}$, with $Y\subset X$, and let $i$ be the injection morphism of $Y$
\oldpage{141}
  into $X$.
    Then
    \[
      i_*(bi^*(a)) = i_*(b)a
      \quad\mbox{[$a\in A(X),b\in A(Y)$].}
    \]
\end{enumerate}

From these axioms we will prove two lemmas that will be useful in the \hyperref[section3]{next section}.

\begin{lemma}{1}
\label{lemma1}
  Let $X\in\cat{V}$, and let $E$ be a rank-$p$ vector bundle on $X$.
  Consider, for all $i$ with $1\leq i\leq p$, the bundle $X^{(i)}$ of flags of length $i$ in $E$.
  Then $X^{(i)}\in\cat{V}$, and the homomorphism $A(X)\to A(X^{(i)})$ induced by the projection $X^{(i)}\to X$ is \emph{injective}.
\end{lemma}

\begin{proof}
  By what was said in \cref{section1}, we can restrict, by induction on $i$, to the case of the projective bundle $X^{(1)}=\mathbb{P}(E)$ associated to $E$.
  Then our claims are an immediate consequence of \hyperref[axiomV1]{(V1)} and \hyperref[axiomA1]{(A1)} [the element $1_{\mathbb{P}(E)}=(\xi_E)^0$ being free over the ring $A(X)$].
\end{proof}

\begin{lemma}{2}
\label{lemma2}
  Let $X\in\cat{V}$, and let $E$ be a rank-$p$ vector bundle on $X$, $s$ a regular section of $E$, and $(E_i)_{0\leq i\leq p}$ a decreasing sequence of vector subbundles of $E$, with $\rank E_i=p-i$.
  Define
  \[
    \xi_i = p_X\cl_X(E_{i-1}/E_i)
    \quad\mbox{($1\leq i\leq p$).}
  \]
  For every $1\leq i\leq p$, let $Y_i$ be the subset of $X$ consisting of those $x\in X$ such that $s(x)\in E_i$.
  Suppose that, for all $i$, $Y_i$ is a non-singular subvariety of $X$, and that $Y_i\in\cat{V}$.
  Let $s_i$ be the section of $(E_i/E_{i+1})|Y_i$ defined by $s$, and suppose that, for $1\leq i\leq p-1$, $s_i$ is transversal to the zero section.
  Then, under these conditions,
  \[
    p_X(Y_p) = \prod_{1\leq i\leq p}\xi_i.
  \]
\end{lemma}

\begin{proof}
  We will proceed by induction on $j$ (where $1\leq j\leq p$) that
  \[
    \label{lemma2equationstar}
    p_X(Y_j) = \prod_{1\leq i\leq j}\xi_i.
    \tag{$\star$}
  \]

  For $j=1$, this equation is exactly axiom~\hyperref[axiomA2]{(A2)}.
  Suppose that the equation has been proven for some $j<p$, and we will prove it for $j+1$.
  Applying \hyperref[axiomA2]{(A2)} to the section $s_j$ of $(E_j/E_{j+1})|Y_j$, we see that
  \[
    p_{Y_j}(Y_{j+1}) = p_{Y_j}\cl_{Y_j}((E_j/E_{j+1})|Y_j).
  \]
  Let $u_j$ be the injection morphism $Y_j\to X$.
  From the functoriality of $\cl$ and $p$ we see that the right-hand side of the above equation is $u_j^*(p_X\cl_X(E_j/E_{j+1}))$, whence
  \[
    p_{Y_j}(Y_{j+1}) = u_j^*(\xi_{j+1}).
  \]

  \oldpage{142}
  Using \hyperref[axiomA3]{(A3)}, we thus conclude that
  \[
    p_X(Y_{j+1}) = (u_j)_*(u_j^*(\xi_{j+1})).
  \]

  The right-hand side can also be written as $(u_j)_*(1_{y_j}u_j^*(\xi_{j+1}))$, which is equal, by \hyperref[axiomA4]{(A4)}, to $(u_j)_*(1_{Y_j})\xi_{j+1} = p_X(Y_j)\xi_{j+1}$.
  Using the induction hypothesis \hyperref[lemma2equationstar]{($\star$)}, we indeed recover the analogous formula with $j+1$ instead of $j$.
\end{proof}

It is only in \cref{section3} that we will need the following corollary.
\begin{corollary}
\phantomsection
\label{lemma2corollary}
  Under the conditions of \cref{lemma2}, if $s$ is non-zero at every point (i.e. everywhere non-zero), then $\prod_{1\leq i\leq p}\xi_i=0$.
\end{corollary}

\begin{remark}
  The introduction of the operation $i$ in \hyperref[datac]{c.}, and the axioms \hyperref[axiomA2]{(A2)}, \hyperref[axiomA3]{(A3)}, and \hyperref[axiomA4]{(A4)} concerning this operation, only serve to provide us with technical means of being able to prove \hyperref[lemma2corollary]{the corollary} to \cref{lemma2}.
  (In \hyperref[axiomA4]{(A4)}, it would have sufficed to suppose that $b=1_Y$.)
  We will only need to use the data of \hyperref[dataa]{a.} and \hyperref[datab]{b.}, axiom~\hyperref[axiomA2]{(A1)}, and \hyperref[lemma2corollary]{the corollary} to \cref{lemma2}.
\end{remark}


\subsection*{Particular cases}

Note first of all that the condition \hyperref[axiomV1]{(V1)} is satisfied for all reasonable categories of algebraic spaces, and, anyway, for the category of all non-singular algebraic spaces, for the category of non-singular quasi-projective algebraic spaces, and for the category of non-singular projective algebraic spaces.
The verification in these two latter cases presents no difficulty, and is left to the reader (this result being a particular case of a more general result on blow-up varieties).

We now given some particular cases where the conditions in this section are satisfied.

\begin{enumerate}
  \item
    $\cat{V}$ is the category of non-singular quasi-projective algebraic spaces, and $A(X)$ is the ring of cycle classes on $X$ under \emph{rational equivalence}, with the usual definition of $f^*$ and $f_*$.
    Of course, we grade $A(X)$ by taking the class of a cycle on $X$ that is everywhere of codimension~$p$ to be of degree~$2p$ [so that $A(X)$ has only even degrees, as we would expect, in a cohomological theory, for a graded \emph{commutative} ring].
    The homomorphism $\PP(X)\to A^2(X)$ is an isomorphism, given by sending any rank-$1$ vector bundle $L$ on $X$ to the set of divisors of rational sections of $L$ that are not zero on any component of $X$.
    For the theory of linear equivalence, including the verification of properties \hyperref[axiomA1]{(A1)} to \hyperref[axiomA4]{(A4)} (with only \hyperref[axiomA1]{(A1)} not being immediate), see the expos\'{e}s by Chevalley and Grothendieck in \cite{4}.

    The conditions that we demand are also satisfied by taking $A(X)$ to be the ring of cycle classes under \emph{algebraic} equivalence.
    But for a theory
\oldpage{143}
    of Chern classes, we rather prefer to work with rational equivalence, which gives a finer theory.

    We cannot yet define a ring structure on the group $A(X)$ of cycle classes on an \emph{arbitrary} (not necessarily quasi-projective) non-singular variety, nor, for a morphism $f\colon X\to Y$, a morphism $f^*\colon A(Y)\to A(X)$ in such a way that the necessary conditions are satisfied.
    Further, it is not even certain that this might be possible.
    We can think of replacing the ring of cycle classes (under rational equivalence) by the graded ring associated to the ring $K(X)$ of classes of coherent algebraic sheaves on $X$, filtered in the natural way (by considering the dimension of the supports of the sheaves).
    Unfortunately, we would have to prove that this filtration is compatible with the ring ring structure (and with the ``inverse image'' homomorphisms), which I do not know how to do in the quasi-projective case, by using the rational equivalence\todo.
    However, it seems that these difficulties disappear when we tensor with the field of rationals $\mathbb{Q}$, i.e. if we ignore the phenomenons of torsion.

  \item
    $\cat{V}$ is the category of all non-singular algebraic spaces.
    If $X$ is such a variety, then we denote by $\Omega_X^\bullet$ the sheaf of germs of regular differential forms on $X$, and by $A(X)$ the cohomology group $\HH^\bullet(X,\Omega_X^\bullet)$.
    We grade this group by taking $\HH^p(X,\Omega_X^q)$ to be of degree $p+q$, and we make this an algebra by means of the cup product.
    We thus obtain an anticommutative graded algebra, which is clearly a contravariant functor with respect to $X$.
    By following the formalism developed by Grothendieck in \cite{3}, we define in a natural way the homomorphisms $i_*\colon A(Y)\to A(X)$ associated to an injection $i\colon Y\to X$ (and it is probably possible to define $i_*$ for every \emph{proper} morphism $i\colon Y\to X$).
    \renewcommand*{\thefootnote}{\fnsymbol{footnote}}\footnote{(Note added during editing). This homomorphism $i_*$ is now defined in full generality.}
    Finally, we define a morphism $\PP(X)\to\HH^1(X,\Omega_X^1)\subset A^2(X)$ in a classical way, by writing, for example, $\PP(X)=\HH^1(X,\sh{O}_X^\times)$ (where $\sh{O}_X^\times$ denotes the sheaf of germs of invertible regular functions on $X$), and by considering the homomorphism $f\mapsto\mathrm{d}f/f$ from $\sh{O}_X^\times$ to $\Omega_X^1$.
    We can again easily verify that conditions \hyperref[axiomA1]{(A1)} to \hyperref[axiomA4]{(A4)} are satisfied, with \hyperref[axiomA1]{(A1)} being a consequence of the Leray spectral sequence of the continuous map $\mathbb{P}(E)\to X$ [the spectral sequence being trivial, as follows from considering from the class $\xi_E$ on $\mathbb{P}(E)$.]

  \item
    The base field $k$ is the field of complex numbers, $\cat{V}$ is the category of non-singular algebraic spaces, and $A(X)=\HH^\bullet(X,\mathbb{Z})$ (with $X$ being endowed with its ``usual'' topology).
    The definition of \hyperref[datab]{b.} (either by Poincar\'{e} duality on divisor classes, or as an obstruction class in the classical exact sequence $0\to\mathbb{Z}\to\sh{O}_X\to\sh{O}_X^\times\to0$ of sheaves on $X$, endowed with its usual topology) is well known.
    The definition of \hyperref[datac]{c.} classical comes
\oldpage{144}
    from Poincar\'{e} duality, and properties \hyperref[axiomA1]{(A1)} to \hyperref[axiomA4]{(A4)} are well known (with \hyperref[axiomA2]{(A2)} again following from the Leray spectral sequence).
\end{enumerate}


\section{Definition and fundamental properties of Chern classes}
\label{section3}

Let $X\in\cat{V}$, and let $E$ be a rank-$p$ vector bundle on $X$, and $\xi_E=p_X(\cl(L_E))$ the fundamental class in $A^2(\mathbb{P}(E))$.
By axiom~\hyperref[axiomA1]{(A1)} of the \hyperref[section2]{previous section}, $(\xi_E)^p$ can be written as a unique linear combination of the $(\xi_E)^i$ (for $0\leq i\leq p-1$) with coefficients in $A(X)$.
This means that we can find, in a unique way, elements $c_i(E)\in A^{2i}(X)$ (defined for every integer $i\geq0$) satisfying the conditions
\[
\label{equation1}
  \begin{gathered}
    \sum_{i=0}^p c_i(E)(\xi_E)^{p-i} = 0,
  \\c_0(E)=1,\quad\text{and}\quad c_i(E)=0\mbox{ for $i>p$.}
  \end{gathered}
\tag{1}
\]
The $c_i(E)$ are called the \emph{Chern classes} of $E$, with $c_i(E)$ being the $i$th Chern class.
We set
\[
\label{equation2}
  c(E) = \sum_i c_i(E)
\tag{2}
\]
and $c(E)$ is called the \emph{(total) Chern class} of $E$;
its data is thus equivalent to the data of all the $c_i(E)$.

\begin{theorem}{1}
\label{theorem1}
  Suppose that we have the data of \hyperref[dataa]{\rm{a.}}, \hyperref[datab]{\rm{b.}}, and \hyperref[datac]{\rm{c.}} of the \hyperref[section2]{previous section}, satisfying axioms \hyperref[axiomA1]{\rm{(A1)}} to \hyperref[axiomA4]{\rm{(A4)}}.
  Then the Chern classes (defined by \cref{equation1}) satisfy the following conditions:
  \begin{enumerate}[\rm(i)]
    \item\label{theorem1i}
      \emph{Functoriality. ---}
      Let $f\colon X\to Y$ be a morphism in $\cat{V}$, and let $E$ be a vector bundle on $Y$.
      Then
      \[
      \label{equation3}
        c(f^{-1}(E)) = f^*(c(E))
      \tag{3}
      \]
      [where $f^{-1}(E)$ denotes the vector bundle on $X$ given by the inverse image of $E$ under $f$].
    \item\label{theorem1ii}
      \emph{Normalisation. ---}
      If $E$ is a rank-$1$ vector bundle on $X\in\cat{V}$, then
      \[
      \label{equation4}
        c(E) = 1+p_X(\cl_X(E)).
      \tag{4}
      \]
    \item\label{theorem1iii}
      \emph{Additivity. ---}
      Let $X\in\cat{V}$, and let $0\to E'\to E\to E''\to 0$ be an exact sequence of vector bundles on $X$.
      Then
      \[
      \label{equation5}
        c(E) = c(E')c(E'').
      \tag{5}
      \]
  \end{enumerate}

  Furthermore, properties \hyperref[theorem1i]{\rm{(i)}}, \hyperref[theorem1ii]{\rm{(ii)}}, and \hyperref[theorem1iii]{\rm{(iii)}} entirely \emph{characterise} Chern classes.
\end{theorem}

\begin{proof}
\oldpage{145}
  We first prove the \emph{uniqueness} of a theory of Chern classes satisfying \hyperref[theorem1i]{\rm{(i)}}, \hyperref[theorem1ii]{\rm{(ii)}}, and \hyperref[theorem1iii]{\rm{(iii)}}.
  Let $X\in\cat{V}$, and let $E$ be a rank-$p$ vector bundle on $X$, and $X'$ the flag variety associated to $E$, with $f'\colon X\to X$ the canonical projection.
  By \cref{lemma1}, we know that $X'\in\cat{V}$ and that $f^*\colon A(X)\to A(X')$ is injective.
  Thus $c(E)$ is known if we known $f(c(E))$, which, by \hyperref[theorem1i]{\rm{(i)}}, is equal to $c(f^{-1}(E))$.
  But $f^{-1}(E)$ splits completely.
  We are thus reduced to determining $c(E)$ when $E$ is a rank-$p$ vector bundle, which splits completely, and is thus endowed with a composition series $(E_i)_{0\leq i\leq p}$, with $\rank E_i=p-i$,
  But then the additivity formula \hyperref[theorem1iii]{\rm{(iii)}} proves (by induction on $p$) that $c(E)=\prod_{i=1}^p c(E_{i-1}/E_i)$, and finally the normalisation formula \hyperref[theorem1ii]{\rm{(ii)}} implies that
  \[
  \label{equation6}
    c(E) = \prod_{i=1}^p (1+p_X\cl_X(E_{i-1}/E_i)).
  \tag{6}
  \]

  We now show that the Chern classes defined by \cref{equation1} do indeed satisfy \hyperref[theorem1i]{\rm{(i)}}, \hyperref[theorem1ii]{\rm{(ii)}}, and \hyperref[theorem1iii]{\rm{(iii)}}.

  \begin{proof}[Proof of \rm{(i)}]
    \todo
  \end{proof}

  \begin{proof}[Proof of \rm{(ii)}]
    \todo
  \end{proof}

  \begin{proof}[Proof of \rm{(iii)}]
    \todo
  \end{proof}

  So we are under the conditions of \hyperref[lemma2corollary]{the corollary} to \cref{lemma2} of the \hyperref[section2]{previous section}, which implies that
  \[
    \prod_{i=1}^p (\xi_E+\xi'_i) = 0.
  \]
  This proves, by the definition of the $c_i(E)$, that the $c_i(E)$ are the elementary symmetric functions in the $\xi_i$, which is exactly \cref{equation6}.

  This finishes the proof of \cref{theorem1}.
\end{proof}

\begin{corollary}
  Let $X\in\cat{V}$, and let $E$ and $F$ be vector bundles on $X$.
  Then
  \[
  \label{equation7}
    c_i(\check{E}) = (-1)^i c_i(E)
  \tag{7}
  \]
  and, similarly, the Chern classes of the exterior powers $\bigwedge E$ (resp. of the tensor product $E\otimes F$) can be expressed in terms of the Chern classes of $E$ and the rank of $E$ (resp. in terms of the Chern classes of $E$ and $F$ and the ranks of $E$ and $F$) by the well-known calculation of elementary functions (see the book by Hirzebruch).
\end{corollary}

\begin{proof}
  Passing to a flag variety, as per usual, we can restrict to the case where $E$ and $F$ are completely split.
  In this case, the formulas follow immediately from \cref{equation6}.
\end{proof}


\section{Remarks and various supplements}
\label{section4}


%% Bibliography %%

\nocite{*}
\bibliographystyle{acm}

\begin{thebibliography}{4}

  \bibitem{1}
  {\sc Atiyah, M.}
  \newblock Complex analytic connections in fibre bundles.
  \newblock {\em Trans. Amer. math. Soc.} {\bf 85} (1957), pp.~181--207.

  \bibitem{2}
  {\sc Chern, Shung-Shen.}
  \newblock On the characteristic classes of complex sphere bundles and algebraic varieties.
  \newblock {\em Amer. J. Math.} {\bf 75} (1953), pp.~565--597.

  \bibitem{3}
  {\sc Grothendieck, Alexander.}
  \newblock Th\'{e}or\`{e}me de dualit\'{e} four les faisceaux alg\'{e}briques coh\'{e}rents.
  \newblock {\em S\'{e}minaire Bourbaki} {\bf 9}, no.~149 (1956--57).

  \bibitem{4}
  ``Classification des groupes de Lie''.
  \newblock {\em S\'{e}minaire Chevalley}, {Volume~1} (1956--58).

\end{thebibliography}

\end{document}
