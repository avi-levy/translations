\documentclass{article}

\usepackage{amssymb,amsmath}

\usepackage{hyperref}
\usepackage[nameinlink]{cleveref}
\usepackage{enumerate}

\usepackage{mathrsfs}
%% Fancy fonts --- feel free to remove! %%
\usepackage{Baskervaldx}
\usepackage{mathpazo}


\crefname{section}{Section}{Sections}
\crefname{equation}{}{}


%% Theorem environments %%

\usepackage{amsthm}

  \theoremstyle{plain}

  \newtheorem{innercustomtheorem}{Theorem}
  \crefname{innercustomtheorem}{Theorem}{Theorems}
  \newenvironment{theorem}[1]
    {\renewcommand\theinnercustomtheorem{#1}\innercustomtheorem}
    {\endinnercustomtheorem}

  \newtheorem{innercustomproposition}{Proposition}
  \crefname{innercustomproposition}{Proposition}{Propositions}
  \newenvironment{proposition}[1]
    {\renewcommand\theinnercustomproposition{#1}\innercustomproposition}
    {\endinnercustomproposition}

  \newtheorem{innercustomlemma}{Lemma}
  \crefname{innercustomlemma}{Lemma}{Lemmas}
  \newenvironment{lemma}[1]
    {\renewcommand\theinnercustomlemma{#1}\innercustomlemma}
    {\endinnercustomlemma}

  \newtheorem{innercustomcorollary}{Corollary}
  \crefname{innercustomcorollary}{Corollary}{Corollaries}
  \newenvironment{corollary}[1]
    {\renewcommand\theinnercustomcorollary{#1}\innercustomcorollary}
    {\endinnercustomcorollary}


  \theoremstyle{definition}

  \newtheorem*{remark}{Remark}

  \newtheorem{innercustomdefinition}{Definition}
  \crefname{innercustomdefinition}{Definition}{Definitions}
  \newenvironment{definition}[1]
    {\renewcommand\theinnercustomdefinition{#1}\innercustomdefinition}
    {\endinnercustomdefinition}


%% Shortcuts %%

\newcommand{\sh}{\mathscr}
\newcommand{\cat}{\mathcal}

\renewcommand{\geq}{\geqslant}
\renewcommand{\leq}{\leqslant}

\newcommand{\todo}{\textbf{ !TODO! }}
\newcommand{\oldpage}[1]{\marginpar{\footnotesize$\Big\vert$ \textit{p.~#1}}}


%% Document %%

\usepackage{embedall}
\begin{document}

\renewcommand{\abstractname}{Translator's note.}

\title{The theory of Chern classes}
\author{A. Grothendieck}
\date{}
\maketitle

\begin{abstract}
  \renewcommand*{\thefootnote}{\fnsymbol{footnote}}
  \emph{This text is one of a series\footnote{\url{https://github.com/thosgood/translations}} of translations of various papers into English.}
  \emph{The translator takes full responsibility for any errors introduced in the passage from one language to another, and claims no rights to any of the mathematical content herein.}
  
  \emph{What follows is a translation (last updated \today) of the French paper:}

  \medskip\noindent
  \textsc{Grothendieck, A.}
  ``La th\'{e}orie des classes de Chern''.
  \emph{Bulletin de la Soci\'{e}t\'{e} Math\'{e}matique de France}, Volume~\textbf{86} (1958) , pp.~137-154.
  \textsc{doi}: \href{https://www.doi.org/10.24033/bsmf.1501}{10.24033/bsmf.1501}.
\end{abstract}

\setcounter{footnote}{0}

\tableofcontents


%% Content %%

\bigskip\bigskip


\section*{Introduction}

In this appendix, we will develop an axiomatic theory of Chern classes that will allow us, in particular, to define the Chern classes of an algebraic vector bundle $E$ on a non-singular quasi-projective algebraic variety $X$ as elements of the Chow ring $A(X)$ of $X$, i.e. as classes of cycles under rational equivalence.
This expos\'{e} is inspired by the book of Hirzebruch on one hand (where the essential \emph{formal properties} characterising a theory of Chern classes was brought to light), and by an idea of Chern \cite{2} that consists of using the multiplicative structure of the ring of classes of cycles on bundle of projective spaces $P(E)$ associated to $E$, to reach an effective \emph{construction} of Chern classes.
We note that the exposition given here \todo



%% Bibliography %%

\nocite{*}
\bibliographystyle{acm}

\begin{thebibliography}{4}

  \bibitem{1}
  {\sc Atiyah, M.}
  \newblock Complex analytic connections in fibre bundles.
  \newblock {\em Trans. Amer. math. Soc.} {\bf 85} (1957), pp.~181--207.

  \bibitem{2}
  {\sc Chern, Shung-Shen.}
  \newblock On the characteristic classes of complex sphere bundles and algebraic varieties.
  \newblock {\em Amer. J. Math.} {\bf 75} (1953), pp.~565--597.

  \bibitem{3}
  {\sc Grothendieck, Alexander.}
  \newblock Th\'{e}or\`{e}me de dualit\'{e} four les faisceaux alg\'{e}briques coh\'{e}rents.
  \newblock {\em S\'{e}minaire Bourbaki} {\bf 9}, no.~149 (1956--57).

  \bibitem{4}
  ``Classification des groupes de Lie''.
  \newblock {\em S\'{e}minaire Chevalley}, {Volume~1} (1956--58).

\end{thebibliography}

\end{document}
