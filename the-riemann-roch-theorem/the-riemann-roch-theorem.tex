\documentclass[10pt]{scrartcl}
\DeclareOldFontCommand{\sc}{\normalfont\scshape}{\@nomath\sc}

\usepackage{amsthm,amssymb,amsmath}
\newcommand{\todo}{\textbf{ !TODO! }}

\begin{document}

\title{The Riemann-Roch theorem}
\subtitle{Following some unpublished results by A.~Grothendieck}
\author{Armand BOREL and Jean-Pierre SERRE}
\date{1958}
\maketitle

\begin{abstract}
\todo
\textbf{Translator's note.}
{What follows is a translation of the paper ``{Le théorème de Riemann-Roch}'', Armand Borel and Jean-Pierre Serre. \emph{Bulletin de la S.M.F.}, tome~86 (1958), p.~97--136}.

\end{abstract}


% Content

\section*{Introduction}
What follows constitutes the notes from a seminar that took place in Princeton in the autumn of 1957 on the works of Grothendieck;
the new results that are included are due to Grothendieck;
our contribution is solely of an editorial nature.

The ``Riemann-Roch theorem'' of which we speak here holds true for (non-singular) algebraic varieties over a field of arbitrary characteristic;
in the classical case, where the base field is $\mathbb{C}$, this theorem encapsulates, as a particular example, the result proven a few years ago by Hirzebruch.\cite{9}


% Bibliography

\bibliographystyle{acm}
\bibliography{\jobname}

\end{document}
