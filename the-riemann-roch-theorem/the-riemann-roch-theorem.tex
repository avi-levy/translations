\documentclass[10pt]{scrartcl}
\DeclareOldFontCommand{\sc}{\normalfont\scshape}{\@nomath\sc}

\usepackage{amssymb,amsmath}
\usepackage{mathrsfs}


% Theorem environments

\usepackage{amsthm}

\theoremstyle{plain}

\newtheorem{innercustomproposition}{Proposition}
\newenvironment{proposition}[1]
  {\renewcommand\theinnercustomproposition{#1}\innercustomproposition}
  {\endinnercustomproposition}

\newtheorem{innercustomlemma}{Lemma}
\newenvironment{lemma}[1]
  {\renewcommand\theinnercustomlemma{#1}\innercustomlemma}
  {\endinnercustomlemma}


% Shortcuts

\newcommand{\sh}{\mathscr}

\newcommand{\todo}{\textbf{ !TODO! }}


\begin{document}

\title{The Riemann-Roch theorem}
\subtitle{Following some unpublished results by A.~Grothendieck}
\author{Armand BOREL and Jean-Pierre SERRE}
\date{1958}
\maketitle

\begin{abstract}
\todo

\textbf{Translator's note.}
{What follows is a translation of the paper ``{Le théorème de Riemann-Roch}'', Armand Borel and Jean-Pierre Serre. \emph{Bulletin de la S.M.F.}, tome~86 (1958), p.~97--136}.

\todo \textbf{old pages}
\end{abstract}


% Content

\section*{Introduction}
What follows constitutes the notes from a seminar that took place in Princeton in the autumn of 1957 on the works of Grothendieck;
the new results that are included are due to Grothendieck;
our contribution is solely of an editorial nature.

The ``Riemann-Roch theorem'' of which we speak here holds true for (non-singular) algebraic varieties over a field of arbitrary characteristic;
in the classical case, where the base field is $\mathbb{C}$, this theorem encapsulates, as a particular example, the result proven a few years ago by Hirzebruch \cite{9}.

The full statement and proof of the Riemann-Roch theorem can be found in sections 7 to 16, with the last section being devoted to an application.
Sections 1 to 6 contain some preliminaries on coherent algebraic sheaves \cite{12}.
The terminology that we follow is the same as in \cite{12}, up to one difference: to conform with a custom which is becoming more and more widespread, we use the word ``morphism'' instead of ``regular maps''.


\section{Auxiliary results on sheaves}

(All the varieties considered below are algebraic varieties over an algebraically closed field $k$ of arbitrary characteristic. Unless otherwise mentioned, all the sheaves considered are coherent algebraic sheaves.)

\begin{proposition}{1}
  Let $U$ be an open subset of a variety $V$, and let $\sh{F}$ be a coherent sheaf on $V$ and $\sh{G}$ a coherent subsheaf of $\sh{F}|U$ (the restriction of $\sh{F}$ to $U$).
  Then there exists a coherent sheaf $\sh{G}'\subset\sh{F}$ such that $\sh{G}'|U=\sh{G}$.
\end{proposition}

(In fact, the proof will show that there exists a \emph{largest} such sheaf having this property.)

\begin{proof}
  For every open subset $W\subset V$, we define $\sh{G}'_W$ as the set of sections of $\sh{F}$ over $W$ that belong to $\sh{G}$ over $U\cap W$.
  Everything reduces to showing that the sheaf $\sh{G}'$ associated to this presheaf is coherent.
  Since this is a local questions, we can suppose that $V$ is an affine variety.
  Let $A$ be its coordinate ring.
  There exist elements $f_i\in A$ such that $U=\bigcup V_{f_i}$, where $V_{f_i}=U_i$ denotes the set of points of $V$ where $f_i\neq0$.
  If, in the definition of $\sh{G}'$, we replace the open subset $U$ by the open subset $U_i$, then we obtain a sheaf $\sh{G}'_i\subset\sh{F}$, and it is clear that $\sh{G}'=\bigcap\sh{G}'_i$.
  By known results on coherent sheaves \cite[p.~209]{12}, it suffices to show that the $\sh{G}'_i$ are coherent.
  We can thus restrict to considering the case where $V$ is affine, and where $U=V_f$, for some $f\in A$.
  In this case, the sheaf $\sh{F}$ is defined by an $A$-module $M$, and the subsheaf $\sh{G}$ of $\sh{F}|U$ is defined by a submodule $N$ of $M_f=M\otimes_A A_f$.
  Let $N'$ be the inverse image of $N$ in $M$ under the canonical map $M\to M_f$.
  The module $N'$ then corresponds to a coherent subsheaf of $\sh{F}$, and we can immediately verify (by taking the $V_{f'}$ to be the $W$, for example) that this sheaf is exactly $\sh{G}'$, which finishes the proof.
\end{proof}

\begin{lemma}{1}
  Let $U$ be an open subset of an affine variety $V$, and let $\sh{F}$ be a (coherent) sheaf on $U$.
  Then $\sh{F}$ is generated by its sections (over $U$).
\end{lemma}

\begin{proof}
  Let $x\in U$, and let $f$ be a regular function on $V$, zero on $V\setminus U$ and non-zero at $x$.
  We have $V_f\subset U\subset V$.
  Since $V_f$ is affine, we know \cite{12} that $\sh{F}_x$ is generated by its sections over $V_f$, and it thus suffices to prove that these sections can be extended to $U$, after multiplying by a suitable power of $f$.
  This follows from the more general following lemma:
\end{proof}

\begin{lemma}{2}
  Let $X$ be a variety, $f$ a regular function on $X$, $\sh{F}$ a sheaf on $X$, and $s$ a section of $\sh{F}$ over $U=X_f$.
  Then there exists an integer $n>0$ such that $f^ns$ can be extended to a section of $\sh{F}$ over $X$.
\end{lemma}

\begin{proof}
  We can cover $X$ by finitely many affine opens $X_i$.
  By applying \cite[lemma~1, p.~247]{12} (or by arguing directly, as in proposition~1), we see that there exists an integer $n$ and sections $s_i$ of $\sh{F}$ over the $X_i$ that extend $f^ns$ over $X_i\cap U$.
  Since the $s_i-s_j$ are zero on $X_i\cap X_j\cap U$, there exists an integer $m$ such that $f^m(s_i-s_j)=0$ on $X_i\cap X_j$ (\cite[p.~235]{12}, or arguing directly), and $m$ can be chosen independent of the pair $(i,j)$.
  The $f^ms_i$ then define a section $s'$ of $\sh{F}$ over $X$ that indeed extend $f^{n+m}s$.
\end{proof}

\begin{proposition}{2}
  If $U$ is an open subset of a variety $V$, then every sheaf $\sh{F}$ over $U$ can be extended to $V$.
\end{proposition}

\begin{proof}
  We show that, if $U\neq V$, we can extend $\sh{F}$ to an open subset $U'\supset U$, with $U'\neq U$;
  from that fact that every (strictly) increasing chain of open subsets stabilises, this will imply the proposition.
  Let $x\in V\setminus U$, and let $W$ be an affine open that contains $x$;
  let $U'=W\cup U$.
\end{proof}


% Bibliography

\nocite{*}
\bibliographystyle{acm}
\bibliography{\jobname}

\end{document}
