\documentclass{article}

\usepackage{amssymb,amsmath}
\usepackage{hyperref}
\usepackage[nameinlink]{cleveref}
\usepackage{enumerate}
\usepackage{mathrsfs}
%% Fancy fonts --- feel free to remove! %%
\usepackage{Baskervaldx}
\usepackage{mathpazo}

\usepackage{tikz-cd}


%% Theorem environments %%

\usepackage{amsthm}

  \theoremstyle{plain}

  \newtheorem{innercustomtheorem}{Theorem}
  \crefname{innercustomtheorem}{Theorem}{Theorems}
  \newenvironment{theorem}[1]
    {\renewcommand\theinnercustomtheorem{#1}\innercustomtheorem}
    {\endinnercustomtheorem}

  \newtheorem{innercustomproposition}{Proposition}
  \crefname{innercustomproposition}{Proposition}{Propositions}
  \newenvironment{proposition}[1]
    {\renewcommand\theinnercustomproposition{#1}\innercustomproposition}
    {\endinnercustomproposition}

  \newtheorem{innercustomlemma}{Lemma}
  \crefname{innercustomlemma}{Lemma}{Lemmas}
  \newenvironment{lemma}[1]
    {\renewcommand\theinnercustomlemma{#1}\innercustomlemma}
    {\endinnercustomlemma}

  \newtheorem{innercustomcorollary}{Corollary}
  \crefname{innercustomcorollary}{Corollary}{Corollaries}
  \newenvironment{corollary}[1]
    {\renewcommand\theinnercustomcorollary{#1}\innercustomcorollary}
    {\endinnercustomcorollary}


  \theoremstyle{definition}

  \newtheorem*{remark}{Remark}
  \newtheorem*{remarks}{Remarks}
  \newtheorem*{definition}{Definition}
  \newtheorem*{examples}{Examples}


%% Shortcuts %%

\newcommand{\sh}{\mathscr}
\newcommand{\cat}{\mathcal}
\newcommand{\HH}{\mathrm{H}}
\newcommand{\RR}{\mathrm{R}}
\newcommand{\PP}{\mathbf{P}}
\renewcommand{\geq}{\geqslant}
\renewcommand{\leq}{\leqslant}

\DeclareMathOperator{\Tor}{Tor}
\DeclareMathOperator{\ch}{ch}
\DeclareMathOperator{\rank}{rank}

\newcommand{\todo}{\textbf{ !TODO! }}
\newcommand{\oldpage}[1]{\marginpar{\footnotesize$\Big\vert$ \textit{p.~#1}}}


\renewcommand{\thesubsection}{\thesection.\alph{subsection}}
\crefname{equation}{}{}
\crefname{subsection}{§}{§§}


%% Document %%

\usepackage{embedall}
\begin{document}

\renewcommand{\abstractname}{Translator's note.}

\title{The Riemann-Roch theorem}
\author{Armand BOREL and Jean-Pierre SERRE\\(Following some unpublished results by A.~Grothendieck)}
\date{}
\maketitle

\begin{abstract}
  \renewcommand*{\thefootnote}{\fnsymbol{footnote}}
  \emph{This text is one of a series\footnote{\url{https://github.com/thosgood/translations}} of translations of various papers into English.}
  \emph{The translator takes full responsibility for any errors introduced in the passage from one language to another, and claims no rights to any of the mathematical content herein.}
  
  \emph{What follows is a translation (last updated \today) of the French paper:}

  \medskip\noindent
  \textsc{Borel, Armand}; \textsc{Serre, Jean-Pierre}. Le théorème de Riemann-Roch. \emph{Bulletin de la Société Mathématique de France}, Volume~\textbf{86} (1958) , pp.~97-136. \textsc{doi}: \href{https://www.doi.org/10.24033/bsmf.1500}{10.24033/bsmf.1500}.
\end{abstract}

\tableofcontents


%% Content %%

\section*{Introduction}

\oldpage{97}
What follows constitutes the notes from a seminar that took place in Princeton in the autumn of 1957 on the works of Grothendieck;
the new results that are included are due to Grothendieck;
our contribution is solely of an editorial nature.

The ``Riemann-Roch theorem'' of which we speak here holds true for (non-singular) algebraic varieties over a field of arbitrary characteristic;
in the classical case, where the base field is $\mathbb{C}$, this theorem encapsulates, as a particular example, the result proven a few years ago by Hirzebruch \cite{9}.

The full statement and proof of the Riemann-Roch theorem can be found in sections 7 to 16, with the last section being devoted to an application.
Sections 1 to 6 contain some preliminaries on coherent algebraic sheaves \cite{12}.
The terminology that we follow is the same as in \cite{12}, up to one difference: to conform with a custom which is becoming more and more widespread, we use the word ``morphism'' instead of ``regular maps''.


\section{Supplementary results concerning sheaves}
\label{section1}

(All the varieties considered below are algebraic varieties over an algebraically closed field $k$ of arbitrary characteristic. Unless otherwise mentioned, all the sheaves considered are coherent algebraic sheaves.)

\begin{proposition}{1}
\label{proposition1}
\oldpage{98}
  Let $U$ be an open subset of a variety $V$, and let $\sh{F}$ be a coherent sheaf on $V$ and $\sh{G}$ a coherent subsheaf of $\sh{F}|U$ (the restriction of $\sh{F}$ to $U$).
  Then there exists a coherent sheaf $\sh{G}'\subset\sh{F}$ such that $\sh{G}'|U=\sh{G}$.
\end{proposition}

(In fact, the proof will show that there exists a \emph{largest} such sheaf having this property.)

\begin{proof}
  For every open subset $W\subset V$, we define $\sh{G}'_W$ as the set of sections of $\sh{F}$ over $W$ that belong to $\sh{G}$ over $U\cap W$.
  Everything reduces to showing that the sheaf $\sh{G}'$ associated to this presheaf is coherent.
  Since this is a local questions, we can suppose that $V$ is an affine variety.
  Let $A$ be its coordinate ring.
  There exist elements $f_i\in A$ such that $U=\bigcup V_{f_i}$, where $V_{f_i}=U_i$ denotes the set of points of $V$ where $f_i\neq0$.
  If, in the definition of $\sh{G}'$, we replace the open subset $U$ by the open subset $U_i$, then we obtain a sheaf $\sh{G}'_i\subset\sh{F}$, and it is clear that $\sh{G}'=\bigcap\sh{G}'_i$.
  By known results on coherent sheaves \cite[p.~209]{12}, it suffices to show that the $\sh{G}'_i$ are coherent.
  We can thus restrict to considering the case where $V$ is affine, and where $U=V_f$, for some $f\in A$.
  In this case, the sheaf $\sh{F}$ is defined by an $A$-module $M$, and the subsheaf $\sh{G}$ of $\sh{F}|U$ is defined by a submodule $N$ of $M_f=M\otimes_A A_f$.
  Let $N'$ be the inverse image of $N$ in $M$ under the canonical map $M\to M_f$.
  The module $N'$ then corresponds to a coherent subsheaf of $\sh{F}$, and we can immediately verify (by taking the $V_{f'}$ to be the $W$, for example) that this sheaf is exactly $\sh{G}'$, which finishes the proof.
\end{proof}

\begin{lemma}{1}
\label{lemma1}
  Let $U$ be an open subset of an affine variety $V$, and let $\sh{F}$ be a (coherent) sheaf on $U$.
  Then $\sh{F}$ is generated by its sections (over $U$).
\end{lemma}

\begin{proof}
  Let $x\in U$, and let $f$ be a regular function on $V$, zero on $V\setminus U$ and non-zero at $x$.
  We have $V_f\subset U\subset V$.
  Since $V_f$ is affine, we know \cite{12} that $\sh{F}_x$ is generated by its sections over $V_f$, and it thus suffices to prove that these sections can be extended to $U$, after multiplying by a suitable power of $f$.
  This follows from the more general following lemma:
\end{proof}

\begin{lemma}{2}
\label{lemma2}
  Let $X$ be a variety, $f$ a regular function on $X$, $\sh{F}$ a sheaf on $X$, and $s$ a section of $\sh{F}$ over $U=X_f$.
  Then there exists an integer $n>0$ such that $f^ns$ can be extended to a section of $\sh{F}$ over $X$.
\end{lemma}

\begin{proof}
  We can cover $X$ by finitely many affine opens $X_i$.
  By applying \cite[lemma~1, p.~247]{12} (or by arguing directly, as in \cref{proposition1}), we see that there exists an integer $n$ and sections $s_i$ of $\sh{F}$ over the $X_i$ that extend $f^ns$ over $X_i\cap U$.
  Since the $s_i-s_j$ are zero on $X_i\cap X_j\cap U$, there exists an integer $m$ such that $f^m(s_i-s_j)=0$ on $X_i\cap X_j$ (\cite[p.~235]{12}, or arguing directly), and $m$ can be chosen independent of the pair $(i,j)$.
  The $f^ms_i$ then define a section $s'$ of $\sh{F}$ over $X$ that indeed extend $f^{n+m}s$.
\end{proof}

\begin{proposition}{2}
\label{proposition2}
\oldpage{99}
  If $U$ is an open subset of a variety $V$, then every sheaf $\sh{F}$ over $U$ can be extended to $V$.
\end{proposition}

\begin{proof}
  We show that, if $U\neq V$, we can extend $\sh{F}$ to an open subset $U'\supset U$, with $U'\neq U$;
  from that fact that every (strictly) increasing chain of open subsets stabilises, this will imply the proposition.
  Let $x\in V\setminus U$, and let $W$ be an affine open that contains $x$;
  let $U'=W\cup U$.
  We are thus led to extending the sheaf $\sh{F}|W\cap U$ to $W$, or, in other words, we can restrict to proving the proposition in the specific case where $V$ is affine.
  In this case, \cref{lemma1} shows that $\sh{F}$ is generated by its sections, i.e. it is of the form $\sh{L}/\sh{R}$, where $\sh{L}$ is the direct sum of the sheaves $\sh{O}_U$.
  The sheaf $\sh{L}$ can be extended in the obvious way to $V$, and, by \cref{proposition1}, there exists a subsheaf $\sh{R}'$ of $\sh{L}$ on $V$ whose restriction to $U$ is $\sh{R}$.
  The sheaf $\sh{L}/\sh{R}'$ is then the desired extension.
\end{proof}

\begin{remark}
  \Cref{proposition1,proposition2} correspond to the geometric fact that the closure of any algebraic subvariety of $U$ is an algebraic subvariety of $V$.
  These propositions do not extend \emph{as is} to the ``analytic'' case.
  The most we can hope for (by results of Rothstein) is that they still hold true if we make certain restrictions on the dimensions of $V\setminus U$ and the varieties appearing in the local primary decomposition of the sheaf $\sh{F}$.
\end{remark}


\section{Proper maps of quasi-projective varieties}
\label{section2}

A variety $X$ is said to be \emph{quasi-projective} is it is isomorphic to a locally closed subvariety of a projective space.
It is said to be \emph{projective} if it is isomorphic to a closed subvariety of a projective space.
\emph{From here on in, all the varieties considered are assumed to be quasi-projective.}

\begin{lemma}{3}
\label{lemma3}
  Let $\PP$ be a projective space, $U$ an arbitrary variety, and $G$ a closed subset of $\PP\times U$.
  Then the projection of $G$ in $U$ is closed.
\end{lemma}

This is a translation into geometric language of the well-known fact that a projective space is a ``complete'' variety, in the sense of Weil.
We briefly recall the principal of the proof:

\begin{proof}
  Since the question is local with respect to $U$, we can assume that $U$ is affine, and even that $U$ is as affine open of the space $k^n$.
  We can also assume that $G$ is irreducible.
  So we choose projective coordinates $x_i$ in $\PP$ such that $G$ meets the set $\PP_0\times U$ of points where $x_0\neq0$.
  If $A$ denotes the coordinate ring of $U$, then the coordinate ring of the affine variety $\PP_0\times U$ is $A[x_i/x_0]=B_0$;
  the set $G$ defines (and is defined by) a prime ideal $\mathfrak{p}$ of $B_0$.
  If $\mathfrak{p}'$ denotes $A\cap\mathfrak{p}$, then the prime ideal $\mathfrak{p}'$ corresponds to the closure of the projection $G'$ of $G$ in $U$.
  A point in this closure is thus a homomorphism $f\colon A\to k$ (where $k$ denotes the base field) that is zero on $\mathfrak{p}'$;
\oldpage{100}
  this point is the image of a point in $G$ that lies in $\PP_0\times U$ if and only if $f$ can be extended to a homomorphism $g\colon B_0\to k$ that is zero on $\mathfrak{p}$.
  So let $L$ be the field of functions of $G$;
  the field $L$ contains $A/\mathfrak{p}'$ as a subring.
  By the theorem of extension of specialisations, there exists a valuation $v$ of $L$, with values in $k$, that extends $f$.
  Let $\Phi$ be the place associated to this valuation.
  If $v(x_i/x_0)\geq0$ for all $i$, then the place $\Phi$ is finite over the $x_i/x_0$, and thus induces, on $B_0/\mathfrak{p}\subset L$, a homomorphism $g$ that extends $f$.
  If $v(x_i/x_0)<0$ for some $i$, then we replace $x_0$ by the $x_i$ that gives the smallest possible value of $v(x_i/x_0)$, and we are then back in the previous case.
\end{proof}

If $f\colon X\to Y$ is a morphism, then we write $G_f$ to denote its graph.
It is trivial that $G_f$ is closed in $X\times Y$.

\begin{lemma}{4}
\label{lemma4}
  Let $f\colon X\to Y$ and $g\colon Y\to Z$ be morphisms, where $X$ and $Y$ are subvarieties of projective spaces $\PP$ and $\PP'$ (respectively).
  Suppose that $G_f$ is closed in $P\times Y$, and that $G_g$ is closed in $\PP'\times Z$.
  Then $G_{gf}$ is closed in $\PP\times Z$.
\end{lemma}

\begin{proof}
  We have $G_f \subset \PP\times Y = \PP\times G_g \subset \PP\times \PP'\times Z$, and since each one is closed in the next, we see that $G_f$ can be identified with a closed subset of $\PP\times \PP'\times Z$.
  Since $G_{gf}$ is exactly the projection of $G_f$ to the factor $\PP\times Z$, the lemma follows from \cref{lemma3}.
\end{proof}

\begin{lemma}{5}
\label{lemma5}
  Let $f\colon X\to Y$ be a morphism, and let $X\subset \PP$ and $X\subset \PP'$ be embeddings of $X$ into projective spaces.
  If $G_f$ is closed in $\PP\times Y$, then it is also closed in $\PP'\times Y$.
\end{lemma}

\begin{proof}
  We apply \cref{lemma4} to the morphisms $X\xrightarrow{i}X\xrightarrow{f}Y$, where $i$ denotes the identity morphism.
  Everything then reduces to showing that the graph $G_i$ of $i$ in $\PP\times X$ is closed, which follows from the fact that it is given by the intersection of $\PP\times X$ with the diagonal of $\PP\times \PP$.
\end{proof}

\Cref{lemma5} justifies the following definition:

\begin{definition}
  A map $f\colon X\to Y$ is said to be \emph{proper} if it is a morphism and if its graph $G_f$ is closed in $\PP\times Y$, where $\PP$ is a projective space containing $X$.
\end{definition}

We can give a definition of proper maps that is analogous to the definition of complete varieties:

\begin{proposition}{3}
\label{proposition3}
  For a morphism $f\colon X\to Y$ to be proper, it is necessary and sufficient, for every variety $Z$, and every closed subset $T$ of $X\times Z$, for the image of $T$ in $Y\times Z$ to be closed.
\end{proposition}

\begin{proof}
  Let $\PP$ be a projective space inside which $X$ can be embedded;
  since $G_f$ is closed in $\PP\times Y$, the product $G_f\times Z$ is closed in $\PP\times Y\times Z$, and so $T$ can be embedded as a closed subset into $\PP\times Y\times Z$.
\oldpage{101}
  Applying \cref{lemma3}, we see that the projection of $T$ to $Y\times Z$ (which is exactly $(f\times1)(T)$) is closed.
  Conversely, suppose that this property holds true, and apply it to $Z=\PP$, with the set $T$ being the diagonal of $X\times X$, embedded into $X\times \PP$.
  The image of $T$ in $Y\times Z=Y\times \PP$ is then exactly $G_f$, which is indeed closed.
\end{proof}

\begin{proposition}{4}
\label{proposition4}
  \begin{enumerate}[(i)]
    \item The identity morphism $i\colon X\to X$ is proper.
    \item The composition of two proper maps is proper.
    \item The direct product of two proper maps is proper.
    \item The image of a closed subset by a proper map is a closed subset.
    \item An injection $Y\to X$ is proper if and only if $Y$ is closed in $X$.
    \item Every morphism from a projective variety is proper.
    \item A projection $Y\times Z\to Y$ is proper if and only if $Z$ is projective (assuming the variety $Y$ to be non-empty).
  \end{enumerate}
\end{proposition}

\begin{proof}
  We indicate, as an example, how to prove \emph{(vii)} (since the other claims are even easier to prove).
  If $Z$ is projective, then we apply the criteria of \cref{proposition3};
  so let $Z'$ be an arbitrary variety, and $T$ a closed subset of $Y\times Z\times Z'$;
  we need to show that the projection of $T$ in $Y\times Z'$ is closed, which follows from \cref{lemma3}.
  Conversely, if $Y\times Z\to Y$ is proper, then the composition $Z\to Y\times Z\to Y$ is proper.
  Since the image of this map is a point, we immediately deduce that $Z$ is projective (by returning to the definition).
\end{proof}

\begin{corollary}{5}
\label{corollary5}
  For a morphism $f\colon X\to Y$ to be proper, it is necessary and sufficient for it to factor as $X\to \PP\times Y\to Y$, where $X\to \PP\times Y$ is an injection into a closed subvariety, and $\PP\times Y\to Y$ is the projection onto the second factor (where $\PP$ denotes some projective space).
\end{corollary}

\begin{proof}
  By the definition of a proper map, this condition is necessary (if we take $\PP$ to be a projective space into which we can embed $X$).
  It is sufficient by \emph{(ii)}, \emph{(v)}, and \emph{(vii)}.
\end{proof}

\begin{proposition}{5}
\label{proposition5}
  Suppose that the base field $k$ is the field of complex numbers.
  For a morphism $f\colon X\to Y$ to be proper (in the above sense), it is necessary and sufficient for it to be proper (in the topological sense) when we endow $X$ and $Y$ with the ``usual'' topology.
\end{proposition}

\begin{proof}
  Suppose that $f$ is proper in the algebraic sense, and let $K$ be a compact subset of $Y$ (for the usual topology).
  Suppose that $X$ is embedded into some projective space $\PP$.
  Since $\PP$ is compact, we know that $f^{-1}(K)=G_f\cap(\PP\times K)$ is compact, which shows that $f$ is proper in the topological sense.
  Conversely, assume that this condition is satisfied, and aim to prove that the condition of \cref{proposition3} is satisfied:
\oldpage{102}
  the image of $T$ in $Y\times Z$ is closed for the usual topology, and thus also for the Zariski topology \cite[proposition~7, p.~12]{13}.
\end{proof}

\begin{remark}
  The notion of a proper map can be extended to ``abstract'' (that is, non-quasi-projective) varieties:
  it suffices to take the criteria of \cref{proposition3} as a definition \cite{4}.
  \Cref{proposition4,proposition5} still hold true (if we replace ``projective'' with ``complete'' in item~\emph{(vii)} of \cref{proposition4}).
  The proofs are essentially the same:
  instead of using embeddings into projective spaces, we use the fact that every variety is the image of a quasi-projective variety under a proper map (Chow's lemma \cite{4,14}).
\end{remark}


\section{Image of a sheaf under a proper map}
\label{section3}

Let $f\colon X\to Y$ be a morphism, with $Y$ a variety, and let $\sh{F}$ be a (coherent algebraic, as always) sheaf on $X$.
We define, by the classical procedure of Leray, sheaves $\RR^qf(\sh{F})$ on $Y$ by setting
\[
  \RR^qf(\sh{F})_U = \HH^q(f^{-1}(U),\sh{F})
  \quad\mbox{for every open subset $U$ of $Y$.}
\]
For $q=0$, we have the sheaf associated to the presheaf given by the $\HH^0(f^{-1}(U),\sh{F})$;
this is the \emph{direct image} of the sheaf $\sh{F}$.
We can show \cite{7} that the $\RR^qf$ are the \emph{derived functors} of the functor $\sh{F}\to\RR^0f(\sh{F})$ (where $\sh{F}$ runs over the category of all sheaves on $X$, coherent or not).

\begin{examples}
  \begin{enumerate}
    \item If $X\to Y$ is an injection of a closed subvariety, then the sheaf $\RR^0f(\sh{F})$ is exactly the sheaf $\sh{F}$ extended by $0$ outside of $X$, and the sheaves $\RR^qf(\sh{F})$, for $q\geq1$, are zero (let $U$ be an affine open; then $f^{-1}(U)=U\cap X$ is affine, whence $\RR^qf(\sh{F})=0$).
      \label{example1}
    \item Let $Y$ be a point.
      A sheaf on a point is simply a group (or a $k$-vector space, if we are talking about algebraic sheaves).
      The $\RR^qf(\sh{F})$ are then simply the cohomology groups $\HH^q(X,\sh{F})$;
      we note that these are not necessarily vector spaces \emph{of finite dimension} (or, in other words, not necessarily \emph{coherent} sheaves on $Y$).
        \label{example2}
    \item Suppose that $f\colon X\to Y$ defines a birational isomorphism between the varieties $X$ and $Y$ (assumed to be projective and non-singular).
      Take $\sh{F}$ to be the sheaf $\sh{O}_X$ of local rings of $X$;
      we immediately see that $\RR^0f(\sh{O}_X)=\sh{O}_Y$.
      Is it true that $\RR^qf(\sh{O}_X)=0$ for $q\geq1$?
      We can at least verify this for ``blow-ups'', and it would be interesting to know the answer in the general case.
      \label{example3}
  \end{enumerate}
\end{examples}

We note that Leray's theory can be translated without any changes (see \cite{7});
there is a spectral sequence abutting $\HH^\bullet(X,\sh{F})$, and with $E_2^{p,q}=\HH^p(Y,\RR^qf(\sh{F}))$.
If we apply, for example, this spectral sequence
\oldpage{103}
to \hyperref[example3]{Example~3} above, then we see that $\RR^qf(\sh{O}_X)=0$ for $q\geq1$ implies that $\HH^\bullet(X,\sh{O}_X)=\HH^\bullet(Y,\sh{O}_Y)$.

We have seen (\hyperref[example2]{Example~2}) that the $\RR^qf(\sh{F})$ are not, in general, coherent sheaves on $Y$.
However:

\begin{theorem}{1}
\label{theorem1}
  If $f\colon X\to Y$ is proper, then the $\RR^qf(\sh{F})$, for $q>0$, are coherent sheaves on $Y$, for any coherent sheaf $\sh{F}$ on $X$.
\end{theorem}

Let $\PP$ be a projective space into which we can embed $X$, and let $G_f$ be the graph of $f$ in $\PP\times Y$.
By definition of what it means to be a proper map, $G_f$ is closed in $\PP\times Y$.
Let $\sh{F}'$ be the sheaf on $\PP\times Y$ obtained by extending $\sh{F}$ by $0$ outside of $G_f=X$ (see \hyperref[example2]{Example~2});
if $\pi$ denotes the projection from $\PP\times Y$ to $Y$, then we immediately see that $\RR^q\pi(\sh{F}')=\RR^qf(\sh{F})$.
\emph{We can thus restrict to proving \cref{theorem1} for $\pi\colon \PP\times Y\to Y$}.
Furthermore, since the question is local with respect to $Y$, we can assume that $Y$ is an affine variety.

On $\PP$ we have a ``standard'' bundle $L$ of dimension~$1$, whose sections are the linear forms (see \cite[chap.~III, §~2]{12});
this fibre defines a fibre on $X=\PP\times Y$ that we denote also by $L$.
The sheaf associated to $L^n$ on $\PP\times Y$ will be denoted $\sh{O}_X(n)$.
We then have:

\begin{lemma}{6}
\label{lemma6}
  Every coherent algebraic sheaf $\sh{F}$ on $X=\PP\times Y$, with $Y$ affine, is isomorphic to a quotient of some direct sum of sheaves of the form $\sh{O}_X(n)$.
\end{lemma}

\begin{proof}
  When $Y$ consists of a single point, this is théorème~1 of FAC, p.~247.  
  We are going to reduce to this particular case: let $Y\subset\PP'$ be an embedding of $Y$ into a projective space, and let $\overline{Y}$ be the closure of $Y$.
  By \cref{proposition2}, the sheaf $\sh{F}$ can be extended to a sheaf $\overline{\sh{F}}$ on $\PP\times\overline{Y}$.
  The embedding of $\overline{Y}$ into $\PP'$ defines, on $\overline{Y}$ (and thus also on $\PP\times\overline{Y}$), a bundle $L'$ of dimension $1$.
  The product bundle $LL'$ corresponds to the well known embedding of $\PP\times\PP'$ into a projective space $P''$ (the ``Segre'' embedding, given by the products $x_i y_j$ of the homogeneous coordinates of the two projective spaces).
  Then applying the result from \cite{12} cited above to $\overline{\sh{F}}$ and $\PP''$, we see that $\overline{\sh{F}}$ is a quotient of a direct sum of sheaves of the form $\sh{O}_{\PP\times\overline{Y}}(L^nL'^n)$;
  by restricting to $\PP\times Y$, and taking into account the fact that $\PP'$ is \emph{trivial} on $Y$, we indeed obtain the desired result.
\end{proof}

  [Of course, we could also give a direct proof, copied from the one in \cite{12}.]

\begin{lemma}{7}
\label{lemma7}
  The $\RR^q\pi(\sh{O}_X(n))$ are coherent sheaves on $Y$.
\end{lemma}

\begin{proof}
  We explicitly calculate the sheaves $\RR^q\pi(\sh{O}_X(n))$.
  If $U$ is an affine open
  \oldpage{104}
  subset of $Y$, then
  \[
    \RR^q\pi(\sh{O}_X(n))_U = \HH^q(\PP\times U,\sh{O}_X(n)).
  \]

  If $\mathfrak{U}=\{U_i\}$ is an affine cover of $\PP$, then the $U_i\times U$ form an affine cover $\mathfrak{U}'$ of $\PP\times U$;
  taking into account the fact that $\sh{O}_X(n)$ ``comes from'' $\PP$, we see that the complex $C(\mathfrak{U}',\sh{O}_X(n))$ is isomorphic to the tensor product $C(\mathfrak{U}',\sh{O}_{\PP}(n))\otimes_k\HH^0(U,\sh{O}_U)$.
  The universal coefficient formula then shows that
  \[
    \HH^q(\PP\times U,\sh{O}_X(n)) = \HH^q(\PP,\sh{O}_{\PP}(n))\otimes_k\HH^0(U,\sh{O}_U).
  \]

  This latter equality implies that $\RR^q\pi(\sh{O}_X(n))$ is isomorphic to the sheaf $\sh{O}_Y\otimes_k V^q$, where $V^q=\HH^q(\PP,\sh{O}_P(n))$.
  Since $V^q$ is a vector space of finite dimension over $k$, it is indeed a coherent sheaf on $Y$, which proves the lemma.
\end{proof}

  [The above proof applies more general to any projection $\pi\colon Y\times Z\to Y$ with $Z$ projective, whenever the sheaf $\sh{F}$ is of the form $\sh{G}\otimes\sh{H}$, with $\sh{G}$ coherent on $Y$ and $\sh{H}$ coherent on $Z$.
  We then find that $\RR^q\pi(\sh{F})=\sh{G}\otimes\HH^q(Z,\sh{H})$.
  We could consider, even more generally, the case of a product map $Y\times Z\to Y'\times Z'$ ...]

\begin{proof}[Proof of \cref{theorem1}]
  We can now prove \cref{theorem1} for an arbitrary sheaf $\sh{F}$ on $X=\PP\times Y$.
  We argue by decreasing induction on the integer $q$.
  If $q>\dim X$, then it is clear that $\RR^q\pi(\sh{F})=0$.
  So suppose that the theorem is proven for $q+1$.
  By \cref{lemma6}, there exists an exact sequence $0\to\sh{R}\to\sh{L}\to\sh{F}\to0$, where $\sh{L}$ is isomorphic to a direct sum of sheaves of the form $\sh{O}_X(n)$.
  The exact sequence of cohomology (or the exact sequence of derived functors) shows that we have an exact sequence
  \[
    \RR^q\pi(\sh{R}) \to
    \RR^q\pi(\sh{L}) \to
    \RR^q\pi(\sh{F}) \to
    \RR^{q+1}\pi(\sh{R}) \to
    \RR^{q+1}\pi(\sh{L}).
  \]

  Given the induction hypothesis and \cref{lemma7}, the sheaves $\RR^q\pi(\sh{L})$, $\RR^{q+1}\pi(\sh{R})$, and $\RR^{q+1}\pi(\sh{L})$ are coherent.
  It follows that $\RR^q\pi(\sh{F})$ admits a subsheaf of finite type, with the quotient being coherent.
  An immediate argument then shows that $\RR^q\pi(\sh{F})$ is of finite type.
  This result, having been proven for any coherent sheaf, also holds for $\sh{R}$.
  The image of $\RR^q\pi(\sh{R})$ in $\RR^q\pi(\sh{L})$ is then a coherent sheaf (\cite[p.~208]{12}), and $\RR^q\pi(\sh{F})$ is an extension of two coherent sheaves, and is thus coherent (\emph{id.}).
\end{proof}

\begin{remarks}
  \begin{enumerate}
    \item \Cref{theorem1} holds true even if we don't suppose that $X$ is quasi-projective (we can restrict to this case by using Chow's lemma and the ``devissage'' of coherent sheaves, see \cite{6}).
    \item Grauert and Remmert have proven the analytic analogue of \cref{theorem1} for the projection $\pi\colon\PP\times Y\to Y$.
      Needless to say that the proof is more difficult!
  \end{enumerate}
\end{remarks}


\section{The group $K(X)$ of classes of sheaves on a variety $X$}
\label{section4}

\oldpage{105}
Let $X$ be an algebraic variety, and let $F(X)$ be the free abelian group generated by the set $\cat{C}$ of (coherent algebraic, as always) sheaves on $X$.
An element of $F(X)$ is thus a formal linear combination
\[
  x = \sum n_i\sh{F}_i
  \quad
  \mbox{where $n_i\in\mathbb{Z}$ and $\sh{F}_i\in\cat{C}$.}
\]
We agree, of course, to identify isomorphic sheaves [if not, then $F(X)$ would not even be a ``set''!].

Let
\[
  (E)\quad
  0\to\sh{F}'\to\sh{F}\to\sh{F''}\to0
\]
be an exact sequence of sheaves.
To this exact sequence we associate the element $Q(E)=\sh{F}-\sh{F}'-\sh{F}''$ of $F(X)$.

\begin{definition}
  We define the \emph{group of classes of sheaves on $X$} to be the quotient of $F(X)$ by the subgroup generated by the $Q(E)$, where $E$ runs over all short exact sequences.
\end{definition}

This group will be denoted by $K(X)$ in what follows.
If $\sh{F}$ is a group on $X$, then its canonical image in $K(X)$ will be denoted $\gamma_X(\sh{F})$, or $\gamma(\sh{F})$, or simply $\sh{F}$, depending on the risk of confusion.
The $\gamma(\sh{F})$ generate $K(X)$, and the map $\sh{F}\to\gamma(\sh{F})$ is ``additive'';
in other words, if we have the exact sequence (E), then $\gamma(\sh{F})=\gamma(\sh{F}')+\gamma(\sh{F}'')$.
Conversely, by the very definition of $K(X)$, every additive map from the set of sheaves into an abelian group $G$ can be written in the form $\sh{F}\mapsto\pi(\gamma(\sh{F}))$, where $\pi\colon K(X)\to G$ is a uniquely determined homomorphism.

We can apply the above to construction to many other situations, apart from that of sheaves.
We will need, in particular, to apply it to the case of \emph{vector bundles} on $X$.
So let $\cat{V}$ be the set of vector bundles;
we define $F_1(X)$ to be the free group generated by $\cat{V}$, and $K_1(X)$ to be the quotient of $F_1(X)$ by the subgroup generated by the $Q_1(E)=\sh{F}-\sh{F}'-\sh{F}''$, where (E) now denotes a short exact sequence of vector bundles.
If $X$ is connected (which we will assume), then we know that we can identify vector bundles with locally free sheaves on $X$;
we thus have $\cat{V}\subset\cat{C}$, and the injection $\cat{V}\to\cat{C}$ defines a canonical homomorphism $\varepsilon\colon K_1(X)\to K(X)$.

\begin{theorem}{2}
\label{theorem2}
  Suppose that $X$ is an irreducible non-singular quasi-projective variety.
  Then the homomorphism $\varepsilon\colon K_1(X)\to K(X)$ defined above is a bijection.
\end{theorem}

We will need a certain number of auxiliary results on the relation between $\cat{V}$ and $\cat{C}$:

\begin{lemma}{8}
\label{lemma8}
  Let $0\to\sh{Z}\to\sh{L}'\to\sh{L}\to0$ be an exact sequence such that $\sh{L}',\sh{L}\in\cat{V}$.
  Then $\sh{Z}\in\cat{V}$.
\end{lemma}

\begin{proof}
  \oldpage{106}
  If $P\in X$, then the local module $\sh{L}_P$ is free over $\sh{O}_P$, and so a direct factor of $\sh{L}'_P$, which proves that $\sh{Z}_P$ is a projective $\sh{O}_P$-module, and thus free, since $\sh{O}_P$ is a local ring.
  But a coherent algebraic sheaf such that all its stalks are free is itself locally free (see \cite[p.~242, lines~10--11 from the bottom]{12}).
\end{proof}

\begin{lemma}{9}
\label{lemma9}
  Let $n=\dim X$, and let $0\to\sh{Z}\to\sh{L}_p\to\ldots\to\sh{L}_0\to\sh{F}\to0$ be an exact sequence, with $\sh{L}_i\in\cat{V}$.
  If $p\geq n-1$, then $\sh{Z}\in\cat{V}$.
\end{lemma}

\begin{proof}
  This is again a local question.
  So let $P\in X$;
  the fact that the local ring $\sh{O}_P$ is a \emph{regular} local ring of dimension $n$ means that we can apply the syzygy theorem, and show that $\sh{Z}_P$ is $\sh{O}_P$-free, whence the desired result (noting that the hypothesis that $X$ is non-singular is used in an essential way).
\end{proof}

\begin{lemma}{10}
\label{lemma10}
  Every $\sh{F}\in\cat{C}$ is a quotient of some $\sh{L}\in\cat{V}$.
\end{lemma}

\begin{proof}
  Let $X\subset \PP$ be a projective embedding of $X$, and let $\overline{X}$ be its closure in $P$.
  By \cref{proposition2}, $\sh{F}$ can be extended to a sheaf $\sh{F}'$ on $\overline{X}$.
  By \cite[théorème~1, p.~247]{12} (and see also \cref{lemma6}), the sheaf $\sh{F}'$ is the quotient of a direct sum of sheaves of the form $\sh{O}_{\overline{X}}(n)$, and thus a locally free sheaf on $\overline{X}$.
  By restriction to $X$, we obtain the desired result.
\end{proof}

\begin{corollary}{\!\!}
  For every $\sh{F}\in\cat{C}$, there exists an exact sequence
  \[
    0\to\sh{L}_n\to\sh{L}_{n-1}\to\ldots\to\sh{L}_0\to\sh{F}\to0
  \]
  with $\sh{L}_i\in\cat{V}$.
\end{corollary}

\begin{proof}
  This is a consequence of \cref{lemma9,lemma10}.
  We can state this corollary in a different way by saying that there exists a ``complex'' $\sh{L}$ in $\cat{V}$ that is acyclic in degrees $\geq1$, and such that $\HH_0(\sh{L})=\sh{F}$.
\end{proof}

\begin{proof}[Proof of \cref{theorem2}]
  We now continue to the proof of \cref{theorem2}.
  If $\sh{F}\in\cat{C}$, we take an acyclic complex $\sh{L}$ in $\cat{V}$ such that $\HH_0(\sh{L})=\sh{F}$.
  We set
  \[
    \gamma_1(\sh{L}) = \sum(-1)^p\gamma_1(\sh{L}_p);
  \]
  this is an element of $K_1(X)$;
  we define $\gamma(\sh{L})\in K(X)$ analogously.
  Suppose that the following two lemmas have been proven:
  \begin{lemma}{11}
  \label{lemma11}
    $\gamma_1(\sh{L})$ depends only on $\sh{F}$.
  \end{lemma}
  \begin{lemma}{12}
  \label{lemma12}
    $\gamma_1(\sh{L})$ is an additive function of $\sh{F}$.
  \end{lemma}

  Then, setting $\eta(\sh{F})=\gamma_1(\sh{L})$, we obtain a homomorphism $\eta\colon K(X)\to K_1(X)$.
  By the definition of $\sh{L}$, we have that
  \[
    \eta(\gamma_1(\sh{L})) = \gamma(\sh{L}) = \gamma(\sh{F}),
  \]
  whence $\varepsilon\circ\eta=1$.
  \oldpage{107}
  The proof that $\eta\circ\varepsilon=1$ is even more trivial.
  So everything relies only on proving \cref{lemma11,lemma12}.
\end{proof}

\begin{proof}[Proof of \cref{lemma11}]
  We first start by stating a corollary of \cref{lemma10}:
  \begin{lemma}{13}
  \label{lemma13}
    Let $\sh{A},\sh{B},\sh{C}\in\cat{C}$, and let $u\colon\sh{A}\to\sh{B}$ and $v\colon\sh{C}\to\sh{B}$, with $u$ and $v$ surjective.
    Then there exist $\sh{L}\in\cat{V}$, and $u'\colon\sh{L}\to\sh{C}$ and $v'\colon\sh{L}\to\sh{A}$, such that $v\circ u'=u\circ v'$, with $u'$ and $v'$ surjective:
    \[
      \begin{tikzcd}
        \sh{L} \ar[r,"v'"] \ar[d,swap,"u'"]
        & \sh{A} \ar[d,"u"]
      \\\sh{C} \ar[r,swap,"v"]
        & \sh{B}
      \end{tikzcd}
    \]
  \end{lemma}

  \begin{proof}
    Let $(\sh{A},\sh{C})$ be the subsheaf of $\sh{A}\times\sh{C}$ given by the elements having the same image in $\sh{B}$.
    Since $u$ and $v$ are surjective, the canonical projections $(\sh{A},\sh{C})\to\sh{A}$ and $(\sh{A},\sh{C})\to\sh{C}$ are surjective.
    By applying \cref{lemma10} to $(\sh{A},\sh{C})$, we obtain the desired result.
  \end{proof}

  Now let $\sh{L}$ and $\sh{L}'$ be two resolutions of $\sh{F}$, which we intend to compare.
  We will show that there exists a third resolution $\sh{L}''$ of $\sh{F}$, along with surjective homomorphisms $\sh{L}''\to\sh{L}$ and $\sh{L}''\to\sh{L}'$ that induce the identity on $\HH^0$;
  everything will then rely only on proving that $\gamma_1(\sh{L}'')=\gamma_1(\sh{L})$, for example.
  But, if we denote by $\sh{L}_1$ the kernel of $\sh{L}''\to\sh{L}$, then $\sh{L}_1\in\cat{V}$, by \cref{lemma8}, and the exact sequence of homology shows that $\HH^q(\sh{L}_1)=0$ for all $q\geq0$.
  We thus immediately deduce that $\gamma_1(\sh{L}_1)=0$, and, since $\gamma_1(\sh{L}'')=\gamma_1(\sh{L})+\gamma_1(\sh{L}_1)$, this gives the desired result.
  Everything thus relies only on proving the existence of the resolution $\sh{L}''$, which we do dimension by dimension (where \cref{lemma9} tells us that the procedure will eventually stop), using the following lemma:
  \begin{lemma}{14}
  \label{lemma14}
    Let
    \[
      0\to\sh{Z}\to\sh{L}\to\sh{B}\to0
      \quad\text{and}\quad
      0\to\sh{Z}'\to\sh{L}'\to\sh{B}'\to0
    \]
    be exact sequences (with $\sh{L},\sh{L}'\in\cat{V}$), and let $\sh{B}''\to\sh{B}$ and $\sh{B}''\to\sh{B}'$ be surjective maps.
    We can then extend $\sh{B}''$ to an exact sequence
    \[
      0\to\sh{Z}''\to\sh{L}''\to\sh{B}''\to0
      \quad
      \mbox{with $\sh{L}''\in\cat{V}$}
    \]
    and find surjective homomorphisms
    \[
      \sh{Z}''\to\sh{Z},
      \quad
      \sh{Z}''\to\sh{Z}',
      \quad
      \sh{L}''\to\sh{L},
      \quad
      \sh{L}''\to\sh{L}'
    \]
    such that the following diagram commutes:
    \[
      \begin{tikzcd}
        0 \rar
        & \sh{Z} \rar
        & \sh{L} \rar
        & \sh{B} \rar
        & 0
      \\0 \rar
        & \sh{Z}'' \rar \uar \dar
        & \sh{L}'' \rar \uar \dar
        & \sh{B}'' \rar \uar \dar
        & 0
      \\0 \rar
        & \sh{Z}' \rar
        & \sh{L}' \rar
        & \sh{B}' \rar
        & 0
      \end{tikzcd}
    \]
  \end{lemma}
  \begin{proof}
\oldpage{108}
    By applying \cref{lemma13} to $\sh{L}\to\sh{B}$ and $\sh{B}''\to\sh{B}$, we see that $\sh{L}_1\to\sh{L}$ and $\sh{L}_1\to\sh{B}''$ are surjective.
    By applying the same lemma to $\sh{L}_1\to\sh{B}''\to\sh{B}'$ and $\sh{L}'\to\sh{B}'$, we see that $\sh{L}_2\to\sh{L}_1$ and $\sh{L}_2\to\sh{L}'$ are surjective, making the diagram commute.

    Now let $\sh{L}_3\to\sh{Z}$ and $\sh{L}'_3\to\sh{Z}'$ be surjective, and set
    \[
      \sh{L}'' = \sh{L}_2\oplus\sh{L}_3\oplus\sh{L}'_3.
    \]
    We define $\sh{L}''\to\sh{B}''$ as being $0$ on $\sh{L}_3$ and $\sh{L}'_3$, and equal to $\sh{L}_2\to\sh{L}_1\to\sh{B}$ on $\sh{L}_2$;
    we define $\sh{L}''\to\sh{L}$ as being equal to $\sh{L}_2\to\sh{L}_1\to\sh{L}$ on $\sh{L}_2$, to $0$ on $\sh{L}'_3$, and to $\sh{L}_2\to\sh{Z}\to\sh{L}$ on $\sh{L}_3$;
    we define $\sh{L}''\to\sh{L}'$ analogously.
    We then define $\sh{Z}''$ as the kernel of $\sh{L}''\to\sh{B}''$, and $\sh{Z}''\to\sh{Z}$ and $\sh{Z}''\to\sh{Z}'$ as restrictions of the maps from $\sh{L}''$.
    The commutativity of the diagram is then immediate;
    furthermore, $\sh{Z}''$ evidently contains $\sh{L}_3$, which maps to $\sh{Z}$;
    \emph{a fortiori}, $\sh{Z}''\to\sh{Z}$ is surjective, and so too is $\sh{Z}''\to\sh{Z}'$, which finished the proof of \cref{lemma13}.
  \end{proof}
  This finishes the proof of \cref{lemma11}.
\end{proof}

\begin{proof}[Proof of \cref{lemma12}]
  Let $0\to\sh{F}'\to\sh{F}\to\sh{F}''\to0$ be an exact sequence.
  We are going to show that there exists resolutions $\sh{L}'$, $\sh{L}$, and $\sh{L}''$ of these sheaves, such that we also have an exact sequence $0\to\sh{L}'\to\sh{L}\to\sh{L}''\to0$.
  The additivity of $\gamma_1(\sh{L})$ will then be evident.
  To construct these resolutions, we proceed, again, dimension by dimension.
  Everything relies on proving that, given an exact sequence as above, we can extend it to a commutative diagram
  \[
    \begin{tikzcd}
      0 \rar
      & \sh{F}' \rar
      & \sh{F} \rar
      & \sh{F}'' \rar
      & 0
    \\0 \rar
      & \sh{L}' \rar\uar
      & \sh{L} \rar\uar
      & \sh{L}'' \rar\uar
      & 0
    \end{tikzcd}
  \]
  with $\sh{L}',\sh{L},\sh{L}''\in\cat{V}$, and where the $\sh{L}\to\sh{F}$ are surjective.

  For this, we first take some surjective $\sh{L}''\to\sh{F}''$, and apply \cref{lemma13} to $\sh{F}\to\sh{F}''$ and $\sh{L}\to\sh{F}''$.
  From this we obtain surjective $\sh{L}_1\to\sh{F}$ and $\sh{L}_1\to\sh{L}''$ that make the diagram commute.
  Next, we take some surjective $\sh{L}_2\to\sh{F}'$, and set $\sh{L}=\sh{L}_2\oplus\sh{L}_1$.
  We define $\sh{L}\to\sh{F}$ and $\sh{L}\to\sh{L}''$ in the evident way, and we take $\sh{L}'$ to be the kernel of $\sh{L}\to\sh{L}''$.
  Then $\sh{L}_2\subset\sh{L}'$, which shows that $\sh{L}'$ maps to $\sh{F}'$, and all the desired conditions are satisfied.
\end{proof}

This completes the proof of \cref{theorem2}.

\begin{remark}
  The hypothesis that $X$ is quasi-projective was used only in \cref{lemma10}, to prove that every coherent sheaf on $X$ is the quotient of some locally free sheaf.
  We do not know if this lemma can be extended to ``abstract'' algebraic varieties.
\end{remark}


\section{Operations on $K(X)$}
\label{section5}

\subsection{Ring structure on $K(X)$}
\label{subsection5a}

Let $\sh{F}$ and $\sh{G}$ be (coherent) sheaves on $X$.
The $\Tor_p(\sh{F},\sh{G})$ ($p=0,1,\ldots$) are coherent sheaves on $X$ (of course, the $\Tor$
\oldpage{109}
are taken over the sheaf of local rings of $X$).
Since $X$ is non-singular, $\Tor_p(\sh{F},\sh{G})=0$ when $p>\dim X$, which means that the alternating sum $\chi(\sh{F},\sh{G})=\sum(-1)^p\Tor_p(\sh{F},\sh{G})$ is a well-defined element of $K(X)$.
The exact sequence of $\Tor$ shows that $\chi(\sh{F},\sh{G})$ is bilinear in $\sh{F}$ and $\sh{G}$, and so extends to a bilinear map from $K(X)\times K(X)$ to $K(X)$, that we will denote by $(x,x')\mapsto x\cdot x'$.

\begin{proposition}{6}
\label{proposition6}
  The product defined above is commutative and associative.
\end{proposition}

\begin{proof}
  Commutativity is trivial, since each $\Tor$ is commutative.
  The associativity follows from the ``associativity formula'' of the $\Tor$ (see \cite{3}):
  we define the ``simultaneous $\Tor$'' $\Tor(\sh{F},\sh{G},\sh{H})$, and two spectral sequences abutting to $\Tor(\sh{F},\sh{G},\sh{H})$, and with $E_2$ pages $\Tor(\sh{F},\Tor(\sh{G},\sh{H}))$ and $\Tor(\Tor(\sh{F},\sh{G}),\sh{H})$ (respectively);
  we use the fact that the Euler-Poincaré characteristics are invariant in a spectral sequence.

  We can give a simpler proof by using \cref{theorem2}:
  note that, if $\sh{F}$ and $\sh{G}$ are locally free, then
  \[
    \chi(\sh{F},\sh{G}) = \gamma(\sh{F}\otimes\sh{G})
  \]
  which means that associativity is evident when $\sh{F}$, $\sh{G}$, and $\sh{H}$ are locally free.
  Since $K(X)$ is generated by the $\gamma(\sh{F})$, with $\sh{F}$ locally free, this proves associativity.
\end{proof}

[The above product thus corresponds to the tensor product of vector bundles.]


\subsection{The exterior power operations}
\label{subsection5b}

Let $E$ be a vector bundle.
The exterior powers $\bigwedge^p E$ are vector bundles, and their class in $K(X)=K_1(X)$ will be denoted by $\lambda^p(E)$.
If we have an exact sequence
\[
  0\to E'\to E\to E''\to0
\]
then we define, by a well known procedure (Koszul's thesis!) a filtration of $\bigwedge(E)$ with quotients $\bigwedge(E')\otimes\bigwedge(E'')$.
From this we obtain the following formula:
\[
  \lambda^p(E) = \sum_{r+s=p} \lambda^r(E')\cdot\lambda^s(E'').
\]

This formula can be understood as an additivity formula by introducing the formal series (in $t$)
\[
  \lambda_t(E) = \sum \lambda^p(E)t^p;
\]
this is an element of $K(X)[[t]]$, starting with $1$.
The above formula
\oldpage{110}
implies that
\[
  \lambda_t(E) = \lambda_t(E')\lambda_t(E'').
\]

The map $E\mapsto\lambda_t(E)$ can thus be extended to a homomorphism $x\mapsto\lambda_t$ from $K(X)=K_1(X)$ to the multiplicative group $U$ of formal series
\[
  1+a_1t+\ldots+a_nt^n+\ldots
  \quad
  \mbox{with $a_i\in K(X)$.}
\]
By definition, $\lambda^p(x)$ is the coefficient of $t^p$ in $\lambda_t(x)$.

In characteristic $0$, Grothendieck has shown that, for every sheaf $\sh{F}$, $\lambda^p(\sh{F})$ is equal to the alternating sum of the ``alternating $\Tor$'' of $p$ copies of $\sh{F}$.
In characteristic $\neq0$, we do not know any analogous formula.


\subsection{The operation $f^!$}
\label{subsection5c}

Let $f\colon Y\to X$ be a morphism.
If $E$ is a vector bundle over $X$, then the bundle $f^{-1}(E)$ is a vector bundle over $Y$.
This operation is additive, and thus extends to a homomorphism $f^!\colon K(X)\to K(Y)$.
By arguing on the fibres, we immediately see that \emph{$f^!$ is a ring homomorphism, compatible with the operation $\lambda^p$, and such that $(fg)^!=g^!f^!$}.

If $\sh{F}$ is a coherent sheaf on $X$, then we can directly define $f^!(\sh{F})$ as the alternating sum of the $\Tor_p^{\sh{O}_X}(\sh{O}_Y,\sh{F})$;
indeed, this expression is additive in $\sh{F}$ (by the exact sequence of $\Tor$), and reduces to $\sh{O}_Y\otimes\sh{F}$ when $\sh{F}$ is locally free.


\subsection{The operation $f_!$}
\label{subsection5d}

Again, let $f\colon Y\to X$ be a morphism, that we now assume to be \emph{proper}.
If $\sh{F}$ is coherent on $Y$, then we have seen (\cref{theorem1}, \cref{section3}) that the $\RR^qf(\sh{F})$ ($q=0,1,\ldots$) are coherent sheaves on $X$, and their alternating sum is a well-defined element of $K(X)$.
Since this alternating sum is additive in $\sh{F}$ (by the exact sequence of cohomology), we thus obtain an additive homomorphism $f_!\colon K(Y)\to K(X)$.
In the particular case where $Y$ is a closed subvariety of $X$, and $f$ is the canonical injection $Y\to X$, this operation reduces to extending by $0$ outside of $Y$.

The map $f_!$ is not compatible with multiplication.
We do, however, have the following formula:
\[
  f_!(y\cdot f^!(x)) = f_!(y)\cdot x
  \quad
  \mbox{for $x\in K(X)$ and $y\in K(Y)$.}
\]

It suffices to prove this formula when $y=\gamma_Y(\sh{F})$ and $x=\gamma_X(\sh{L})$, where $\sh{F}$ (resp. $\sh{L}$) is a coherent sheaf on $Y$ (resp. a locally free sheaf on $X$).
In this case, we even have the more precise formula
\[
\label{equation*}
  \RR^qf(\sh{F}\otimes_{\sh{O}_Y}f^{-1}(\sh{L})) = \RR^qf(\sh{F})\otimes_{\sh{O}_X}\sh{L}
  \tag{$\star$}
\]
where we set
\[
  f^{-1}(\sh{L}) = \sh{L}\otimes_{\sh{O}_X}\sh{O}_Y.
\]

\oldpage{111}
To prove \cref{equation*}, we first note that
\[
  \sh{F}\otimes_{\sh{O}_Y}f^{-1}(\sh{L}) = \sh{F}\otimes_{\sh{O}_X}\sh{L}.
\]
Returning to the definition of $\RR^qf$, we define a canonical homomorphism from the right-hand side of \cref{equation*} to the left-hand side;
to show that it is an isomorphism we can argue locally.
We can thus restrict to the case where $\sh{L}=\sh{O}_X$, and in this case our claim is trivial.

If we have to proper maps $Z\xrightarrow{g}Y\xrightarrow{f}X$, and if $\sh{F}$ is a coherent sheaf on $Z$, then we can construct a spectral sequence with second page
\[
  E_2^{p,q} = \RR^pf(\RR^qg(\sh{F}))
\]
that abuts to $\RR^n(fg)(\sh{F})$:
this is a particular case of the spectral sequence of composition of functors (see \cite{7}).
From this, we obtain the formula
\[
  (fg)_!(\sh{F}) = f_!(g_!(\sh{F}))
\]
whence finally the fact that $(fg)_!=f_!g_!$.


\section{Chern classes}
\label{section6}

The fact that $K_1(X)=K(X)$ allows us to extend the definition of Chern classes to arbitrary coherent sheaves.

First consider the case where the base field is $\mathbb{C}$;
every vector bundle $E$ over $X$ defines Chern classes $c_i(E)\in\HH^{2i}(X,\mathbb{Z})$.
If we have an exact sequence
\[
  0\to E'\to E\to E''\to0
\]
then we know that $c_p(E)=\sum_{r+s=p}c_r(E')\cdot c_s(E'')$.

As in \cref{subsection5b}, this can be understood as a multiplicative property of the Chern polynomial $c_t(E)=\sum c_p(E)t^p$, and allows us to define $c_t(x)$ for all $x\in K(X)$.
The $c_p(x)$ are homogeneous elements of degree $2p$ of $\HH^\bullet(X,\mathbb{Z})$.

In the case of an arbitrary base field, Grothendieck proceeded in the same way, but replacing $\HH^\bullet(X)$ with the graded ring $A(X)$ of \emph{cycle classes} on $X$, under linear equivalent (à la Chow).
We recall only that a cycle $Z$ on $X$< of codimension $p$ [i.e. a degree-$p$ element of $A(X)$] is said to be linearly equivalent to zero if there exists a cycle $H$ on $X\times\mathbf{D}$ (where $\mathbf{D}$ denotes the projective line) such that $Z=H_a-H_b$ for points $a,b\in\mathbf{D}$;
we denote by $H_a$ the projection to $X$ of $H\cdot(X\times\{a\})$ if it is proper.
Chow and Samuel\footnote{See Chow~\cite{5} and Samuel~\cite{10}. See also the \emph{Séminaire Chevalley}~\cite{11}.} have shown that
\oldpage{112}
this equivalence relation possesses all sorts of reasonable properties, and Chow has shown (unpublished) that we could also construct a theory of Chern classes for vector bundles\footnote{See the paper by Grothendieck which follows this present work \cite{8}.}, with these classes being elements of $A(X)$.
Note that, even in the classical case, this definition is somehow \emph{finer} than the cohomological definition (since two cycles can indeed be homologous without being linearly equivalent).

In what follows, we denote by $A(X)$ either this ring of cycle classes or $\HH^\bullet(X)$, and leave it to the reader to choose between the two theories.

Note that, in all cases, if $f\colon Y\to X$ is a morphism (resp. a proper morphism), then we can associate to it a homomorphism $f^*\colon A(X)\to A(Y)$ (resp. a homomorphism $f_*\colon A(Y)\to A(X)$).
The formula
\[
  f_*(y\cdot f^*(x)) = f_*(y)\cdot x
\]
holds true.

All of the usual formal constructions explained in the work by Hirzebruch \cite{9} can be applied to the Chern classes $c_p(x)$ of an element $x\in K(X)$.
We can, for example, define the \emph{Todd class $T(x)\in A(X)\otimes\mathbb{Q}$} of an element $x$:
we write $c_t(x)$ formally in the form $\prod(1+a_it)$, and let $T(x)=\prod a_i/(1-e^{-a_i})$.
Then
\[
  T(x+y) = T(x)\cdot T(y).
\]

Similarly, we can define the ``exponential'' Chern class, denoted by $\ch(x)$ (which is also an element of $A(X)\otimes\mathbb{Q}$), by
\[
  \ch(x) = \rank(x)+\sum(e^{a_i}-1)
\]
where $\rank(x)$ is the \emph{rank} of $x$ [it is the unique homomorphism from $K(X)$ to $\mathbb{Z}$ that sends a vector bundle to its dimension].
Then
\begin{align*}
  \ch(x+y) &= \ch(x)+\ch(y)
\\\ch(xy) &= \ch(x)\cdot\ch(y)
\end{align*}
by the analogous properties of vector bundles.
With this, we can calculate $\ch(x)$ in terms of $c_t(x)$ and $\rank(x)$ by ``universal'' formulas.
If $f\colon Y\to X$ is a morphism, then
\begin{align*}
  c_p(f^!(x)) &= f^*(c_p(x))
\\\ch(f^!(x)) = f^*(\ch(x))
\end{align*}
for $x\in K(X)$.

These formulas are well known when $x=\gamma(E)$, with $E$ a vector bundle on $X$, and the general case follows from linearity, by applying \cref{theorem2}.


\section{Statement of the Riemann-Roch theorem. First simplifications}
\label{section7}

\oldpage{113}
Let $f\colon Y\to X$ be a proper morphism, with $X$ and $Y$ irreducible non-singular quasi-projective varieties.
We denote by $T(X)$ the Todd class of the tangent bundle of $X$;
it is an element of $A(X)\otimes\mathbb{Q}$.
Now let $y\in K(Y)$.
Then:

\begin{theorem}{(Riemann-Roch)}
\label{theoremriemannroch}
  $f_*(\ch(y)\cdot T(Y)) = \ch(f_!(y))\cdot T(X)$.
\end{theorem}

[The two sides of the equality are thought of as elements of $A(X)\otimes\mathbb{Q}$;
with this in mind, we can say that R-R is a formula ``module torsion'';
Grothendieck has more precise formulas, without torsion --- i.e. that hold in $A(X)$ --- but he does not yet know how to prove them except in characteristic zero.]

We now show how the \hyperref[theoremriemannroch]{Riemann-Roch theorem}, in Grothendieck's form, \emph{implies the Riemann-Roch formula of Hirzebruch \cite{9}}:

We apply \hyperref[theoremriemannroch]{R-R} to a projective $Y$, with $X$ consisting of a single point, and $y$ the class of a coherent sheaf $\sh{F}$ on $Y$.
Since $A(X)$ is simply $\mathbb{Z}$ in dimension $0$, and is zero in higher dimensions, $f_*(u)$, for $u\in A(Y)$, is simply the terms $x_n(u)$ of degree $n=\dim Y$ in $u$.
Also, $T(X)=1$, and $f_!(y)$ is the alternating sum of the sheaves $\RR^qf(\sh{F})$;
a sheaf on a point is simply a vector space of finite dimension;
in particular, $\RR^qf(\sh{F})$ is the vector space $\HH^q(Y,\sh{F})$.
On $X$, the map $\sh{G}\mapsto\ch(\sh{G})$ consists simply of taking the \emph{rank} of a sheaf;
the right-hand side of \hyperref[theoremriemannroch]{R-R} thus becomes $\sum(-1)^p\dim\HH^p(Y,\sh{F}) = \chi(Y,\sh{F})$, and \hyperref[theoremriemannroch]{R-R} then reduces to the form given by Hirzebruch:
\[
  x_n(\ch(\sh{F})\cdot T(Y)) = \chi(Y,\sh{F}).
\]

We note that this formula is proven for any coherent sheaf, and not simply for vector bundles;
this generality is somewhat illusory, by the linear character of \hyperref[theoremriemannroch]{R-R} and by \cref{theorem2}.

The proof of \hyperref[theoremriemannroch]{R-R} proceeds by reduction to particular cases of a projection and an injection.
For this, we use the following lemma:

\begin{lemma}{15}
\label{lemma15}
  Let $Z\xrightarrow{g}Y\xrightarrow{f}X$ be proper morphisms.
  Let $z\in K(Z)$ and let $y=g^!(z)$. Then:
  \begin{enumerate}[(a)]
    \item If \hyperref[theoremriemannroch]{R-R} is true for $\{g,z\}$ and $\{f,y\}$, then it is also true for $\{fg,z\}$.
    \item If \hyperref[theoremriemannroch]{R-R} is true of $\{fg,z\}$ and $\{f,y\}$, and if $f_*$ is injective, then \hyperref[theoremriemannroch]{R-R} is also true for $\{g,z\}$.
  \end{enumerate}
\end{lemma}

\begin{proof}[Proof of (a)]
  By \hyperref[theoremriemannroch]{R-R} for $g$, we have
  \[
    g_*(\ch(z)\cdot T(Z)) = \ch(y)\cdot T(Y).
  \]
  \oldpage{114}
  Applying $f_*$ to both sides, and taking into account the fact that $(fg)_*=f_*g_*$, we see that
  \[
    (fg)_*(\ch(z)\cdot T(Z)) = f_*(\ch(y)\cdot T(Y)).
  \]

  Applying \hyperref[theoremriemannroch]{R-R} for $\{f,y\}$, we see that the right-hand side of the above is equal to $\ch((fg)_!(z)\cdot T(X))$, which indeed proves \hyperref[theoremriemannroch]{R-R} for $\{fg,z\}$.
\end{proof}

\begin{proof}[Proof of (b)]
  Let
  \begin{align*}
    u &= g_*(\ch(z)\cdot T(Z))
  \\v &= \ch(y)\cdot T(Y).
  \end{align*}
  We wish to show that $u=v$.
  Given the hypothesis on $f_*$, it suffices to prove that $f_*(u)=f_*(v)$.
  But \hyperref[theoremriemannroch]{R-R} for $\{fg,z\}$ proves that
  \[
    f_*(u) = \ch(x)\cdot T(X)
    \quad
    \mbox{where $x=(fg)_!(z)=f_!(y)$.}
  \]
  Similarly, \hyperref[theoremriemannroch]{R-R} for $\{f,y\}$ proves that
  \[
    f_*(v) = \ch(x)\cdot T(X).
  \]
\end{proof}

Now let $Y$ and $Y'$ be varieties, and consider their product $Y\times Y'$.
The projections $Y\times Y'\to Y$ and $Y\times Y'\to Y'$ define homomorphisms $K(Y)\to K(Y\times Y')$ and $K(Y')\to K(Y\times Y')$, whence a homomorphism $K(Y)\otimes K(Y')\to K(Y\times Y')$.
By an abuse of language, we also denote by $y\otimes y'$ the image in $K(Y\times Y')$ of the tensor product of two elements $y\in K(Y)$ and $y'\in K(Y')$.

\begin{lemma}{16}
\label{lemma16}
  Let $f\colon Y\to X$ and $f'\colon Y'\to X'$ be proper morphisms, and let $y\in K(Y)$ and $y'\in K(Y')$.
  If \hyperref[theoremriemannroch]{R-R} is true for $\{f,y\}$ and $\{f',y'\}$, then it is true for $\{f\times f',y\otimes y'\}$.
\end{lemma}

(We denote by $f\times f'\colon Y\times Y'\to X\times X'$ the product of $f$ and $f'$.)

\begin{proof}
  The proof consists of a calculation analogous to that for \cref{lemma15};
  we have to use the following formulas:
  \begin{enumerate}[(i)]
    \item $(f\times f')_!(y\otimes y') = f_!(y)\otimes f'_!(y')$;
  \end{enumerate}
\end{proof}


%% Bibliography %%

\nocite{*}
\bibliographystyle{acm}

\begin{thebibliography}{10}

  \bibitem{1}
  {\sc Atiyah, M.}
  \newblock Vector bundles over an elliptic curve.
  \newblock {\em Proc. London math. Soc. 7\/} (1957), 414--452.

  \bibitem{2}
  {\sc Borel, A., and Hirzebruch, F.}
  \newblock Characteristic classes and homogeneous spaces, {II}.
  \newblock {\em Amer. J. Math.\/} (to appear).

  \bibitem{3}
  {\sc Cartan, H., and Eilenberg, S.}
  \newblock {\em Homological algebra}, vol.~19 of {\em Princeton Math. Series}.
  \newblock Princeton University Press, 1956.

  \bibitem{4}
  {\sc Chevalley, C.}
  \newblock La notion de correspondance propre en géométrie algébrique.
  \newblock In {\em Séminaire Bourbaki}, vol.~10.
  \newblock (Talk number 152).

  \bibitem{5}
  {\sc Chow, W.}
  \newblock On equivalence classes of cycles in an algebraic variety.
  \newblock {\em Ann. Math. 64\/} (1956), 450--479.

  \bibitem{6}
  {\sc Grothendieck, A.}
  \newblock Sur les faisceaux algébriques et les faisceaux analytiques
    cohérents.
  \newblock In {\em Séminaire H. Cartan}, vol.~9.
  \newblock (Talk number 2).

  \bibitem{7}
  {\sc Grothendieck, A.}
  \newblock Sur quelques points d'algèbre homologique.
  \newblock {\em Tohoku math. J. 9\/} (1957), 119--221.

  \bibitem{8}
  {\sc Grothendieck, A.}
  \newblock {\em Bull. Soc. math. France 80\/} (1958), 137--154.

  \bibitem{9}
  {\sc Hirzebruch, F.}
  \newblock {\em Neue topologische Methoden in der algebraischen Geometrie}.
  \newblock Ergebnisse der Mathematik. Berlin, Springer, 1956.
  \newblock neue Folge, Heft~9.

  \bibitem{10}
  {\sc Samuel, P.}
  \newblock Rational equivalence of arbitrary cycles.
  \newblock {\em Amer. J. Math. 78\/} (1956), 383--400.

  \bibitem{11}
  {\sc Chevalley, S.~C.}
  \newblock {\em Anneaux de Chow et applications}, vol.~2.
  \newblock 1958.

  \bibitem{12}
  {\sc Serre, J.-P.}
  \newblock Faisceaux algébriques cohérents.
  \newblock {\em Ann. Math. 61\/} (1955), 197--279.

  \bibitem{13}
  {\sc Serre, J.-P.}
  \newblock Géométrie algébrique et géométrie analytique.
  \newblock {\em Ann. Inst. Fourier, Grenoble 6\/} (1955--1956), 1--42.

  \bibitem{14}
  {\sc Serre, J.-P.}
  \newblock Sur la cohomologie des variétés algébriques.
  \newblock {\em J. Math. pures et appl. 36\/} (1957), 1--16.

\end{thebibliography}


\end{document}
