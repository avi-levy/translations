\documentclass{article}

\usepackage{amssymb,amsmath}
\usepackage{hyperref}
\usepackage[nameinlink]{cleveref}
\usepackage{enumerate}
\usepackage{mathrsfs}
%% Fancy fonts --- feel free to remove! %%
\usepackage{Baskervaldx}
\usepackage{mathpazo}


%% Theorem environments %%

\usepackage{amsthm}

  \theoremstyle{plain}

  \newtheorem{innercustomproposition}{Proposition}
  \crefname{innercustomproposition}{Proposition}{Propositions}
  \newenvironment{proposition}[1]
    {\renewcommand\theinnercustomproposition{#1}\innercustomproposition}
    {\endinnercustomproposition}

  \newtheorem{innercustomlemma}{Lemma}
  \crefname{innercustomlemma}{Lemma}{Lemmas}
  \newenvironment{lemma}[1]
    {\renewcommand\theinnercustomlemma{#1}\innercustomlemma}
    {\endinnercustomlemma}

  \newtheorem{innercustomcorollary}{Corollary}
  \crefname{innercustomcorollary}{Corollary}{Corollaries}
  \newenvironment{corollary}[1]
    {\renewcommand\theinnercustomcorollary{#1}\innercustomcorollary}
    {\endinnercustomcorollary}


  \theoremstyle{definition}

  \newtheorem*{remark}{Remark}
  \newtheorem*{definition}{Definition}
  \newtheorem*{examples}{Examples}


%% Shortcuts %%

\newcommand{\sh}{\mathscr}
\newcommand{\HH}{\mathrm{H}}
\newcommand{\RR}{\mathrm{R}}
\renewcommand{\geq}{\geqslant}
\renewcommand{\leq}{\leqslant}

\newcommand{\todo}{\textbf{ !TODO! }}
\newcommand{\oldpage}[1]{\marginpar{\footnotesize$\Big\vert$ \textit{p.~#1}}}


%% Document %%

\usepackage{embedall}
\begin{document}

\renewcommand{\abstractname}{Translator's note.}

\title{The Riemann-Roch theorem}
\author{Armand BOREL and Jean-Pierre SERRE\\(Following some unpublished results by A.~Grothendieck)}
\date{}
\maketitle

\begin{abstract}
  \renewcommand*{\thefootnote}{\fnsymbol{footnote}}
  \emph{This text is one of a series\footnote{\url{https://github.com/thosgood/translations}} of translations of various papers into English.}
  \emph{What follows is a translation (last updated \today) of the French paper:}

  \medskip\noindent
  \textsc{Borel, Armand}; \textsc{Serre, Jean-Pierre}. Le théorème de Riemann-Roch. \emph{Bulletin de la Société Mathématique de France}, Volume~\textbf{86} (1958) , pp.~97-136. \textsc{doi}: \href{https://www.doi.org/10.24033/bsmf.1500}{10.24033/bsmf.1500}.
\end{abstract}

\tableofcontents


%% Content %%

\section*{Introduction}

\oldpage{97}
What follows constitutes the notes from a seminar that took place in Princeton in the autumn of 1957 on the works of Grothendieck;
the new results that are included are due to Grothendieck;
our contribution is solely of an editorial nature.

The ``Riemann-Roch theorem'' of which we speak here holds true for (non-singular) algebraic varieties over a field of arbitrary characteristic;
in the classical case, where the base field is $\mathbb{C}$, this theorem encapsulates, as a particular example, the result proven a few years ago by Hirzebruch \cite{9}.

The full statement and proof of the Riemann-Roch theorem can be found in sections 7 to 16, with the last section being devoted to an application.
Sections 1 to 6 contain some preliminaries on coherent algebraic sheaves \cite{12}.
The terminology that we follow is the same as in \cite{12}, up to one difference: to conform with a custom which is becoming more and more widespread, we use the word ``morphism'' instead of ``regular maps''.


\section{Supplementary results concerning sheaves}

(All the varieties considered below are algebraic varieties over an algebraically closed field $k$ of arbitrary characteristic. Unless otherwise mentioned, all the sheaves considered are coherent algebraic sheaves.)

\begin{proposition}{1}
\label{proposition1}
\oldpage{98}
  Let $U$ be an open subset of a variety $V$, and let $\sh{F}$ be a coherent sheaf on $V$ and $\sh{G}$ a coherent subsheaf of $\sh{F}|U$ (the restriction of $\sh{F}$ to $U$).
  Then there exists a coherent sheaf $\sh{G}'\subset\sh{F}$ such that $\sh{G}'|U=\sh{G}$.
\end{proposition}

(In fact, the proof will show that there exists a \emph{largest} such sheaf having this property.)

\begin{proof}
  For every open subset $W\subset V$, we define $\sh{G}'_W$ as the set of sections of $\sh{F}$ over $W$ that belong to $\sh{G}$ over $U\cap W$.
  Everything reduces to showing that the sheaf $\sh{G}'$ associated to this presheaf is coherent.
  Since this is a local questions, we can suppose that $V$ is an affine variety.
  Let $A$ be its coordinate ring.
  There exist elements $f_i\in A$ such that $U=\bigcup V_{f_i}$, where $V_{f_i}=U_i$ denotes the set of points of $V$ where $f_i\neq0$.
  If, in the definition of $\sh{G}'$, we replace the open subset $U$ by the open subset $U_i$, then we obtain a sheaf $\sh{G}'_i\subset\sh{F}$, and it is clear that $\sh{G}'=\bigcap\sh{G}'_i$.
  By known results on coherent sheaves \cite[p.~209]{12}, it suffices to show that the $\sh{G}'_i$ are coherent.
  We can thus restrict to considering the case where $V$ is affine, and where $U=V_f$, for some $f\in A$.
  In this case, the sheaf $\sh{F}$ is defined by an $A$-module $M$, and the subsheaf $\sh{G}$ of $\sh{F}|U$ is defined by a submodule $N$ of $M_f=M\otimes_A A_f$.
  Let $N'$ be the inverse image of $N$ in $M$ under the canonical map $M\to M_f$.
  The module $N'$ then corresponds to a coherent subsheaf of $\sh{F}$, and we can immediately verify (by taking the $V_{f'}$ to be the $W$, for example) that this sheaf is exactly $\sh{G}'$, which finishes the proof.
\end{proof}

\begin{lemma}{1}
\label{lemma1}
  Let $U$ be an open subset of an affine variety $V$, and let $\sh{F}$ be a (coherent) sheaf on $U$.
  Then $\sh{F}$ is generated by its sections (over $U$).
\end{lemma}

\begin{proof}
  Let $x\in U$, and let $f$ be a regular function on $V$, zero on $V\setminus U$ and non-zero at $x$.
  We have $V_f\subset U\subset V$.
  Since $V_f$ is affine, we know \cite{12} that $\sh{F}_x$ is generated by its sections over $V_f$, and it thus suffices to prove that these sections can be extended to $U$, after multiplying by a suitable power of $f$.
  This follows from the more general following lemma:
\end{proof}

\begin{lemma}{2}
\label{lemma2}
  Let $X$ be a variety, $f$ a regular function on $X$, $\sh{F}$ a sheaf on $X$, and $s$ a section of $\sh{F}$ over $U=X_f$.
  Then there exists an integer $n>0$ such that $f^ns$ can be extended to a section of $\sh{F}$ over $X$.
\end{lemma}

\begin{proof}
  We can cover $X$ by finitely many affine opens $X_i$.
  By applying \cite[lemma~1, p.~247]{12} (or by arguing directly, as in \cref{proposition1}), we see that there exists an integer $n$ and sections $s_i$ of $\sh{F}$ over the $X_i$ that extend $f^ns$ over $X_i\cap U$.
  Since the $s_i-s_j$ are zero on $X_i\cap X_j\cap U$, there exists an integer $m$ such that $f^m(s_i-s_j)=0$ on $X_i\cap X_j$ (\cite[p.~235]{12}, or arguing directly), and $m$ can be chosen independent of the pair $(i,j)$.
  The $f^ms_i$ then define a section $s'$ of $\sh{F}$ over $X$ that indeed extend $f^{n+m}s$.
\end{proof}

\begin{proposition}{2}
\label{proposition2}
\oldpage{99}
  If $U$ is an open subset of a variety $V$, then every sheaf $\sh{F}$ over $U$ can be extended to $V$.
\end{proposition}

\begin{proof}
  We show that, if $U\neq V$, we can extend $\sh{F}$ to an open subset $U'\supset U$, with $U'\neq U$;
  from that fact that every (strictly) increasing chain of open subsets stabilises, this will imply the proposition.
  Let $x\in V\setminus U$, and let $W$ be an affine open that contains $x$;
  let $U'=W\cup U$.
  We are thus led to extending the sheaf $\sh{F}|W\cap U$ to $W$, or, in other words, we can restrict to proving the proposition in the specific case where $V$ is affine.
  In this case, \cref{lemma1} shows that $\sh{F}$ is generated by its sections, i.e. it is of the form $\sh{L}/\sh{R}$, where $\sh{L}$ is the direct sum of the sheaves $\sh{O}_U$.
  The sheaf $\sh{L}$ can be extended in the obvious way to $V$, and, by \cref{proposition1}, there exists a subsheaf $\sh{R}'$ of $\sh{L}$ on $V$ whose restriction to $U$ is $\sh{R}$.
  The sheaf $\sh{L}/\sh{R}'$ is then the desired extension.
\end{proof}

\begin{remark}
  \Cref{proposition1,proposition2} correspond to the geometric fact that the closure of any algebraic subvariety of $U$ is an algebraic subvariety of $V$.
  These propositions do not extend \emph{as is} to the ``analytic'' case.
  The most we can hope for (by results of Rothstein) is that they still hold true if we make certain restrictions on the dimensions of $V\setminus U$ and the varieties appearing in the local primary decomposition of the sheaf $\sh{F}$.
\end{remark}


\section{Proper maps of quasi-projective varieties}

A variety $X$ is said to be \emph{quasi-projective} is it is isomorphic to a locally closed subvariety of a projective space.
It is said to be \emph{projective} if it is isomorphic to a closed subvariety of a projective space.
\emph{From here on in, all the varieties considered are assumed to be quasi-projective.}

\begin{lemma}{3}
\label{lemma3}
  Let $P$ be a projective space, $U$ an arbitrary variety, and $G$ a closed subset of $P\times U$.
  Then the projection of $G$ in $U$ is closed.
\end{lemma}

This is a translation into geometric language of the well-known fact that a projective space is a ``complete'' variety, in the sense of Weil.
We briefly recall the principal of the proof:

\begin{proof}
  Since the question is local with respect to $U$, we can assume that $U$ is affine, and even that $U$ is as affine open of the space $k^n$.
  We can also assume that $G$ is irreducible.
  So we choose projective coordinates $x_i$ in $P$ such that $G$ meets the set $P_0\times U$ of points where $x_0\neq0$.
  If $A$ denotes the coordinate ring of $U$, then the coordinate ring of the affine variety $P_0\times U$ is $A[x_i/x_0]=B_0$;
  the set $G$ defines (and is defined by) a prime ideal $\mathfrak{p}$ of $B_0$.
  If $\mathfrak{p}'$ denotes $A\cap\mathfrak{p}$, then the prime ideal $\mathfrak{p}'$ corresponds to the closure of the projection $G'$ of $G$ in $U$.
  A point in this closure is thus a homomorphism $f\colon A\to k$ (where $k$ denotes the base field) that is zero on $\mathfrak{p}'$;
\oldpage{100}
  this point is the image of a point in $G$ that lies in $P_0\times U$ if and only if $f$ can be extended to a homomorphism $g\colon B_0\to k$ that is zero on $\mathfrak{p}$.
  So let $L$ be the field of functions of $G$;
  the field $L$ contains $A/\mathfrak{p}'$ as a subring.
  By the theorem of extension of specialisations, there exists a valuation $v$ of $L$, with values in $k$, that extends $f$.
  Let $\Phi$ be the place associated to this valuation.
  If $v(x_i/x_0)\geq0$ for all $i$, then the place $\Phi$ is finite over the $x_i/x_0$, and thus induces, on $B_0/\mathfrak{p}\subset L$, a homomorphism $g$ that extends $f$.
  If $v(x_i/x_0)<0$ for some $i$, then we replace $x_0$ by the $x_i$ that gives the smallest possible value of $v(x_i/x_0)$, and we are then back in the previous case.
\end{proof}

If $f\colon X\to Y$ is a morphism, then we write $G_f$ to denote its graph.
It is trivial that $G_f$ is closed in $X\times Y$.

\begin{lemma}{4}
\label{lemma4}
  Let $f\colon X\to Y$ and $g\colon Y\to Z$ be morphisms, where $X$ and $Y$ are subvarieties of projective spaces $P$ and $P'$ (respectively).
  Suppose that $G_f$ is closed in $P\times Y$, and that $G_g$ is closed in $P'\times Z$.
  Then $G_{gf}$ is closed in $P\times Z$.
\end{lemma}

\begin{proof}
  We have $G_f \subset P\times Y = P\times G_g \subset P\times P'\times Z$, and since each one is closed in the next, we see that $G_f$ can be identified with a closed subset of $P\times P'\times Z$.
  Since $G_{gf}$ is exactly the projection of $G_f$ to the factor $P\times Z$, the lemma follows from \cref{lemma3}.
\end{proof}

\begin{lemma}{5}
\label{lemma5}
  Let $f\colon X\to Y$ be a morphism, and let $X\subset P$ and $X\subset P'$ be embeddings of $X$ into projective spaces.
  If $G_f$ is closed in $P\times Y$, then it is also closed in $P'\times Y$.
\end{lemma}

\begin{proof}
  We apply \cref{lemma4} to the morphisms $X\xrightarrow{i}X\xrightarrow{f}Y$, where $i$ denotes the identity morphism.
  Everything then reduces to showing that the graph $G_i$ of $i$ in $P\times X$ is closed, which follows from the fact that it is given by the intersection of $P\times X$ with the diagonal of $P\times P$.
\end{proof}

\Cref{lemma5} justifies the following definition:

\begin{definition}
  A map $f\colon X\to Y$ is said to be \emph{proper} if it is a morphism and if its graph $G_f$ is closed in $P\times Y$, where $P$ is a projective space containing $X$.
\end{definition}

We can give a definition of proper maps that is analogous to the definition of complete varieties:

\begin{proposition}{3}
\label{proposition3}
  For a morphism $f\colon X\to Y$ to be proper, it is necessary and sufficient, for every variety $Z$, and every closed subset $T$ of $X\times Z$, for the image of $T$ in $Y\times Z$ to be closed.
\end{proposition}

\begin{proof}
  Let $P$ be a projective space inside which $X$ can be embedded;
  since $G_f$ is closed in $P\times Y$, the product $G_f\times Z$ is closed in $P\times Y\times Z$, and so $T$ can be embedded as a closed subset into $P\times Y\times Z$.
\oldpage{101}
  Applying \cref{lemma3}, we see that the projection of $T$ to $Y\times Z$ (which is exactly $(f\times1)(T)$) is closed.
  Conversely, suppose that this property holds true, and apply it to $Z=P$, with the set $T$ being the diagonal of $X\times X$, embedded into $X\times P$.
  The image of $T$ in $Y\times Z=Y\times P$ is then exactly $G_f$, which is indeed closed.
\end{proof}

\begin{proposition}{4}
\label{proposition4}
  \begin{enumerate}[(i)]
    \item The identity morphism $i\colon X\to X$ is proper.
    \item The composition of two proper maps is proper.
    \item The direct product of two proper maps is proper.
    \item The image of a closed subset by a proper map is a closed subset.
    \item An injection $Y\to X$ is proper if and only if $Y$ is closed in $X$.
    \item Every morphism from a projective variety is proper.
    \item A projection $Y\times Z\to Y$ is proper if and only if $Z$ is projective (assuming the variety $Y$ to be non-empty).
  \end{enumerate}
\end{proposition}

\begin{proof}
  We indicate, as an example, how to prove \emph{(vii)} (since the other claims are even easier to prove).
  If $Z$ is projective, then we apply the criteria of \cref{proposition3};
  so let $Z'$ be an arbitrary variety, and $T$ a closed subset of $Y\times Z\times Z'$;
  we need to show that the projection of $T$ in $Y\times Z'$ is closed, which follows from \cref{lemma3}.
  Conversely, if $Y\times Z\to Y$ is proper, then the composition $Z\to Y\times Z\to Y$ is proper.
  Since the image of this map is a point, we immediately deduce that $Z$ is projective (by returning to the definition).
\end{proof}

\begin{corollary}{5}
\label{corollary5}
  For a morphism $f\colon X\to Y$ to be proper, it is necessary and sufficient for it to factor as $X\to P\times Y\to Y$, where $X\to P\times Y$ is an injection into a closed subvariety, and $P\times Y\to Y$ is the projection onto the second factor (where $P$ denotes some projective space).
\end{corollary}

\begin{proof}
  By the definition of a proper map, this condition is necessary (if we take $P$ to be a projective space into which we can embed $X$).
  It is sufficient by \emph{(ii)}, \emph{(v)}, and \emph{(vii)}.
\end{proof}

\begin{proposition}{5}
\label{proposition5}
  Suppose that the base field $k$ is the field of complex numbers.
  For a morphism $f\colon X\to Y$ to be proper (in the above sense), it is necessary and sufficient for it to be proper (in the topological sense) when we endow $X$ and $Y$ with the ``usual'' topology.
\end{proposition}

\begin{proof}
  Suppose that $f$ is proper in the algebraic sense, and let $K$ be a compact subset of $Y$ (for the usual topology).
  Suppose that $X$ is embedded into some projective space $P$.
  Since $P$ is compact, we know that $f^{-1}(K)=G_f\cap(P\times K)$ is compact, which shows that $f$ is proper in the topological sense.
  Conversely, assume that this condition is satisfied, and aim to prove that the condition of \cref{proposition3} is satisfied:
\oldpage{102}
  the image of $T$ in $Y\times Z$ is closed for the usual topology, and thus also for the Zariski topology \cite[proposition~7, p.~12]{13}.
\end{proof}

\begin{remark}
  The notion of a proper map can be extended to ``abstract'' (that is, non-quasi-projective) varieties:
  it suffices to take the criteria of \cref{proposition3} as a definition \cite{4}.
  \Cref{proposition4,proposition5} still hold true (if we replace ``projective'' with ``complete'' in item~\emph{(vii)} of \cref{proposition4}).
  The proofs are essentially the same:
  instead of using embeddings into projective spaces, we use the fact that every variety is the image of a quasi-projective variety under a proper map (Chow's lemma \cite{4,14}).
\end{remark}


\section{Image of a sheaf under a proper map}

Let $f\colon X\to Y$ be a morphism, with $Y$ a variety, and let $\sh{F}$ be a (coherent algebraic, as always) sheaf on $X$.
We define, by the classical procedure of Leray, sheaves $\RR^qf(\sh{F})$ on $Y$ by setting
\[
  \RR^qf(\sh{F})_U = \HH^q(f^{-1}(U),\sh{F})
  \quad\mbox{for every open subset $U$ of $Y$.}
\]
For $q=0$, we have the sheaf associated to the presheaf given by the $\HH^0(f^{-1}(U),\sh{F})$;
this is the \emph{direct image} of the sheaf $\sh{F}$.
We can show \cite{7} that the $\RR^qf$ are the \emph{derived functors} of the functor $\sh{F}\to\RR^0f(\sh{F})$ (where $\sh{F}$ runs over the category of all sheaves on $X$, coherent or not).

\begin{examples}
  \begin{enumerate}
    \item If $X\to Y$ is an injection of a closed subvariety, then the sheaf $\RR^0f(\sh{F})$ is exactly the sheaf $\sh{F}$ extended by $0$ outside of $X$, and the sheaves $\RR^qf(\sh{F})$, for $q\geq1$, are zero (let $U$ be an affine open; then $f^{-1}(U)=U\cap X$ is affine, whence $\RR^qf(\sh{F})=0$).
      \label{example1}
    \item Let $Y$ be a point.
      A sheaf on a point is simply a group (or a $k$-vector space, if we are talking about algebraic sheaves).
      The $\RR^qf(\sh{F})$ are then simply the cohomology groups $\HH^q(X,\sh{F})$;
      we note that these are not necessarily vector spaces \emph{of finite dimension} (or, in other words, not necessarily \emph{coherent} sheaves on $Y$).
        \label{example2}
    \item Suppose that $f\colon X\to Y$ defines a birational isomorphism between the varieties $X$ and $Y$ (assumed to be projective and non-singular).
      Take $\sh{F}$ to be the sheaf $\sh{O}_X$ of local rings of $X$;
      we immediately see that $\RR^0f(\sh{O}_X)=\sh{O}_Y$.
      Is it true that $\RR^qf(\sh{O}_X)=0$ for $q\geq1$?
      We can at least verify this for ``blow-ups'', and it would be interesting to know the answer in the general case.
      \label{example3}
  \end{enumerate}
\end{examples}

We note that Leray's theory can be translated without any changes (see \cite{7});
there is a spectral sequence abutting $\HH^\bullet(X,\sh{F})$, and with $E_2^{p,q}=\HH^p(Y,\RR^qf(\sh{F}))$.
If we apply, for example, this spectral sequence
\oldpage{103}
to \hyperref[example3]{Example~3} above, then we see that $\RR^qf(\sh{O}_X)=0$ for $q\geq1$ implies that $\HH^\bullet(X,\sh{O}_X)=\HH^\bullet(Y,\sh{O}_Y)$.

We have seen (\hyperref[example2]{Example~2}) that the $\RR^qf(\sh{F})$ are not, in general, coherent sheaves on $Y$.
However:

\begin{theorem}{1}
  If $f\colon X\to Y$ is proper, then the $\RR^qf(\sh{F})$, for $q>0$, are coherent sheaves on $Y$, for any coherent sheaf $\sh{F}$ on $X$.
\end{theorem}


%% Bibliography %%

\nocite{*}
\bibliographystyle{acm}

\begin{thebibliography}{10}

  \bibitem{1}
  {\sc Atiyah, M.}
  \newblock Vector bundles over an elliptic curve.
  \newblock {\em Proc. London math. Soc. 7\/} (1957), 414--452.

  \bibitem{2}
  {\sc Borel, A., and Hirzebruch, F.}
  \newblock Characteristic classes and homogeneous spaces, {II}.
  \newblock {\em Amer. J. Math.\/} (to appear).

  \bibitem{3}
  {\sc Cartan, H., and Eilenberg, S.}
  \newblock {\em Homological algebra}, vol.~19 of {\em Princeton Math. Series}.
  \newblock Princeton University Press, 1956.

  \bibitem{4}
  {\sc Chevalley, C.}
  \newblock La notion de correspondance propre en géométrie algébrique.
  \newblock In {\em Séminaire Bourbaki}, vol.~10.
  \newblock (Talk number 152).

  \bibitem{5}
  {\sc Chow, W.}
  \newblock On equivalence classes of cycles in an algebraic variety.
  \newblock {\em Ann. Math. 64\/} (1956), 450--479.

  \bibitem{6}
  {\sc Grothendieck, A.}
  \newblock Sur les faisceaux algébriques et les faisceaux analytiques
    cohérents.
  \newblock In {\em Séminaire H. Cartan}, vol.~9.
  \newblock (Talk number 2).

  \bibitem{7}
  {\sc Grothendieck, A.}
  \newblock Sur quelques points d'algèbre homologique.
  \newblock {\em Tohoku math. J. 9\/} (1957), 119--221.

  \bibitem{8}
  {\sc Grothendieck, A.}
  \newblock {\em Bull. Soc. math. France 80\/} (1958), 137--154.

  \bibitem{9}
  {\sc Hirzebruch, F.}
  \newblock {\em Neue topologische Methoden in der algebraischen Geometrie}.
  \newblock Ergebnisse der Mathematik. Berlin, Springer, 1956.
  \newblock neue Folge, Heft~9.

  \bibitem{10}
  {\sc Samuel, P.}
  \newblock Rational equivalence of arbitrary cycles.
  \newblock {\em Amer. J. Math. 78\/} (1956), 383--400.

  \bibitem{11}
  {\sc Chevalley, S.~C.}
  \newblock {\em Anneaux de Chow et applications}, vol.~2.
  \newblock 1958.

  \bibitem{12}
  {\sc Serre, J.-P.}
  \newblock Faisceaux algébriques cohérents.
  \newblock {\em Ann. Math. 61\/} (1955), 197--279.

  \bibitem{13}
  {\sc Serre, J.-P.}
  \newblock Géométrie algébrique et géométrie analytique.
  \newblock {\em Ann. Inst. Fourier, Grenoble 6\/} (1955--1956), 1--42.

  \bibitem{14}
  {\sc Serre, J.-P.}
  \newblock Sur la cohomologie des variétés algébriques.
  \newblock {\em J. Math. pures et appl. 36\/} (1957), 1--16.

\end{thebibliography}


\end{document}
