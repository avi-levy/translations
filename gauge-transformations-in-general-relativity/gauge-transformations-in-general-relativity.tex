\documentclass{article}

\usepackage{amssymb,amsmath}

\usepackage{hyperref}
\usepackage[nameinlink]{cleveref}
\usepackage{enumerate}
\usepackage{tikz-cd}
\usepackage{graphicx}

\usepackage{mathrsfs}
%% Fancy fonts --- feel free to remove! %%
\usepackage{ebgaramond-maths}
\usepackage{mathpazo}


\crefname{section}{Section}{Sections}
\crefname{equation}{}{}

%% Theorem environments %%

\usepackage{amsthm}

  \theoremstyle{plain}

  \newtheorem{innercustomtheorem}{Theorem}
  \crefname{innercustomtheorem}{Theorem}{Theorems}
  \newenvironment{theorem}[1]
    {\renewcommand\theinnercustomtheorem{#1}\innercustomtheorem}
    {\endinnercustomtheorem}

  \newtheorem{innercustomproposition}{Proposition}
  \crefname{innercustomproposition}{Proposition}{Propositions}
  \newenvironment{proposition}[1]
    {\renewcommand\theinnercustomproposition{#1}\innercustomproposition}
    {\endinnercustomproposition}

  \newtheorem{innercustomlemma}{Lemma}
  \crefname{innercustomlemma}{Lemma}{Lemmas}
  \newenvironment{lemma}[1]
    {\renewcommand\theinnercustomlemma{#1}\innercustomlemma}
    {\endinnercustomlemma}

  \newtheorem{innercustomcorollary}{Corollary}
  \crefname{innercustomcorollary}{Corollary}{Corollaries}
  \newenvironment{corollary}[1]
    {\renewcommand\theinnercustomcorollary{#1}\innercustomcorollary}
    {\endinnercustomcorollary}


  \theoremstyle{definition}

  \newtheorem*{remark}{Remark}

  \newtheorem{innercustomdefinition}{Definition}
  \crefname{innercustomdefinition}{Definition}{Definitions}
  \newenvironment{definition}[1]
    {\renewcommand\theinnercustomdefinition{#1}\innercustomdefinition}
    {\endinnercustomdefinition}


%% Shortcuts %%

\renewcommand{\gg}{\mathfrak{g}}
\renewcommand{\geq}{\geqslant}
\renewcommand{\leq}{\leqslant}

\newcommand{\todo}{\textbf{ !TODO! }}
\newcommand{\oldpage}[1]{\marginpar{\footnotesize$\Big\vert$ \textit{p.~#1}}}


%% Document %%

\usepackage{embedall}
\begin{document}

\renewcommand{\abstractname}{Translator's note.}

\title{Gauge transformations in general relativity}
\author{J. Winogradzki}
\date{18\textsuperscript{th} of January, 1955}
\maketitle

\begin{abstract}
  \renewcommand*{\thefootnote}{\fnsymbol{footnote}}
  \emph{This text is one of a series\footnote{\url{https://github.com/thosgood/translations}} of translations of various papers into English.}
  \emph{What follows is a translation (last updated \today) of the French paper:}

  \medskip\noindent
  \textsc{Winogradzki, J}. Les transformations de jauge en relativité généralisée. \emph{Séminaire L. de Broglie. Théories physiques}, Volume~24 (1954-1955), Talk no.~10, 10~p. {\footnotesize\url{http://www.numdam.org/item/SLDB_1954-1955__24__A9_0/}}
\end{abstract}

\renewcommand*{\thefootnote}{(\arabic{footnote})}
\setcounter{footnote}{0}

\tableofcontents


%% Content %%

\bigskip\bigskip
\oldpage{10-01}
The Universe of the theory of General Relativity is a quadri-dimensional affine and metric space, with the metric being given by a second-order tensor which represents the field.
In general, the affine connection and the metric tensor are asymmetric.
The equations of the field were established in different ways, one of which, due to Einstein, is particularly interesting.
The principal of General Relativity being insufficient to determine the equations of the generalised field, Einstein substituted the principal of General Relativity with a stricter theory of Relativity, and it is from this principal that he deduced the field equations.
He showed that, effectively, the field equations are practically determined if we ask for them to be invariant under $\lambda$-transformations.
\footnote{\textsc{A. Einstein}. Generalization of Gravitation Theory. Appendix II of \emph{the Meaning of Relativity}. Princeton, 1953. \emph{(We will use the notation from this paper.)}\\\textsc{A. Einstein}. Extension du group relativiste. Dans \emph{Louis de Broglie Physicien et Penseur}. Albin Michel, 1953.}

It is not necessary to impose a priori the precise nature of the extension of the relativistic group; some simple consideration suggests a postulate from which it can be determined.
The same postulate leads to the field equations.
\footnote{\textsc{J. Winogradzki}, \emph{C.R. Acad. Sc.}, \textbf{239}, 1954, p.~1359.}


\section{Axiomatic foundations}
\label{section1}

One of the classical methods for establishing the pure gravitational field equations consists of deducing them from the set (E) of the following postulates:
\begin{enumerate}[I.]
  \item The universe is a quadri-dimensional affine and metric space, with the metric being given by a second-order tensor.
  \item The affine connection and the metric tensor are symmetric.
  \item The field equations derive from a variational principle.
\oldpage{10-02}
    We vary the affine connection and the metric tensor.
  \item The Hamiltonian is given by the curvature density.
  \item The variations take place without any a priori conditions.
\end{enumerate}

If we remove postulate~II, then the Universe has multiple curvature densities that can be deduced from one another by transposition, either of the metric tensor, or of the affine connection, or both:
\[
  \gg^{ik}R_{ik}, \,\,
  \gg^{ik}R_{ki}, \,\,
  \gg^{ik}\widetilde{R}_{ik}, \,\,
  \gg^{ik}\widetilde{R}_{ki}
\]
with $\gg^ik$ the density of the metric tensor, and $R_{ik}$ and $\widetilde{R}_{ik}$ the two Ricci tensors:
\[
\label{equation1}
  R_{ik} = \Gamma_{ik,m}^m - \Gamma_{im,k}^m + \Gamma_{pm}^m\Gamma_{ik}^p - \Gamma_{ip}^m\Gamma_{mk}^p
  \tag{1}
\]
\[
\label{equation2}
  \widetilde{R}_{ik} = \Gamma_{ki,m}^m - \Gamma_{mi,k}^m + \Gamma_{mp}^m\Gamma_{ki}^p - \Gamma_{pi}^m\Gamma_{km}^p.
  \tag{2}
\]
Avoiding any arbitrary choice, we modify postulate~IV:
\begin{enumerate}
  \item[IVa.] The Hamiltonian is given by any one of the curvature densities.
\end{enumerate}

The set (E'), consisting of postulates~I, III, IVa, and V, does not determine the field equations in an unequivocal way, since the field equations then depend on the choice of the Hamiltonian.
The set (E') of postulates is thus unacceptable.
We will try to modify it whilst staying as close as possible to (E).

The set (E) of postulates is equivalent to the set containing postulates~I, III, IVa, and another postulate that combines II and V:
\begin{enumerate}
  \item[Va.] The variations take places under the a priori conditions $g_{ik}=g_{ki}$ and $\Gamma_{km}^i=\Gamma_{mk}^i$.
\end{enumerate}


\end{document}
