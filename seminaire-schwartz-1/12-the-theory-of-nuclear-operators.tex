\documentclass{article}

\usepackage{amssymb,amsmath}

\usepackage{hyperref}
\usepackage[nameinlink]{cleveref}
\usepackage{enumerate}
\usepackage{tikz-cd}

\usepackage{mathrsfs}
%% Fancy fonts --- feel free to remove! %%
\usepackage{Baskervaldx}
\usepackage{mathpazo}


\crefname{section}{Section}{Sections}
\crefname{equation}{}{}


%% Theorem environments %%

\usepackage{amsthm}

  \theoremstyle{plain}

  \newtheorem{innercustomtheorem}{Theorem}
  \crefname{innercustomtheorem}{Theorem}{Theorems}
  \newenvironment{theorem}[1]
    {\renewcommand\theinnercustomtheorem{#1}\innercustomtheorem}
    {\endinnercustomtheorem}

  \newtheorem{innercustomproposition}{Proposition}
  \crefname{innercustomproposition}{Proposition}{Propositions}
  \newenvironment{proposition}[1]
    {\renewcommand\theinnercustomproposition{#1}\innercustomproposition}
    {\endinnercustomproposition}

  \newtheorem{innercustomlemma}{Lemma}
  \crefname{innercustomlemma}{Lemma}{Lemmas}
  \newenvironment{lemma}[1]
    {\renewcommand\theinnercustomlemma{#1}\innercustomlemma}
    {\endinnercustomlemma}

  \newtheorem{innercustomcorollary}{Corollary}
  \crefname{innercustomcorollary}{Corollary}{Corollaries}
  \newenvironment{corollary}[1]
    {\renewcommand\theinnercustomcorollary{#1}\innercustomcorollary}
    {\endinnercustomcorollary}


  \theoremstyle{definition}

  \newtheorem*{remark}{Remark}

  \newtheorem{innercustomdefinition}{Definition}
  \crefname{innercustomdefinition}{Definition}{Definitions}
  \newenvironment{definition}[1]
    {\renewcommand\theinnercustomdefinition{#1}\innercustomdefinition}
    {\endinnercustomdefinition}


%% Shortcuts %%

\newcommand{\sh}{\mathscr}
\newcommand{\cat}{\mathcal}
\newcommand{\BB}{\mathcal{B}}
\newcommand{\LL}{\mathcal{L}}
\newcommand{\projotimes}{{\otimes}_\pi}
\newcommand{\injotimes}{{\otimes}_\varepsilon}
\newcommand{\cprojotimes}{\widehat{\otimes}_\pi}
\newcommand{\cinjotimes}{\widehat{\otimes}_\varepsilon}
\newcommand{\cotimes}{\widehat{\otimes}}
\newcommand{\tr}{\operatorname{Tr}}
\newcommand{\transpose}[1]{{}^t\!{#1}}

\renewcommand{\geq}{\geqslant}
\renewcommand{\leq}{\leqslant}

\newcommand{\todo}{\textbf{ !TODO! }}
\newcommand{\oldpage}[1]{\marginpar{\footnotesize$\Big\vert$ \textit{p.~#1}}}


%% Document %%

\usepackage{embedall}
\begin{document}

\renewcommand{\abstractname}{Translator's note.}

\title{The theory of nuclear operators}
\author{L. Schwartz}
\date{17\textsuperscript{th} of February, 1954}
\maketitle

\begin{abstract}
  \renewcommand*{\thefootnote}{\fnsymbol{footnote}}
  \emph{This text is one of a series\footnote{\url{https://github.com/thosgood/translations}} of translations of various papers into English.}
  \emph{The translator takes full responsibility for any errors introduced in the passage from one language to another, and claims no rights to any of the mathematical content herein.}
  
  \emph{What follows is a translation (last updated \today) of the French paper:}

  \medskip\noindent
  \textsc{Schwartz, L.}
  ``La th\'{e}orie des op\'{e}rateurs nucl\'{e}aires''.
  \emph{S\'{e}minaire Schwartz}, Volume~\textbf{1} (1953-1954), Talk no.~12, 7~p.
  {\footnotesize\url{http://www.numdam.org/item/SLS_1953-1954__1__A13_0/}}
\end{abstract}

\setcounter{footnote}{0}

\tableofcontents


%% Content %%

\section{The trace}
\label{section1}

\oldpage{1}

Let $E$ be a \emph{Banach} space, whose dual we will call $E'$.
We know, by definition, that there exists a bijective and isometric correspondence between the space $\BB(E,E')$ of continuous bilinear forms on $E\times E'$ and the dual $E\cprojotimes E'$.
To the canonical bilinear form $(x,x')\mapsto\langle x,x'\rangle$ thus corresponds a continuous linear form on $E\cprojotimes E'$ that we call ``the trace'', and that we denote by $\tr$.
If $u=\sum_v x_v\otimes y'_v$ then, by definition, $\tr(u)=\sum_v\langle x_v,y'_v\rangle$.
The trace form is of norm $1$.
Furthermore, every $u\in E\cprojotimes E'$ can be written in the form $u=\sum_{n\geq0}x_n\otimes y'_n$ with $\sum_{n\geq0}\|x_n\|\|y'_n\|$ finite, and so the series $\sum_{n\geq0}\langle x_n,y'_n\rangle$ converges absolutely, and, since the trace is continuous, we have that
\[
  \tr(u) = \sum_{n\geq0} \langle x_n, y'_n \rangle.
\]

To justify the name ``trace'', recall that we can identify $E\otimes E'$ with the space of endomorphisms of finite rank of $E$, and that, if $E$ is of finite dimensions, then the trace form agrees with the usual trace of operators.

There exists a canonical continuous map $E'\cprojotimes E\to\LL(E;E)$.
If we do not know whether or not it is bijective, we can only speak of the trace of an element of $E'\cprojotimes E$, and not the trace of the image of the operator in $\LL(E;E)$.

Recall as well that there exists an isomorphism $S$ (for symmetry) between $E\otimes E'$ and $E'\otimes E$, defined by
\[
  S\colon \sum_v x_v\otimes y'_v \mapsto \sum_v y'_v\otimes x_v
\]
for $x_v\in E$ and $y'_v\in E'$.

If we identify $E\otimes E'$ with the space of maps of finite rank from $E$ to $E$, and $E'\otimes E\subset E'\otimes(E')'$ with a space of transformations of $E'$,
\oldpage{2}
then the map $S$ corresponds to the transposition of operators.
Thanks to $S$, the trace is also defined on $E'\cprojotimes E$.
We can thus understand the duality between $E\cprojotimes F$ and $\BB(E,F)$ by means of the trace: let $A\in\BB(E,F)\subset\LL(E;F')$.
If $1$ is the identity in $F$, then $A\otimes1$ sends $E\cprojotimes F$ to $F'\cprojotimes F$.
So if $u\in E\cprojotimes F$, then we can take the trace of $(A\otimes1)(u)\in\LL(F;F')$, and we have
\[
\label{equation1}
  \langle u,A \rangle = \tr((A\otimes1)(u)).
  \tag{1}
\]
Indeed, both sides of the equation (for fixed $A$) are continuous linear forms in $u$, and are equal for $u=x\otimes y$.


\section{The map $E'\cprojotimes F\to\LL_b(E_\tau;F)$ for $E$ and $F$ locally convex}
\label{section2}

The subscript ${}_b$ denotes the uniform convergence topology on bounded subsets of a space of linear maps.

Let $E$ and $F$ be arbitrary locally convex separated spaces.
Elements of $E'\otimes F$ correspond to continuous linear maps of finite rank from $E$ to $F$.
So $E'\otimes F\subset\LL(E_\tau;F)$, since the latter is the space of weakly continuous maps (see Expos\'{e}~8, §1).

\begin{proposition}{1}
\label{proposition1}
  The topology induced on $E'\otimes F$ by $\LL_b(E_\tau;F)$ is identical to the topology of $E'_b\injotimes F$.
\end{proposition}

\begin{proof}
  The topology of $E'_b\injotimes F$ is, by definition, the topology induced on $E'\otimes F$ by $\LL_\varepsilon((E'')_\tau;F)$.
  But an equicontinuous subset of $E''$ is the  polar of a neighbourhood of $0$ in $E'$, which is itself the polar of a bounded subset of $E$, and thus (by the bipolar theorem) is the weakly closed convex balanced hull of a bounded subset of $E$.
  But, in a $\mathfrak{S}$-topology, we can replace the sets of $\mathfrak{S}$ by their closed convex balanced hull.
  Thus $\LL_\varepsilon((E'')_\tau;F)$ and $\LL_b(E_\tau;F)$ induce the same topology on $E'\otimes F$.
\end{proof}

\begin{corollary}{1}
\label{corollary-1}
  If $E$ and $F$ are complete, then there exists a continuous map $\varphi$ from $E'\projotimes F$ to $\LL_b(E_\tau;F)$ that extends the identity on $E'\otimes F$.
\end{corollary}

\begin{proof}
  Indeed, the $\pi$-topology being finer than the $\varepsilon$-topology, there exists a canonical map $E'\cprojotimes F\to E'\cinjotimes F$ which we can compose with the map $E'\cinjotimes F\to \LL_b(E_\tau;F)$.
\end{proof}

\section{Definition of nuclear maps --- the case of Banach spaces}
\label{section3}

\oldpage{3}
\emph{From now on, the only tensor product that we will consider is the $\pi$-product; thus $E\cotimes F$ means $E\cprojotimes F$.}

\begin{definition}{1}
\label{definition1}
  If $E$ and $F$ are Banach spaces, then we write $L^1(E;F)$ to denote the subspace $\varphi(E'\cotimes F)$ of $\LL(E;F)$.
  The elements of $L^1(E;F)$ are called \emph{nuclear} (or \emph{Fredholm}) operators.
  Note that $L^1(E;F)$ is a \emph{quotient space} of $E\cotimes F$.
  The quotient norm of the $\pi$-norm will be called the \emph{trace norm}, or the \emph{nuclear norm}, denoted by $\|\cdot\|_1$ or $\|\cdot\|_{\tr}$.
\end{definition}

We do not know a case where $\varphi$ is not bijective, but we do not know how to prove this in general.

Since $E'\otimes F$ is dense in $E'\cotimes F$, and $\varphi$ is continuous, every nuclear operator is the \emph{``uniform'' limit} (in $\LL_b$) \emph{of operators of finite rank}, and is thus, in particular, \emph{compact} (since the image in $F$ of a ball in $E$ is relatively compact).

\begin{remark}
  If $E=F$ is a Hilbert space, then the \emph{hermitian} nuclear operators are exactly the completely continuous operators $u$ such that the sequence $(\lambda_n)$ of eigenvalues is summable and such that
  \[
    \|u\|_1 = \sum_n|\lambda_n|.
  \]

  In the general Banach case, every nuclear operator $u$ admits a decomposition
  \[
    u = \sum\lambda_i x'_i\otimes y_i
  \]
  where $x'_i\in E'$ are such that $\|x'_i\|_1\leq1$ and $y_i\in F$ are such that $\|y_i\|\leq1$, and such that $\sum|\lambda_i|<\infty$;
  the lower bound of $\sum|\lambda_i|$ for any such decomposition is exactly $\|u\|_1$.
\end{remark}

\begin{proposition}{2}
\label{proposition2}
  Let $u\colon E\to F$ be a nuclear operator, and let $A\colon H\to E$ and $B\colon F\to G$ be continuous maps.
  Then $B\circ u\circ A$ is a nuclear operator, and $\|B\circ u\circ A\|_1 \leq \|A\|\|u\|_1\|B\|$.
\end{proposition}

\begin{proof}
  We have the commutative diagram
  \[
    \begin{tikzcd}[column sep=huge]
      E'\cotimes F \ar[r,"t_{A\otimes B}"] \ar[d,swap,"\varphi"]
      & H'\cotimes G \ar[d,"\varphi"]
    \\\LL(E;F) \ar[r,swap,"u\mapsto B\circ u\circ A"]
      & \LL(H;G)
    \end{tikzcd}
  \]
  (since the two maps from $E'\cotimes F$ to $\LL(H;G)$ that define this diagram are continuous, and agree for $u_0\in E'\cotimes F$ of the form $x'\otimes y$).
  So, if $u$ is nuclear, with $u_0$ an element of $E\cotimes F$ such that $\varphi(u_0)=u$, then $(\transpose{A}\otimes B)u_0\in H'\cotimes G$, and $B\circ u\circ A=\varphi((\transpose{A}\otimes B)(u_0))$, and so $B\circ u\circ A$ is nuclear.
  Taking into account the fact that $\|\transpose{A}\otimes B\|=\|A\|\|B\|$,
\oldpage{4}
  we have that
  \[
    \|B\circ u\circ A\|_1
    \leq \inf_{\varphi(u_0)=u}(\transpose{A}\otimes B)(u_0)
    \leq \inf_{\varphi(u_0)=u}\|A\|\|B\|\|u_0\|
    = \|A\|\|B\|\|u\|_1.
  \]
\end{proof}


\section{Definition of nuclear operators --- the general case}
\label{section4}

\begin{definition}{2}
\label{definition2}
  We say that a linear map $u\colon E\to F$, where $E$ and $F$ are locally convex separated spaces, is \emph{nuclear} if there exist Banach spaces $E_1$ and $F_1$, a nuclear operator $\beta\colon E_1\to F_1$, and continuous operators $\alpha\colon E\to E_1$ and $\gamma\colon F_1\to F$ such that $u=\gamma\circ\beta\circ\alpha$, i.e. such that
  \[
    \begin{tikzcd}[column sep=small]
      E \ar[r,"\alpha"] \ar[rrr,bend right,swap,"u"]
      & E_1 \ar[r,"\beta"]
      & F_1 \ar[r,"\gamma"]
      & F
    \end{tikzcd}
  \]
  commutes.
\end{definition}

\begin{remark}
  It suffices for $F_1$ to be Banach and $E_1$ to be normed, since we can extend $\beta$ to $\widehat{E_1}$.
\end{remark}

To simplify, we call any convex balanced set a \emph{disc}.
By replacing $E_1$ with $\alpha(E)$, and $F_1$ with $F_1/\gamma^{-1}(0)$, we can assume that $\alpha$ is an epijection and $\gamma$ is an injection;
but we know (expos\'{e}~7) that $E_1$ will be isomorphic to $E_{U_1}$, and $F_1$ to $F_{B_1}$, for $U_1$ some open disc of $E$, and $B_1$ some \textbf{compl\'{e}tante}\footnote{\emph{[Translator]. I was unable to find a translation for this term, but I \textbf{think} it refers to the following property: an absolutely convex subset $S$ of a topological vector space is said to be \textbf{compl\'{e}tante} if $S_A$ is a Banach space, where $S_A$ is the subset absorbed by $S$.}} subset of $F$.
Since the dual of $E_{U_1}$ is $\widehat{E'_{A'_1}}$, where $A'_1=U_1^0$, we know that $\beta$ comes from an element $\beta_0$ of $E'_{A'_1}\cotimes F_{B_1}$, and that $u=\gamma\circ\beta\circ\alpha$ comes from an element $u_0=(\transpose{\alpha}\otimes\gamma)(\beta_0)$ of $E'\cotimes F$;
but $\transpose{\alpha}\otimes\gamma$ is exactly the canonical map from $E'_{A'_1}\cotimes F_{B_1}$ in $E'\cotimes F$.
Even stronger: $u$ comes from some $u_0\in E'_{A'_1}\cotimes F_{B_1}$, but we know (expos\'{e}~5) that $u_0$ belongs to a set of the form $\Gamma(A'\otimes B)$, where $A$ and $B$ are compact subsets of $E'_{A'_1}$ and $F_{B_1}$ (respectively).
Whence:

\begin{proposition}{3}
\label{proposition3}
  An operator $u\colon E\to F$ is nuclear if and only if it is defined by an element of some $E'_{A'}\otimes F_B$, where $A'$ and $B$ are compact convex balanced (and thus \textbf{compl\'{e}tante}) subsets.
  We can thus suppose, in \cref{definition2}, that $\alpha$ and $\beta$ are compact maps.
\end{proposition}

Note also that, since $E'_{A'}$ is the dual of $E_{A'_0}$, we have two ``canonical'' continuous maps:
\[
  E'_{A'}\cotimes F_B \to \LL_b(E_{A'_0};F_B) \to \LL_b(E;F).
\]
Since any element of $E'_{A'}\cotimes F_B$ can be written in the form
\[
  u = \sum\lambda_i x'_i\otimes y_i
\]
\oldpage{5}
we have such an equality in $\LL_b(E;F)$.

Conversely, if $\sum|\lambda_i|<+\infty$, if $(x'_i)$ is an equicontinuous sequence, and if $(y_i)$ is contained inside a \textbf{compl\'{e}tante} subset of $F$, then $\sum\lambda_i x'_i\otimes y_i$ converges in $\LL_b(E;F)$, and defines a nuclear operator.

\begin{proposition}{4}
\label{proposition4}
  For an operator $u$ to be nuclear, it is necessary and sufficient for it to be of the form $u=\sum\lambda_i x'_i\otimes y_i$, where $\sum|\lambda_i|<+\infty$, $(x'_i)$ is an equicontinuous sequence, and $(y_i)$ a sequence contained inside some \textbf{compl\'{e}tante} subset.
\end{proposition}

\begin{proposition}{5}
\label{proposition5}
  If $u$ is nuclear, then $B\circ u\circ A$ is nuclear (\cref{proposition2}).
\end{proposition}

\begin{corollary}{1}
\label{corollary-5}
  If $u\colon E\to F$ is nuclear, then it remains nuclear when we strengthen the topology of $E$ and weaken the topology of $F$;
  if $E_1$ is a subspace of $E$, and $F$ a subspace of $F_1$, then the restriction $u\colon E_1\to F_1$ is nuclear.
\end{corollary}

However, if $u$ is a nuclear map from $E$ to $F$, and if $u(E)$ is contained in a subspace $F_2$ of $F$, then $u\colon E\to F_2$ is not necessarily nuclear.
Similarly, if $u$ is zero on a subspace $E_2$ of $E$, then $u\colon E/E_2\to F$ is not necessarily nuclear.


\section{Transpose of a nuclear map}
\label{section5}

\begin{proposition}{6}
\label{proposition6}
  Let $E$ and $F$ be Banach spaces, and $u\in L^1(E;F)$.
  Then $\transpose{u}\in L^1(F';E')$, and $\|\transpose{u}\|_1\leq\|u\|_1$.
  Conversely, if $F$ is reflexive, and $\transpose{u}$ is nuclear, then $u$ is nuclear, and $\|\transpose{u}\|_1=\|u\|)1$.
\end{proposition}

\begin{proof}
  Let $u\in L^1(E;F)$ with $u=\varphi(u_0)$, where $u_0\in E'\cotimes F$.
  Let $i$ be the injection from $F\cotimes E'$ to $F''\cotimes E'$.
  Then $\transpose{u}\colon F'\to E'$ is given by $\transpose{u}=\varphi(i(S(u_0)))$, and so $\transpose{u}$ is nuclear.
  Since $S$ is an isometry, and $\|i\|\leq1$ (in fact, we can even show that $i$ is an isometry), we have that
  \[
    \|\transpose{u}\|_1
    \leq \inf_{\varphi(u_0)=u} \|i(S(u_0))\|
    \leq \inf_{\varphi(u_0)=u} \|u_0\|
    = \|u\|_1.
  \]
  Finally, if $F$ is \emph{reflexive}, and $\transpose{u}$ is nuclear, then $\transpose{\,\transpose{u}}\colon E''\to F''=F$ is nuclear, and so $u\colon E\to F$ is nuclear.
  In all known cases, this property still holds true even without the reflexivity hypothesis on $F$.
\end{proof}

\begin{corollary}{1}
\label{corollary-6}
  Let $E$ and $F$ be locally convex separated spaces;
  if $u\colon E\to F$ is nuclear, then $\transpose{u}\colon F'_c\to E'_b$ is nuclear, and, a fortiori, $\transpose{u}\colon F'_b\to E'_b$ or $\transpose{u}\colon F'_c\to E'_c$.
\end{corollary}

\begin{proof}
  Indeed, $\transpose{u}=\transpose{\alpha}\transpose{\beta}\transpose{\gamma}$ (see \cref{definition2}), with $\transpose{\beta}$ nuclear, $\transpose{\alpha}$ continuous, and $\transpose{\gamma}$ continuous from $F'_c$ to $F'_1$ if $\gamma$ is compact, which we have the right to assume.
\end{proof}


\section{Lifting properties}
\label{section6}

\oldpage{6}
\begin{proposition}{7}
\label{proposition7}
  Let $E$, $F$, and $G$ be locally convex separated spaces, with $E\subset F$;
  let $u\colon E\to G$ be a nuclear map.
  Then there exists a nuclear map $v\colon F\to G$ extending $u$.
  Furthermore, in the Banach case, we can assume that $\|v\|_1\leq\|u\|_1+\varepsilon$.
\end{proposition}

\begin{proof}
  We restrict ourselves to proving the Banach case.

  Consider the diagram that we have already seen (\cref{proposition2}):
  \[
    \begin{tikzcd}[column sep=huge]
      F'\cotimes G \ar[r,"t_{i\otimes1}"] \ar[d,swap,"\varphi"]
      & E'\cotimes G \ar[d,"\varphi"]
    \\L^1(F;G) \ar[r,swap,"u\mapsto i\circ u"]
      & L^1(E;G)
    \end{tikzcd}
  \]
  where $i$ is the injection of $E$ into $F$.
  Then the path \rotatebox[origin=c]{270}{$\Rsh$} is a metric epimorphism, and thus so too is the path \rotatebox[origin=c]{180}{$\Lsh$}.
\end{proof}

\begin{proposition}{8}
\label{proposition8}
  Let $E$, $F$, and $G$ be locally convex separated spaces, with $F\subset E$, and $F$ closed;
  suppose that every compact disc of $E/F$ is the image of a \textbf{compl\'{e}tante} subset of $E$.
  Then every nuclear map $u\colon G\to E/F$ comes from the image (under taking the quotient) of a nuclear map $v\colon G\to E$.
  Furthermore, in the Banach case, we can assume that $\|v\|_1\leq\|u\|1+\varepsilon$.
\end{proposition}

\begin{proof}
  Let $H=E/F$.
  Suppose that $u$ comes from some element $u_0$ of $G'_{A'}\cotimes H_B$, where $B$ is a \emph{compact} \textbf{compl\'{e}tante} subset of $H$ (\cref{proposition3}).
  Let $B_1$ be a \textbf{compl\'{e}tante} subset of $E$ that projects onto $B$.
  We have an epimorphism $E_{B_1}\to H_B$, and it suffices to show that $u_0$ can be obtained from an element of $G'_{A'}\cotimes E_{B_1}$ by projection, i.e. that we can reduce to the Banach case.
  But in this case, we have the following diagram:
  \[
    \begin{tikzcd}[column sep=huge]
      G'\cotimes E \ar[r,"1\otimes P"] \ar[d,swap,"\varphi"]
      & G'\cotimes H \ar[d,"\varphi"]
    \\L^1(G;E) \ar[r,swap,"u\mapsto P\circ u"]
      & L^1(G;H)
    \end{tikzcd}
  \]
  and, again, \rotatebox[origin=c]{270}{$\Rsh$} is a epimorphism, and thus so too is \rotatebox[origin=c]{180}{$\Lsh$}.
\end{proof}

\begin{remark}
  \begin{enumerate}
    \item The conditions of \cref{proposition8} are satisfied if $E$ is a Fr\'{e}chet space (or if $E$ is a dual of a Fr\'{e}chet space) and $F$ is weakly
\oldpage{7}
      closed.
    \item Returning to \cref{proposition7}: if we use \cref{proposition4}, then we can write $u$ in the form $u=\sum\lambda_i y'_i\otimes z_i$, and, if we simultaneously extend (by Hahn-Banach) the $y'_i$ to equicontinuous forms $\overline{y'_i}$ on $F$, then we can set $v=\sum\lambda_i\overline{y'_i}\otimes z_i$, and $v$ extends $u$, which gives another proof of the proposition (and similarly for \cref{proposition8})
  \end{enumerate}
\end{remark}


\end{document}
