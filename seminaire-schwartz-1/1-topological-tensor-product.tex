\documentclass{article}

\usepackage{amssymb,amsmath}

\usepackage{hyperref}
\usepackage[nameinlink]{cleveref}
\usepackage{enumerate}
\usepackage{tikz-cd}

\usepackage{mathrsfs}
%% Fancy fonts --- feel free to remove! %%
\usepackage{Baskervaldx}
\usepackage{mathpazo}


\crefname{section}{Section}{Sections}
\crefname{equation}{}{}


%% Theorem environments %%

\usepackage{amsthm}

  \theoremstyle{plain}

  \newtheorem{innercustomproposition}{Proposition}
  \crefname{innercustomproposition}{Proposition}{Propositions}
  \newenvironment{proposition}[1]
    {\renewcommand\theinnercustomproposition{#1}\innercustomproposition}
    {\endinnercustomproposition}

  \newtheorem*{theorem}{Theorem}

  \theoremstyle{definition}

  \newtheorem*{remark}{Remark}
  \newtheorem*{definition}{Definition}


%% Shortcuts %%

\newcommand{\sh}{\mathscr}
\newcommand{\cat}{\mathcal}
\newcommand{\BB}{\mathcal{B}}
\newcommand{\LL}{\mathcal{L}}
\newcommand{\cotimes}{\widehat{\otimes}}

\renewcommand{\geq}{\geqslant}
\renewcommand{\leq}{\leqslant}

\newcommand{\todo}{\textbf{ !TODO! }}
\newcommand{\oldpage}[1]{\marginpar{\footnotesize$\Big\vert$ \textit{p.~#1}}}


%% Document %%

\usepackage{embedall}
\begin{document}

\renewcommand{\abstractname}{Translator's note.}

\title{The topological tensor product of topological vector spaces}
\author{L. Schwartz}
\date{18\textsuperscript{th} of November, 1953}
\maketitle

\begin{abstract}
  \renewcommand*{\thefootnote}{\fnsymbol{footnote}}
  \emph{This text is one of a series\footnote{\url{https://github.com/thosgood/translations}} of translations of various papers into English.}
  \emph{The translator takes full responsibility for any errors introduced in the passage from one language to another, and claims no rights to any of the mathematical content herein.}
  
  \emph{What follows is a translation (last updated \today) of the French paper:}

  \medskip\noindent
  \textsc{Schwartz, L.}
  ``Produit tensoriel topologique d'espaces vectoriels topologiques''.
  \emph{S\'{e}minaire Schwartz}, Volume~\textbf{1} (1953-1954), Talk no.~1, 3~p.
  {\footnotesize\url{http://www.numdam.org/item/SLS_1953-1954__1__A2_0/}}
\end{abstract}

\setcounter{footnote}{0}


%% Content %%

\oldpage{1}
\begin{theorem}
  If $E$ and $F$ are locally-convex spaces, then there exists exactly one locally-convex topology on $E\otimes F$ that satisfies the following property:
  for any locally-convex space $G$, the canonical isomorphism between the vector space of bilinear maps from $E\times F$ to $G$ and the vector space of linear maps from $E\otimes F$ to $G$ sends the vector space $\BB(E,F;G)$ of continuous bilinear maps from $E\times F$ to $G$ to the vector space $\LL(E\otimes F;G)$ of continuous linear maps from $E\otimes F$ to $G$.
  This isomorphism then sends equicontinuous sets of bilinear maps from $E\times F$ to $G$ to equicontinuous sets of linear maps from $E\otimes F$ to $G$.
  The topology thus defined on $E\otimes F$ is the finest locally-convex topology for which the canonical bilinear map from $E\times F$ to $E\otimes F$ is continuous.
\end{theorem}

\begin{proof}
  Let $T$ be a topology that satisfies the conditions stated in the theorem.
  The identity map from $(E\otimes F)_T$ to $(E\otimes F)_T$ is continuous, and so the canonical bilinear map from $E\times F$ to $(E\otimes F)_T$ is continuous.
  Now, if $T'$ is a locally-convex topology on $E\otimes F$ such that the canonical map from $E\times F$ to $(E\otimes F)_T$ is continuous, then the identity map from $(E\otimes F)_T$ to $(E\otimes F)_T$ is continuous, and so $T$ is finer than $T'$.
  We have thus shown that $T$, if it exists, is the finest locally-convex topology for which the canonical bilinear map from $E\times F$ to $E\otimes F$ is continuous;
  and this also proves the uniqueness of $T$.

  Now we show the existence.
\oldpage{2}
  For this, we will prove
  \begin{proposition}{1}
    If $U$ (resp. $V$) runs over a fundamental system of neighbourhoods of $0$ in $E$ (resp. $F$), then $\Gamma(U\otimes V)$, the balanced convex hull of $U\otimes V$, runs over a fundamental system of neighbourhoods of $0$ in a topology that satisfies the conditions stated in the theorem.
  \end{proposition}

  \begin{proof}
    Since $\Gamma(U\otimes V)$ is convex, balanced, and absorbing, and since these sets form a filter base, they indeed define a fundamental system of neighbourhoods of $0$ in a locally-convex topology $T$ on $E\otimes F$.

    Let $H$ be an equicontinuous subset of $\BB(E,F;G)$.
    Then there exists, for every balanced convex neighbourhood $W$ of $0$ in $G$, a neighbourhood $U$ (resp. $V$) of $0$ in $E$ (resp. $F$) such that, for all $B\in H$, we have that $B(U\times V)\subset W$.
    If we denote by $\widetilde{B}$ the linear map from $E\otimes F$ to $G$ associated to $B$, then we see that, for all $\widetilde{B}\in\widetilde{H}$, we have $\widetilde{B}(U\otimes V)\subset W$, and so, since $W$ is balanced and convex, $\widetilde{B}(\Gamma(U\otimes V))\subset W$;
    thus $\widetilde{H}$ is an equicontinuous subset of $\LL((E\otimes F)_T;G)$.

    Conversely, let $\widetilde{H}$ be an equicontinuous subset of $\LL((E\otimes F)_T;G)$.
    For any neighbourhood $W$ of $0$ in $G$, there then exists a neighbourhood of $0$ in $(E\otimes F)_T$, which we can assume to be of the form $\Gamma(U\otimes V)$, such that $\widetilde{B}(\Gamma(U\otimes V))\subset W$ for all $\widetilde{B}\in\widetilde{H}$.
    Then, a fortiori, $\widetilde{B}(U\otimes V)\subset W$, and so $B(U\times V)\subset W$, which proves that $H$ is an equicontinuous subset of $\BB(E,F;G)$.
  \end{proof}
  This finishes the proof.
\end{proof}

\begin{definition}
  The topology of $E\otimes F$ satisfying the conditions stated in the theorem will be called the \emph{projective tensor product of the topologies of $E$ and $F$}, or the \emph{projective tensor product on $E\otimes F$}, and will be denoted $(E\otimes F)_\pi$.

  $E\otimes F$ will always be, in this expos\'{e}, endowed with this topology.

  $E\cotimes F$ will denote the completion of $E\otimes F$ in this topology.
  Then, if $G$ is a complete locally-convex vector space, every continuous bilinear map from $E\times F$ to $G$ canonically defines a continuous linear map from $E\cotimes F$ to $G$.
\end{definition}

\begin{remark}
  The dual of $E\otimes F$ can be identified with the space $\BB(E,F)$ of continuous bilinear forms on $E\times F$ by means of the map $\widetilde{B}\mapsto B$;
  the same is true for the dual of $E\cotimes F$.
  Since the equicontinuous subsets of $(E\otimes F)$ are then identified with the equicontinuous subsets of $\BB(E,F)$, we see that a fundamental system of neighbourhoods of $0$ in $E\otimes F$ (resp. $E\cotimes F$) consists of polars of equicontinuous subsets of $\BB(E,F)$, in the dual system of $E\otimes F$ (resp. $E\cotimes F$) and $\BB(E,F)$.
\end{remark}

\oldpage{3}
This could also have been used to prove the uniqueness of the tensor product topology, as well as its existence.

\begin{proposition}{2}
\label{proposition2}
  If $E$ and $F$ are separated (resp. metrisable), then $E\otimes F$ is separated (resp. metrisable).
\end{proposition}

\begin{proof}
  Let $u\neq0\in E\otimes F$.
  If $E$ and $F$ are separated, then there exists a continuous bilinear form $B$ on $E\times F$ such that the associated linear form $\widetilde{B}$ on $E\otimes F$ satisfies $\widetilde{B}(u)\neq0$.
  [Indeed, there exist subspaces $E_1$ and $F_1$ of finite dimension such that $u\in E_1\otimes F_1$;
  let $E_2$ and $F_2$ be the topological complements of $E_1$ and $F_1$.
  If $B_1$ is a bilinear form on $E_1\otimes F_1$ such that the associated linear form $\widetilde{B_1}$ satisfies $\widetilde{B_1}(u)\neq0$, then we can take $B$ to be the bilinear form that is equal to $B_1$ on $E_1\times F_1$ and to $U$ on $E_1\times F_2$, $E_2\times F_1$, and $E_2\times F_2$].
  Since $\widetilde{B}$ must be continuous, this proves that $u$ is not an adherent point of $\{0\}$, and thus that $E\otimes F$ is separated.

  If $E$ and $F$ are metrisable, then we can take a countable fundamental system of neighbourhoods of $0$ in $E\otimes F$, the $\Gamma(U_\alpha\otimes V_\beta)$, where $U_\alpha$ (resp. $V_\beta$) runs over a countable fundamental system of neighbourhoods of $0$ in $E$ (resp. $F$);
  thus $E\otimes F$ is metrisable, and $E\cotimes F$ is a Fr\'{e}chet space.
\end{proof}


\end{document}
