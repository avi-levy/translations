\documentclass{article}

\title{The case of normed spaces. Tensor product of linear maps}
\author{L. Schwartz}
\date{25\textsuperscript{th} of November, 1953}

\usepackage{amssymb,amsmath}

\usepackage{hyperref}
\usepackage[nameinlink]{cleveref}
\usepackage{enumerate}
\usepackage{tikz-cd}

\usepackage{mathrsfs}
%% Fancy fonts --- feel free to remove! %%
\usepackage{Baskervaldx}
\usepackage{mathpazo}


\usepackage{fancyhdr}
\usepackage{lastpage}
\makeatletter
\ifx\pdfmdfivesum\undefined
  \let\pdfmdfivesum\mdfivesum
\fi
\edef\filesum{\pdfmdfivesum file {\jobname}}
\pagestyle{fancy}
\makeatletter
\let\runauthor\@author
\let\runtitle\@title
\makeatother
\fancyhf{}
\lhead{\footnotesize\runtitle}
\rhead{\footnotesize Version: \texttt{\filesum}}
\cfoot{\small\thepage\ of \pageref*{LastPage}}


\crefname{section}{Section}{Sections}
\crefname{equation}{}{}


%% Theorem environments %%

\usepackage{amsthm}

  \theoremstyle{plain}

  \newtheorem{innercustomproposition}{Proposition}
  \crefname{innercustomproposition}{Proposition}{Propositions}
  \newenvironment{proposition}[1]
    {\renewcommand\theinnercustomproposition{#1}\innercustomproposition}
    {\endinnercustomproposition}

  \newtheorem*{theorem}{Theorem}

  \theoremstyle{definition}

  \newtheorem*{remark}{Remark}
  \newtheorem*{definition}{Definition}


%% Shortcuts %%

\newcommand{\sh}{\mathscr}
\newcommand{\cat}{\mathcal}
\newcommand{\BB}{\mathcal{B}}
\newcommand{\LL}{\mathcal{L}}
\newcommand{\cotimes}{\widehat{\otimes}}

\renewcommand{\geq}{\geqslant}
\renewcommand{\leq}{\leqslant}

\newcommand{\todo}{\textbf{ !TODO! }}
\newcommand{\oldpage}[1]{\marginpar{\footnotesize$\Big\vert$ \textit{p.~#1}}}


%% Document %%

\usepackage{embedall}
\begin{document}

\maketitle
\thispagestyle{fancy}

\renewcommand{\abstractname}{Translator's note.}

\begin{abstract}
  \renewcommand*{\thefootnote}{\fnsymbol{footnote}}
  \emph{This text is one of a series\footnote{\url{https://github.com/thosgood/translations}} of translations of various papers into English.}
  \emph{The translator takes full responsibility for any errors introduced in the passage from one language to another, and claims no rights to any of the mathematical content herein.}
  
  \emph{What follows is a translation (last updated \today) of the French paper:}

  \medskip\noindent
  \textsc{Schwartz, L.}
  ``Cas des espaces norm\'{e}s. Produit tensoriel d'applications lin\'{e}aires''.
  \emph{S\'{e}minaire Schwartz}, Volume~\textbf{1} (1953-1954), Talk no.~2, pp.~3--7.
  {\footnotesize\url{http://www.numdam.org/item/SLS_1953-1954__1__A3_0/}}
\end{abstract}

\setcounter{footnote}{0}


%% Content %%

\oldpage{3}
\begin{proposition}{1}
  If $U$ (resp. $V$) is
\end{proposition}


\end{document}
