\documentclass{article}

\usepackage{amssymb,amsmath}

\usepackage{hyperref}
\usepackage[nameinlink]{cleveref}
\usepackage{enumerate}
\usepackage{tikz-cd}
\usepackage{graphicx}

\usepackage{mathrsfs}
%% Fancy fonts --- feel free to remove! %%
\usepackage{Baskervaldx}
\usepackage{mathpazo}


\crefname{section}{Section}{Sections}
\crefname{equation}{}{}


%% Theorem environments %%

\usepackage{amsthm}

  \theoremstyle{plain}

  \newtheorem{innercustomtheorem}{Theorem}
  \crefname{innercustomtheorem}{Theorem}{Theorems}
  \newenvironment{theorem}[1]
    {\renewcommand\theinnercustomtheorem{#1}\innercustomtheorem}
    {\endinnercustomtheorem}

  \newtheorem{innercustomproposition}{Proposition}
  \crefname{innercustomproposition}{Proposition}{Propositions}
  \newenvironment{proposition}[1]
    {\renewcommand\theinnercustomproposition{#1}\innercustomproposition}
    {\endinnercustomproposition}

  \newtheorem{innercustomlemma}{Lemma}
  \crefname{innercustomlemma}{Lemma}{Lemmas}
  \newenvironment{lemma}[1]
    {\renewcommand\theinnercustomlemma{#1}\innercustomlemma}
    {\endinnercustomlemma}

  \newtheorem{innercustomcorollary}{Corollary}
  \crefname{innercustomcorollary}{Corollary}{Corollaries}
  \newenvironment{corollary}[1]
    {\renewcommand\theinnercustomcorollary{#1}\innercustomcorollary}
    {\endinnercustomcorollary}


  \theoremstyle{definition}

  \newtheorem*{remark}{Remark}

  \newtheorem{innercustomdefinition}{Definition}
  \crefname{innercustomdefinition}{Definition}{Definitions}
  \newenvironment{definition}[1]
    {\renewcommand\theinnercustomdefinition{#1}\innercustomdefinition}
    {\endinnercustomdefinition}


%% Shortcuts %%

\newcommand{\sh}{\mathscr}
\newcommand{\cat}{\mathcal}

\renewcommand{\geq}{\geqslant}
\renewcommand{\leq}{\leqslant}

\newcommand{\todo}{\textbf{ !TODO! }}
\newcommand{\oldpage}[1]{\marginpar{\footnotesize$\Big\vert$ \textit{p.~#1}}}


%% Document %%

\usepackage{embedall}
\begin{document}

\renewcommand{\abstractname}{Translator's note.}

\title{On modifications and exceptional analytic sets}
\author{Hans Grauert}
\date{}
\maketitle

\begin{abstract}
  \renewcommand*{\thefootnote}{\fnsymbol{footnote}}
  \emph{This text is one of a series\footnote{\url{https://github.com/thosgood/translations}} of translations of various papers into English.}
  \emph{The translator takes full responsibility for any errors introduced in the passage from one language to another, and claims no rights to any of the mathematical content herein.}
  
  \emph{What follows is a translation (last updated \today) of the German paper:}

  \medskip\noindent
  \textsc{Grauert, H.}
  ``\"{U}ber Modifikationen und exzeptionelle analytische Mengen''.
  \emph{Math. Ann.}, Volume~\textbf{146} (1962), pp.~331--368.
  {\footnotesize\url{http://eudml.org/doc/160940}}
\end{abstract}

\setcounter{footnote}{0}

\tableofcontents


%% Content %%

\bigskip\bigskip
The term ``modification'' first appeared in a 1951 publication \cite{1} by H.~Behnke and K.~Stein.
The authors use it to refer to a process that allows a given complex space to be modified.
If $X$ is a complex space, and $N\subset X$ a low-dimensional analytic set, then $N$ is replaced by another set $N'$ such that the complex structure on $X\setminus N$ can be extended to the entire space $X'=(X\setminus N)\cup N'$.
The newly obtained complex space $X'$ is called a \emph{modification} of $X$.


%% Bibliography %%

\nocite{*}
\bibliographystyle{acm}

\end{document}
