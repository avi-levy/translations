\documentclass{article}

\usepackage{amssymb,amsmath}

\usepackage{hyperref}
\usepackage[nameinlink]{cleveref}
\usepackage{enumerate}

\usepackage{mathrsfs}
%% Fancy fonts --- feel free to remove! %%
\usepackage{Baskervaldx}
\usepackage{mathpazo}


\crefname{section}{Section}{Sections}
\crefname{equation}{}{}


%% Theorem environments %%

\usepackage{amsthm}

  \theoremstyle{plain}

  \newtheorem{innercustomproposition}{Proposition}
  \crefname{innercustomproposition}{Proposition}{Propositions}
  \newenvironment{proposition}[1]
    {\renewcommand\theinnercustomproposition{#1}\innercustomproposition}
    {\endinnercustomproposition}

  \newtheorem{innercustomlemma}{Lemma}
  \crefname{innercustomlemma}{Lemma}{Lemmas}
  \newenvironment{lemma}[1]
    {\renewcommand\theinnercustomlemma{#1}\innercustomlemma}
    {\endinnercustomlemma}


  \theoremstyle{definition}

  \newtheorem{innercustomdefinition}{Definition}
  \crefname{innercustomdefinition}{Definition}{Definitions}
  \newenvironment{definition}[1]
    {\renewcommand\theinnercustomdefinition{#1}\innercustomdefinition}
    {\endinnercustomdefinition}


%% Shortcuts %%

\newcommand{\sh}{\mathscr}
\newcommand{\cat}{\mathcal}

\renewcommand{\geq}{\geqslant}
\renewcommand{\leq}{\leqslant}

\newcommand{\todo}{\textbf{ !TODO! }}
\newcommand{\oldpage}[1]{\marginpar{\footnotesize$\Big\vert$ \textit{p.~#1}}}


%% Document %%

\usepackage{embedall}
\begin{document}

\renewcommand{\abstractname}{Translator's note.}

\title{On modifications and exceptional analytic sets}
\author{Hans Grauert}
\date{}
\maketitle

\begin{abstract}
  \renewcommand*{\thefootnote}{\fnsymbol{footnote}}
  \emph{This text is one of a series\footnote{\url{https://github.com/thosgood/translations}} of translations of various papers into English.}
  \emph{The translator takes full responsibility for any errors introduced in the passage from one language to another, and claims no rights to any of the mathematical content herein.}
  
  \emph{What follows is a translation (last updated \today) of the German paper:}

  \medskip\noindent
  \textsc{Grauert, H.}
  ``\"{U}ber Modifikationen und exzeptionelle analytische Mengen''.
  \emph{Math. Ann.}, Volume~\textbf{146} (1962), pp.~331--368.
  {\footnotesize\url{http://eudml.org/doc/160940}}
\end{abstract}

\setcounter{footnote}{0}

\tableofcontents


%% Content %%

\bigskip\bigskip
The term ``modification'' first appeared in a 1951 publication \cite{1} by H.~Behnke and K.~Stein.
The authors use it to refer to a process that allows a given complex space to be modified.
If $X$ is a complex space, and $N\subset X$ a low-dimensional analytic set, then $N$ is replaced by another set $N'$ such that the complex structure on $X\setminus N$ can be extended to the entire space $X'=(X\setminus N)\cup N'$.
The newly obtained complex space $X'$ is called a \emph{modification} of $X$.


%% Bibliography %%

\nocite{*}
\bibliographystyle{acm}

\begin{thebibliography}{20}

  \bibitem{1}
  {\sc Behnke, H. and Stein, K.}
  \newblock Modifikationen komplexer Mannigfaltigkeiten und Riemannscher Gebiete.
  \newblock {\em Math. Ann.} {\bf 124} (1951), pp.~1--16.

  \bibitem{2}
  {\sc Cartan, H.}
  \newblock ``Quotients of complex spaces''.
  \newblock {\em Contributions to Function Theory}, pp.~1--16.
  \newblock Tata Inst. Fund. Res. Bombay 1960.

  \bibitem{3}
  {\sc Cartan, H. and Eilenberg, S.}
  \newblock ``Homological Algebra''.
  \newblock Princeton University Press 1956.

  \bibitem{4}
  {\sc Chow, W.L. and Kodaira, K.}
  \newblock On analytic surfaces with two independent meromorphic functions.
  \newblock {\em Proc. Nat. Acad. Sci. U.S.A.} {\bf 38} (1952), pp.~319--325.

  \bibitem{5}
  {\sc Frenkel, J.}
  \newblock Cohomologie non ab\'{e}linne et espaces fib\'{e}s.
  \newblock {\em Bull. soc. math. France} {\bf 85} (1957), pp.~135--218.

  \bibitem{6}
  {\sc Grauert, H.}
  \newblock On Levi's problem and the imbedding of real-analytic manifolds.
  \newblock {\em Ann. Math.} {\bf 68} (1958), pp.~460--472.

  \bibitem{7}
  {\sc Grauert, H.}
  \newblock ``On point modifications''.
  \newblock {\em Contributions to Function Theory}, pp.~139--142.
  \newblock Tata Inst. Fund. Res. Bombay 1960.

  \bibitem{8}
  {\sc Grauert, H.}
  \newblock Ein Theorem der analytischen Garbentheorie und die Modulr\"{a}ume komplexer Strukturen.
  \newblock {\em Pub. Math.} {\bf 5} (1960), pp.~233--292.

  \bibitem{9}
  {\sc Grauert, H. and Remmert, R.}
  \newblock Plurisubharmonische Funktionen in komplexen R\"{a}umen.
  \newblock {\em Math. Z.} {\bf 65} (1956), pp.~175--194.

  \bibitem{10}
  {\sc Grauert, H. and Remmert, R.}
  \newblock Komplexe R\"{a}ume.
  \newblock {\em Math. Ann.} {\bf 136} (1958), pp.~245--318.

  \bibitem{11}
  {\sc Hirzebruch, F.}
  \newblock Some problems on differentiable and complex manifolds.
  \newblock {\em Ann. Math.} {\bf 60} (1954), pp.~213--236.

  \bibitem{12}
  {\sc Hopf, H.}
  \newblock Schlichte Abbildungen und lokale Modifikation 4-dimensionaler komplexer Mannigfaltigkeiten.
  \newblock {\em Comment. Math. Helv.} {\bf 29} (1955), pp.~132--156.

  \bibitem{13}
  {\sc Kodaira, K.}
  \newblock On K\"{a}hler Varieties of restricted type.
  \newblock {\em Ann. Math.} {\bf 60} (1954), pp.28--48.

  \bibitem{14}
  {\sc Koopman, B.O. and Brown, A.B.}
  \newblock On the covering of analytic loci by complexes.
  \newblock {\em Trans. Am. Math. Soc.} {\bf 34} (1931), pp.~231--251.

  \bibitem{15}
  {\sc Nakano, S.}
  \newblock On complex analytic vector bundles.
  \newblock {\em J. Math. Soc. Japan} {\bf 7} (1955), pp.~1--12.

  \bibitem{16}
  {\sc Narasimhan, R.}
  \newblock The Levi problem for complex spaces.
  \newblock {\em Math. Ann.} {\bf 142} (1961), pp.~355-365.

  \bibitem{17}
  {\sc Remmert, R.}
  \newblock Sur les espaces analytiques holomorphiquement s\'{e}perables et holomorphiquement convexes.
  \newblock {\em C. R. Acad. Sci. (Paris)} {\bf 243} (1956), pp.~118--121.

  \bibitem{18}
  {\sc Remmert, R. and Stein, K.}
  \newblock \"{U}ber die wesentlichen Singularit\"{a}ten analytischer Mengen.
  \newblock {\em Math. Ann.} {\bf 126} (1953), pp.~263--306.

  \bibitem{19}
  {\sc Weil, A.}
  \newblock ``Introduction \`{a} l'\'{e}tude des vari\'{e}t\'{e}s kaehleri\'{e}nnes''.
  \newblock Paris: Hermann 1958.

\end{thebibliography}

\end{document}
