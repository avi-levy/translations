\documentclass{article}

\usepackage{amssymb,amsmath}

\usepackage{hyperref}
\usepackage[nameinlink]{cleveref}
\usepackage{enumerate}
\usepackage{tikz-cd}
\usepackage{graphicx}

\usepackage{mathrsfs}
%% Fancy fonts --- feel free to remove! %%
\usepackage{Baskervaldx}
\usepackage{mathpazo}
\usepackage{esdiff}

\crefname{section}{Section}{Sections}
\crefname{equation}{}{}


%% Shortcuts %%

\newcommand{\HH}{\mathscr{H}}
\newcommand{\LL}{\mathscr{L}}

\renewcommand{\d}{\,\mathrm{d}}
\renewcommand{\geq}{\geqslant}
\renewcommand{\leq}{\leqslant}

\newcommand{\todo}{\textbf{ !TODO! }}
\newcommand{\oldpage}[1]{\marginpar{\footnotesize$\Big\vert$ \textit{p.~#1}}}


%% Document %%

\usepackage{embedall}
\begin{document}

\renewcommand{\abstractname}{Translator's note.}

\title{Tensor fields}
\author{K.H. Tzou}
\date{8\textsuperscript{th} of February, 1955}
\maketitle

\begin{abstract}
  \renewcommand*{\thefootnote}{\fnsymbol{footnote}}
  \emph{This text is one of a series\footnote{\url{https://github.com/thosgood/translations}} of translations of various papers into English.}
  \emph{What follows is a translation (last updated \today) of the French paper:}

  \medskip\noindent
  \textsc{Tzou, K.H.} Les champs tensoriels. \emph{Séminaire L. de Broglie. Théories physiques}, Volume~\textbf{24} (1954-1955), Talk no.~10, 10~p. {\footnotesize\url{http://www.numdam.org/item/SLDB_1954-1955__24__A9_0/}}
\end{abstract}

\tableofcontents


%% Content %%

\bigskip\bigskip
\oldpage{13-01}
The theory of elementary particles consists of associating, to each type of particle, a field.
The field is chosen according to the spin of the particle.
According to the means of description, the fields can be classified into two categories.
Firstly, there are the \emph{spinorial fields}, whose variables are spinors ($\psi_{\alpha\beta\ldots}$) that satisfy first-order differential equations of the following type:
\[
  \left(\gamma_\lambda \diffp{}{{x_\lambda}} + \HH\right)\psi = 0
\]
where $\gamma_\lambda$ is four matrices belonging to a certain non-commutative algebra [L.~de~Broglie, N.~Kemmer].
With spinorial fields, we can account for half-integer-spin particles as well as integer-spin particles.
A spinor with a single index $(\psi_\alpha)$ represents a field of spin~$\frac12$.
A spinorial field with two indices $(\psi_{\alpha\beta})$ does not correspond to a specific well-defined spin, but it can be reduced to a field of total spin~$1$ and a field of spin~$0$ with a zero component.
It is thus a field of maximum total spin~$1$.

In the particular case of integer spin, the description of fields can be given, in a different way, by means of tensors $(\Psi_{\lambda,\rho})$ satisfying second-order differential equations [A.~Proca, N.~Kemmer, M. Fierz].
Since the fundamental second-order differential operator is $\square-\HH^2$, the field equations are equations of Dalembertien type:
\[
  \Big(\square-\HH^2\Big)\Psi = 0.
\]
If $\Psi$ is a scalar, then the field represents particles of spin~$0$.
A vector field does not correspond to a well-defined spin.
If we also require it to satisfy the condition of zero divergence, then it will be a field of total spin~$1$.
If not, it has four independent components, and we can prove that it represents a superposition of four states of well-defined spin: three states of spin~$1$, and one of spin~$0$.
These spin states can be separated in an explicit way (for $\HH\neq0$).
The vector field is thus also, like the spinorial field with two indices,
\oldpage{13-02}
a field of maximum total spin~$1$.

In general, a tensor field without additional conditions does not represent a well-defined spin, but a mixture of certain well-defined spins.
Each non-zero component of the tensor corresponds to a certain spin states.
The spin states can be decomposed in an explicit way (for $\HH\neq0$).
Also, the field does not have a positive defined total energy.
But the decomposition of the spin states is also the covariant decomposition of the positive and negative parts of the total energy.


\section{Free fields}
\label{section1}

The Lagrangian of a tensor field $(\Psi_{\alpha_1\ldots\alpha_n})$ of rank~$n$ is
\[
  \LL = -\frac12\left(
    \diffp{\Psi}{{x_\lambda}} \diffp{\Psi}{{x_\lambda}} + \HH^2\Psi\Psi
  \right).
\]
The stress-energy tensor of the field can then be written as
\[
  T_{\mu\nu} = \frac12\left[
    \diffp{\Psi}{{x_\mu}} \diffp{\Psi}{{x_\nu}}
    + \diffp{\Psi}{{x_\mu}} \diffp{\Psi}{{x_\nu}}
    -\delta_{\mu\nu}\left(
      \diffp{\Psi}{{x_\lambda}} \diffp{\Psi}{{x_\lambda}} + \HH^2\Psi\Psi
    \right)
  \right],
\]
and the total stress-energy as
\[
  P_\nu = \frac1c \int_\sigma T_{\mu\nu}\d\sigma_\mu.
\]
By the general principle of variation of the Lagrangian formalism, the total angular momentum of the field is composed of two parts: $M_{\mu\nu}+S_{\mu\nu}$, where $M_{\mu\nu}$ is the orbital angular momentum
\[
  M_{\mu\nu} = \frac1c \int_\sigma (x_\mu T_{\lambda\nu} - x_\nu T_{\lambda\mu}) \d\sigma_\lambda
\]
and $S_{\mu\nu}$ is the spin of the field
\[
  S_{\mu\nu} = \frac1c \int_\sigma S_{\lambda\mu\nu} \d\sigma_\lambda
\]
where
\begin{align*}
  S_{\lambda\mu\nu} = \sum_{i=1}^n \Bigg[
    &\diffp{{\Psi_{\alpha_1\ldots\alpha_{i-1}\alpha_\mu\alpha_{i+1}\ldots\alpha_n}}}{{x_\lambda}} \Psi_{\alpha_1\ldots\alpha_{i-1}\alpha_\nu\alpha_{i+1}\ldots\alpha_n}
    \\-&
    \diffp{{\Psi_{\alpha_1\ldots\alpha_{i-1}\alpha_\nu\alpha_{i+1}\ldots\alpha_n}}}{{x_\lambda}} \Psi_{\alpha_1\ldots\alpha_{i-1}\alpha_\mu\alpha_{i+1}\ldots\alpha_n}
  \Bigg]
\end{align*}

\oldpage{13-03}
First we will study a vector field or \todo $(A_\lambda)$ to show how the spin states can be explicitly decomposed.
In this case,
\[
  \Big(\square-\HH^2\Big)A_\lambda = 0.
\]
Then
\begin{gather*}
  \HH^2 A_\lambda = \diffp{{F_{\rho\lambda}}}{{x_\rho}} + \diffp{X}{{x_\lambda}}
\\\mbox{where $F_{\lambda\rho} = \diffp{{A_\rho}}{{x_\lambda}} - \diffp{{A_\lambda}}{{x_\rho}}$ and $X = \diffp{{A_\lambda}}{{x_\lambda}}$.}
\end{gather*}
We can decompose $A_\lambda$ into two component fields: $A_\lambda=B_\lambda+C_\lambda$, so that $B_\lambda$ determines $F_{\lambda\rho}$, and $C_\lambda$ determines $X$, in a separated independent way.
To do this, we need
\[
  \diffp{{B_\lambda}}{{x_\lambda}} = 0,
  \quad
  \diffp{{C_\rho}}{{x_\lambda}} - \diffp{{C_\lambda}}{{x_\rho}} = 0.
\]
But the zero rotation reduces $C_\lambda$ to a scalar: $C_\lambda=\HH^{-1}\diffp{\Sigma}{{x_\lambda}}$.
Finally,
\[
\label{equation1}
  \begin{gathered}
    A_\lambda = B_\lambda + \HH^{-1}\diffp{\Sigma}{{x_\lambda}},
    \\\diffp{{B_\lambda}}{{x_\lambda}} = 0.
  \end{gathered}
  \tag{1}
\]
Then $F_{\lambda\rho}=\diffp{{B_\rho}}{{x_\lambda}}-\diffp{{B_\lambda}}{{x_\rho}}$ and $X=\HH\Sigma$.
So $B_\lambda$ determines $F_{\lambda\rho}$, and $\Sigma$ determines $X$ separately.
With \cref{equation1}, $\LL$, $P_\nu$, and $M_{\mu\nu}+S_{\mu\nu}$ all decompose into two independent parts, one of which corresponds to $B_\lambda$ and the other to $\Sigma$.
But $S_{\mu\nu}^{(\Sigma)}=0$, $P_0^{(\Sigma)}<0$, and $P_0^{(\Sigma)}>0$.
The field $A_\lambda$ thus decomposes explicitly in terms of $B_\lambda$ and $\Sigma$, with $B_\lambda$ a vector field of zero divergence, and thus of total spin~$1$, and $\Sigma$ a spinorial field of spin~$0$.

\medskip\hrulefill\bigskip

This decomposition, thanks to the zero divergence, can be generalised to a tensor field $(\Omega_{\alpha\beta})$ of rank~$2$ without any additional conditions, by applying the procedure in \cref{equation1} to each of the two indices.
The schema of decomposition is then
\begin{gather*}
  \Omega_{\alpha\beta} = \omega_{\alpha\beta} + \HH^{-1}\left(
    \diffp{{B_\beta}}{{x_\alpha}} + \diffp{{C_\alpha}}{{x_\beta}}
  \right) + \HH^{-2}\diffp{\Sigma}{{{x_\alpha}}{{x_\beta}}},
\\\diffp{{\omega_{\alpha\beta}}}{{x_\alpha}} = \diffp{{\omega_{\beta\alpha}}}{{x_\alpha}} = 0,
\\\diffp{{B_\alpha}}{{x_\alpha}} = \diffp{{C_\alpha}}{{x_\alpha}} = 0.
\end{gather*}
But $\omega_{\alpha\beta}$ can be decomposed into a symmetric tensor $\theta_{\alpha\beta}$ and an antisymmetric tensor $\varphi_{\alpha\beta}$.
Then
\oldpage{13-04}
\begin{gather*}
  \Omega_{\alpha\beta} = \theta_{\alpha\beta} + \varphi_{\alpha\beta} + \HH^{-1}\left(
    \diffp{{B_\beta}}{{x_\alpha}} + \diffp{{C_\alpha}}{{x_\beta}}
  \right) + \HH^{-2}\diffp{\Sigma}{{x_\alpha}{x_\beta}},
\\\diffp{{\theta_{\alpha\beta}}}{{x_\alpha}} = 0,
  \quad \diffp{{\varphi_{\alpha\beta}}}{{x_\alpha}} = 0,
\\\diffp{{B_\alpha}}{{x_\alpha}} = 0,
  \quad \diffp{{C_\alpha}}{{x_\alpha}} = 0.
\end{gather*}
If $\Phi_{\alpha\beta}$ is an antisymmetric field of rank~$2$, then, without any additional conditions, the decomposition can evidently follow the schema
\[
\label{equation2}
  \Phi_{\alpha\beta} = \varphi_{\alpha\beta} + \HH^{-1}\left(
    \diffp{{B_\beta}}{{x_\alpha}} - \diffp{{B_\alpha}}{{x_\beta}}
  \right).
  \tag{2}
\]
For a symmetric field $\Theta_{\alpha\beta}$ of rank~$2$, the decomposition schema is
\[
  \Theta_{\alpha\beta} = \theta_{\alpha\beta} + \HH^{-1}\left(
    \diffp{{B_\beta}}{{x_\alpha}} - \diffp{{B_\alpha}}{{x_\beta}}
  \right) + \HH^{-2} \diffp{\Sigma}{{x_\alpha}{x_\beta}}.
\]
The composite fields $\Sigma$, $\varphi_{\alpha\beta}$, and $\theta_{\alpha\beta}$ all have a positive defined total energy, and $B_\alpha$ and $C_\alpha$ have a negative total energy.
The positive and negative parts of the total energy and thus completely separated in a covariant way, thanks to the zero divergence.
$\Sigma$ is a spinorial field; $B_\alpha$ and $C_\alpha$ are vector fields with zero divergence, and thus both of spin~$1$.
The antisymmetric field $\varphi_{\alpha\beta}$ of zero divergence has only three independent components.
It is equivalent to a \todo field of zero divergence, and it thus also a field of spin~$1$.
$\theta_{\alpha\beta}$, having six independent components, does not represent a specific well-defined spin.

\medskip\hrulefill\bigskip

The field


\end{document}
