\documentclass{article}

\usepackage{amssymb,amsmath}

\usepackage{hyperref}
\usepackage[nameinlink]{cleveref}
\usepackage{enumerate}
\usepackage{tikz-cd}
\usepackage{graphicx}

\usepackage{mathrsfs}
%% Fancy fonts --- feel free to remove! %%
\usepackage{Baskervaldx}
\usepackage{mathpazo}


\crefname{equation}{}{}
\renewcommand{\thesection}{\Roman{section}}


%% Shortcuts %%

\usepackage{esdiff}

\newcommand{\HH}{\mathscr{H}}
\newcommand{\LL}{\mathscr{L}}

\renewcommand{\d}{\,\mathrm{d}}
\renewcommand{\geq}{\geqslant}
\renewcommand{\leq}{\leqslant}

\newcommand{\todo}{\textbf{ !TODO! }}
\newcommand{\oldpage}[1]{\marginpar{\footnotesize$\Big\vert$ \textit{p.~#1}}}


%% Document %%

\usepackage{embedall}
\begin{document}

\renewcommand{\abstractname}{Translator's note.}

\title{Tensor fields}
\author{K.H. Tzou}
\date{8\textsuperscript{th} of February, 1955}
\maketitle

\begin{abstract}
  \renewcommand*{\thefootnote}{\fnsymbol{footnote}}
  \emph{This text is one of a series\footnote{\url{https://github.com/thosgood/translations}} of translations of various papers into English.}
  \emph{What follows is a translation (last updated \today) of the French paper:}

  \medskip\noindent
  \textsc{Tzou, K.H.} Les champs tensoriels. \emph{Séminaire L. de Broglie. Théories physiques}, Volume~\textbf{24} (1954-1955), Talk no.~10, 10~p. {\footnotesize\url{http://www.numdam.org/item/SLDB_1954-1955__24__A9_0/}}
\end{abstract}

\tableofcontents


%% Content %%

\bigskip\bigskip
\oldpage{13-01}
The theory of elementary particles consists of associating, to each type of particle, a field.
The field is chosen according to the spin of the particle.
According to the means of description, the fields can be classified into two categories.
Firstly, there are the \emph{spinor fields}, whose variables are spinors ($\psi_{\alpha\beta\ldots}$) that satisfy first-order differential equations of the following type:
\[
  \left(\gamma_\lambda \diffp{}{{x_\lambda}} + \HH\right)\psi = 0
\]
where $\gamma_\lambda$ is four matrices belonging to a certain non-commutative algebra [L.~de~Broglie, N.~Kemmer].
With spinor fields, we can account for half-integer-spin particles as well as integer-spin particles.
A spinor with a single index $(\psi_\alpha)$ represents a field of spin~$\frac12$.
A spinor field with two indices $(\psi_{\alpha\beta})$ does not correspond to a specific well-defined spin, but it can be reduced to a field of total spin~$1$ and a field of spin~$0$ with a zero component.
It is thus a field of maximum total spin~$1$.

In the particular case of integer spin, the description of fields can be given, in a different way, by means of tensors $(\Psi_{\lambda,\rho})$ satisfying second-order differential equations [A.~Proca, N.~Kemmer, M. Fierz].
Since the fundamental second-order differential operator is $\square-\HH^2$, the field equations are equations of Dalembertien type:
\[
  \Big(\square-\HH^2\Big)\Psi = 0.
\]
If $\Psi$ is a scalar, then the field represents particles of spin~$0$.
A vector field does not correspond to a well-defined spin.
If we also require it to satisfy the condition of zero divergence, then it will be a field of total spin~$1$.
If not, it has four independent components, and we can prove that it represents a superposition of four states of well-defined spin: three states of spin~$1$, and one of spin~$0$.
These spin states can be separated in an explicit way (for $\HH\neq0$).
The vector field is thus also, like the spinor field with two indices,
\oldpage{13-02}
a field of maximum total spin~$1$.

In general, a tensor field without additional conditions does not represent a well-defined spin, but a mixture of certain well-defined spins.
Each non-zero component of the tensor corresponds to a certain spin states.
The spin states can be decomposed in an explicit way (for $\HH\neq0$).
Also, the field does not have a positive defined total energy.
But the decomposition of the spin states is also the covariant decomposition of the positive and negative parts of the total energy.


\section{Free fields}
\label{section1}

The Lagrangian of a tensor field $(\Psi_{\alpha_1\ldots\alpha_n})$ of rank~$n$ is
\[
  \LL = -\frac12\left(
    \diffp{\Psi}{{x_\lambda}} \diffp{\Psi}{{x_\lambda}} + \HH^2\Psi\Psi
  \right).
\]
The stress-energy tensor of the field can then be written as
\[
  T_{\mu\nu} = \frac12\left[
    \diffp{\Psi}{{x_\mu}} \diffp{\Psi}{{x_\nu}}
    + \diffp{\Psi}{{x_\mu}} \diffp{\Psi}{{x_\nu}}
    -\delta_{\mu\nu}\left(
      \diffp{\Psi}{{x_\lambda}} \diffp{\Psi}{{x_\lambda}} + \HH^2\Psi\Psi
    \right)
  \right],
\]
and the total stress-energy as
\[
  P_\nu = \frac1c \int_\sigma T_{\mu\nu}\d\sigma_\mu.
\]
By the general principle of variation of the Lagrangian formalism, the total angular momentum of the field is composed of two parts: $M_{\mu\nu}+S_{\mu\nu}$, where $M_{\mu\nu}$ is the orbital angular momentum
\[
  M_{\mu\nu} = \frac1c \int_\sigma (x_\mu T_{\lambda\nu} - x_\nu T_{\lambda\mu}) \d\sigma_\lambda
\]
and $S_{\mu\nu}$ is the spin of the field
\[
  S_{\mu\nu} = \frac1c \int_\sigma S_{\lambda\mu\nu} \d\sigma_\lambda
\]
where
\begin{align*}
  S_{\lambda\mu\nu} = \sum_{i=1}^n \Bigg[
    &\diffp{{\Psi_{\alpha_1\ldots\alpha_{i-1}\alpha_\mu\alpha_{i+1}\ldots\alpha_n}}}{{x_\lambda}} \Psi_{\alpha_1\ldots\alpha_{i-1}\alpha_\nu\alpha_{i+1}\ldots\alpha_n}
    \\-&
    \diffp{{\Psi_{\alpha_1\ldots\alpha_{i-1}\alpha_\nu\alpha_{i+1}\ldots\alpha_n}}}{{x_\lambda}} \Psi_{\alpha_1\ldots\alpha_{i-1}\alpha_\mu\alpha_{i+1}\ldots\alpha_n}
  \Bigg]
\end{align*}

\oldpage{13-03}
First we will study a vector field or \todo $(A_\lambda)$ to show how the spin states can be explicitly decomposed.
In this case,
\[
  \Big(\square-\HH^2\Big)A_\lambda = 0.
\]
Then
\begin{gather*}
  \HH^2 A_\lambda = \diffp{{F_{\rho\lambda}}}{{x_\rho}} + \diffp{X}{{x_\lambda}}
\\\mbox{where $F_{\lambda\rho} = \diffp{{A_\rho}}{{x_\lambda}} - \diffp{{A_\lambda}}{{x_\rho}}$ and $X = \diffp{{A_\lambda}}{{x_\lambda}}$.}
\end{gather*}
We can decompose $A_\lambda$ into two component fields: $A_\lambda=B_\lambda+C_\lambda$, so that $B_\lambda$ determines $F_{\lambda\rho}$, and $C_\lambda$ determines $X$, in a separated independent way.
To do this, we need
\[
  \diffp{{B_\lambda}}{{x_\lambda}} = 0,
  \quad
  \diffp{{C_\rho}}{{x_\lambda}} - \diffp{{C_\lambda}}{{x_\rho}} = 0.
\]
But the zero rotation reduces $C_\lambda$ to a scalar: $C_\lambda=\HH^{-1}\diffp{\Sigma}{{x_\lambda}}$.
Finally,
\[
\label{equation1}
  \begin{gathered}
    A_\lambda = B_\lambda + \HH^{-1}\diffp{\Sigma}{{x_\lambda}},
    \\\diffp{{B_\lambda}}{{x_\lambda}} = 0.
  \end{gathered}
  \tag{1}
\]
Then $F_{\lambda\rho}=\diffp{{B_\rho}}{{x_\lambda}}-\diffp{{B_\lambda}}{{x_\rho}}$ and $X=\HH\Sigma$.
So $B_\lambda$ determines $F_{\lambda\rho}$, and $\Sigma$ determines $X$ separately.
With \cref{equation1}, $\LL$, $P_\nu$, and $M_{\mu\nu}+S_{\mu\nu}$ all decompose into two independent parts, one of which corresponds to $B_\lambda$ and the other to $\Sigma$.
But $S_{\mu\nu}^{(\Sigma)}=0$, $P_0^{(\Sigma)}<0$, and $P_0^{(\Sigma)}>0$.
The field $A_\lambda$ thus decomposes explicitly in terms of $B_\lambda$ and $\Sigma$, with $B_\lambda$ a vector field of zero divergence, and thus of total spin~$1$, and $\Sigma$ a spinor field of spin~$0$.

\medskip\hrulefill\bigskip

This decomposition, thanks to the zero divergence, can be generalised to a tensor field $(\Omega_{\alpha\beta})$ of rank~$2$ without any additional conditions, by applying the procedure in \cref{equation1} to each of the two indices.
The schema of decomposition is then
\begin{gather*}
  \Omega_{\alpha\beta} = \omega_{\alpha\beta} + \HH^{-1}\left(
    \diffp{{B_\beta}}{{x_\alpha}} + \diffp{{C_\alpha}}{{x_\beta}}
  \right) + \HH^{-2}\diffp{\Sigma}{{{x_\alpha}}{{x_\beta}}},
\\\diffp{{\omega_{\alpha\beta}}}{{x_\alpha}} = \diffp{{\omega_{\beta\alpha}}}{{x_\alpha}} = 0,
\\\diffp{{B_\alpha}}{{x_\alpha}} = \diffp{{C_\alpha}}{{x_\alpha}} = 0.
\end{gather*}
But $\omega_{\alpha\beta}$ can be decomposed into a symmetric tensor $\theta_{\alpha\beta}$ and an antisymmetric tensor $\varphi_{\alpha\beta}$.
Then
\oldpage{13-04}
\begin{gather*}
  \Omega_{\alpha\beta} = \theta_{\alpha\beta} + \varphi_{\alpha\beta} + \HH^{-1}\left(
    \diffp{{B_\beta}}{{x_\alpha}} + \diffp{{C_\alpha}}{{x_\beta}}
  \right) + \HH^{-2}\diffp{\Sigma}{{x_\alpha}{x_\beta}},
\\\diffp{{\theta_{\alpha\beta}}}{{x_\alpha}} = 0,
  \quad \diffp{{\varphi_{\alpha\beta}}}{{x_\alpha}} = 0,
\\\diffp{{B_\alpha}}{{x_\alpha}} = 0,
  \quad \diffp{{C_\alpha}}{{x_\alpha}} = 0.
\end{gather*}
If $\Phi_{\alpha\beta}$ is an antisymmetric field of rank~$2$, then, without any additional conditions, the decomposition can evidently follow the schema
\[
\label{equation2}
  \Phi_{\alpha\beta} = \varphi_{\alpha\beta} + \HH^{-1}\left(
    \diffp{{B_\beta}}{{x_\alpha}} - \diffp{{B_\alpha}}{{x_\beta}}
  \right).
  \tag{2}
\]
For a symmetric field $\Theta_{\alpha\beta}$ of rank~$2$, the decomposition schema is
\[
  \Theta_{\alpha\beta} = \theta_{\alpha\beta} + \HH^{-1}\left(
    \diffp{{B_\beta}}{{x_\alpha}} - \diffp{{B_\alpha}}{{x_\beta}}
  \right) + \HH^{-2} \diffp{\Sigma}{{x_\alpha}{x_\beta}}.
\]
The composite fields $\Sigma$, $\varphi_{\alpha\beta}$, and $\theta_{\alpha\beta}$ all have a positive defined total energy, and $B_\alpha$ and $C_\alpha$ have a negative total energy.
The positive and negative parts of the total energy and thus completely separated in a covariant way, thanks to the zero divergence.
$\Sigma$ is a spinor field; $B_\alpha$ and $C_\alpha$ are vector fields with zero divergence, and thus both of spin~$1$.
The antisymmetric field $\varphi_{\alpha\beta}$ of zero divergence has only three independent components.
It is equivalent to a \todo field of zero divergence, and it thus also a field of spin~$1$.
$\theta_{\alpha\beta}$, having six independent components, does not represent a specific well-defined spin.

\medskip\hrulefill\bigskip

The field $\theta_{\alpha\beta}$ in fact represents a mixture of two spins: $2$ and $0$.
The trace $\theta_{\alpha\alpha}$ constitutes a component spinor field, and can be explicitly separated from the other components of the field.
The decomposition follows the schema
\begin{gather*}
  \theta_{\alpha\beta} = \overline{\theta}_{\alpha\beta} + \d_{\alpha\beta}\overline{\Sigma},
\\\diffp{{\overline{\theta}_{\alpha\beta}}}{{x_\alpha}} = 0,
  \quad \overline{\theta}_{\alpha\alpha} = 0,
\\\mbox{where $\d_{\alpha\beta}=\delta_{\alpha\beta}-\HH^{-2}\diffp{}{{x_\alpha}{x_\beta}}$.}
\end{gather*}

Then $\theta_{\alpha\alpha}=3\overline{\Sigma}$ constitutes a spinor field, and $\overline{\theta}_{\alpha\beta}$, satisfying the zero divergence and zero trace conditions, and having five independent components, is indeed a field of total spin~$2$.
The two fields have a positive total energy.
Finally, the complete decomposition schema of the symmetric field $\theta_{\alpha\beta}$ with respect to the spin states can be written as
\oldpage{13-05}
\[
\label{equation3}
  \Theta_{\alpha\beta} = \overline{\theta}_{\alpha\beta} + \HH^{-1}\left(
    \diffp{{B_\beta}}{{x_\alpha}} + \diffp{{B_\alpha}}{{x_\beta}}
  \right) + \HH^{-2} \diffp{\Sigma}{{x_\alpha}{x_\beta}} + \d_{\alpha\beta}\overline{\Sigma}.
  \tag{3}
\]
For the general rank-$2$ field $\Omega_{\alpha\beta}$, the complete schema is
\[
\label{equation4}
  \Omega_{\alpha\beta} = \overline{\theta}_{\alpha\beta} + \varphi_{\alpha\beta} + \HH^{-1}\left(
    \diffp{{B_\beta}}{{x_\alpha}} + \diffp{{C_\alpha}}{{x_\beta}}
  \right) + \HH^{-2} \diffp{\Sigma}{{x_\alpha}{x_\beta}} + \d_{\alpha\beta}\overline{\Sigma}.
  \tag{4}
\]

The explicit decomposition of the spin states can be generalised to tensor fields of arbitrary rank.

\medskip\hrulefill\bigskip

Up until now we have assumed that the proper mass of the quanta is different from zero ($\HH\neq0$).
If the proper mass is strictly zero, then the problem is very different.
For a vector field $(A_\lambda)$, for example, if we try the same ``decomposition'' as in the non-zero mass case, the schema should be
\[
\label{equation5}
  A_\lambda = B_\lambda + \diffp{\Sigma}{{x_\lambda}},
  \quad
  \diffp{{B_\lambda}}{{x_\lambda}} = 0.
  \tag{5}
\]
But then $\diffp{{A_\lambda}}{{x_\lambda}}=0$ as well, and $A_\lambda$ is, from the physical point of view, equivalent to $B_\lambda$, whereas $\Sigma$ does not constitute a field.
The schema in \cref{equation5} automatically imposes upon $A_\lambda$ the zero-divergence condition.
$B_\lambda$ has only two independent components, corresponding to the two possible states of total spin~$1$, in the case of zero proper mass.
Then, if $A_\lambda$ does not satisfy any additional conditions, the procedure in \cref{equation5} cannot be applied to it.
There is thus no way of decomposing the spin states of the field $A_\lambda$, and it is impossible to say which spin, or which mixture of spins, can represent this field.
If $A_\lambda$ satisfies the zero divergence conditions, then \cref{equation5} can be applied, and $A_\lambda$ is then equivalent to $B_\lambda$.
So \cref{equation5} is thus the gauge transformation, and $A_\lambda$ is gauge invariant.
The gauge invariance is a necessary condition for being able to define the spin of quanta with zero proper mass.

If $\Theta_{\alpha\beta}$ is a symmetric field of rank~$2$ of zero proper mass, then the trace $\Theta_{\alpha\alpha}$ constitutes a component spinor field, and can be explicitly separated, despite the zero proper mass.
This follows from the fact that the trace is only a simple algebraic relation between certain components of the field.
The separation of the trace follows the schema
\[
\label{equation6}
  \Theta_{\alpha\beta} = \overline{\Theta}_{\alpha\beta} + \delta_{\alpha\beta}\overline{\Sigma},
  \quad
  \overline{\Theta}_{\alpha\alpha} = 0.
  \tag{6}
\]

\oldpage{13-06}
$\Theta_{\alpha\beta}$ constitutes a spinor field of zero proper mass.
$\overline{\Theta}_{\alpha\beta}$ has again nine independent components, and does not correspond to a specific well-defined spin.
But it is impossible to decompose it again by using the zero divergence, as we did in the non-zero proper mass case.
If $\overline{\Theta}_{\alpha\beta}$ satisfies the zero-divergence condition, then it will be gauge invariant with respect to the following gauge transformation (see \cref{equation3}):
\[
\label{equation7}
  \left\{
  \begin{gathered}
    \overline{\Theta}_{\alpha\beta} = \overline{\theta}_{\alpha\beta} + \diffp{{B_\beta}}{{x_\alpha}} + \diffp{{B_\alpha}}{{x_\beta}} + \diffp{\Sigma}{{x_\alpha}{x_\beta}},
  \\\overline{\Theta}_{\alpha\alpha} = 0,
    \quad \diffp{{\overline{\theta}_{\alpha\beta}}}{{x_\alpha}} = 0,
    \quad \diffp{{B_\alpha}}{{x_\alpha}} = 0.
  \end{gathered}
  \right.
  \tag{7}
\]
$\overline{\Theta}_{\alpha\beta}$ is equivalent to $\overline{\theta}_{\alpha\beta}$ and $B_\alpha$, and $\Sigma$ does not constitute a field.
In this case, $\overline{\Theta}_{\alpha\beta}$ has only two independent components, corresponding to the two possible states of spin~$2$ in the case of zero proper mass.
This approach can be easily generalised to tensor fields of zero proper mass of arbitrary rank.


\section{Interactions with the spinor field}
\label{section2}

The interaction of a vector (or \todo) field $(A_\lambda)$ with the Dirac spinor field $(\Psi)$ can be introduced with the Lagrangian of the interaction
\[
  \LL' = \HH^{-1}\SS X + \
\]


\end{document}
