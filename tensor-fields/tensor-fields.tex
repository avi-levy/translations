\documentclass{article}

\usepackage{amssymb,amsmath}

\usepackage{hyperref}
\usepackage[nameinlink]{cleveref}
\usepackage{enumerate}
\usepackage{tikz-cd}
\usepackage{graphicx}

\usepackage{mathrsfs}
%% Fancy fonts --- feel free to remove! %%
\usepackage{newtxmath} %% necessary to get \partial with ebgaramond-maths
\usepackage{ebgaramond-maths}
\usepackage{mathpazo}

\makeatletter
  \DeclareSymbolFont{ntxletters}{OML}{ntxmi}{m}{it}
  \SetSymbolFont{ntxletters}{bold}{OML}{ntxmi}{b}{it}
  \re@DeclareMathSymbol{\partial}{\mathord}{ntxletters}{"40}
\makeatother


\crefname{section}{Section}{Sections}
\crefname{equation}{}{}


%% Shortcuts %%

\newcommand{\ham}{\mathscr{H}}

\renewcommand{\geq}{\geqslant}
\renewcommand{\leq}{\leqslant}

\newcommand{\todo}{\textbf{ !TODO! }}
\newcommand{\oldpage}[1]{\marginpar{\footnotesize$\Big\vert$ \textit{p.~#1}}}


%% Document %%

\usepackage{embedall}
\begin{document}

\renewcommand{\abstractname}{Translator's note.}

\title{Tensor fields}
\author{K.H. Tzou}
\date{8\textsuperscript{th} of February, 1955}
\maketitle

\begin{abstract}
  \renewcommand*{\thefootnote}{\fnsymbol{footnote}}
  \emph{This text is one of a series\footnote{\url{https://github.com/thosgood/translations}} of translations of various papers into English.}
  \emph{What follows is a translation (last updated \today) of the French paper:}

  \medskip\noindent
  \textsc{Tzou, K.H.} Les champs tensoriels. \emph{Séminaire L. de Broglie. Théories physiques}, Volume~\textbf{24} (1954-1955), Talk no.~10, 10~p. {\footnotesize\url{http://www.numdam.org/item/SLDB_1954-1955__24__A9_0/}}
\end{abstract}

\tableofcontents


%% Content %%

\bigskip\bigskip
\oldpage{13-01}
The theory of elementary particles consists of associating, to each type of particle, a field.
The field is chosen according to the spin of the particle.
According to the means of description, the fields can be classified into two categories.
Firstly, there are the \emph{spinorial fields}, whose variables are spinors ($\psi_{\alpha\beta\ldots}$) that satisfy first-order differential equations of the following type:
\[
  \left(\gamma_\lambda \frac{\partial}{\partial x_\lambda} + \ham\right)\psi = 0
\]
where $\gamma_\lambda$ is


\end{document}
