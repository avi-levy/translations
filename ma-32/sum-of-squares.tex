\documentclass{article}

\title{On the representation of definite forms as the sum of squares of forms}
\author{David Hilbert}
\date{}

\usepackage{amssymb,amsmath}

\usepackage{hyperref}
\usepackage[nameinlink]{cleveref}
\usepackage{enumerate}

\usepackage{mathrsfs}
%% Fancy fonts --- feel free to remove! %%
\usepackage{Baskervaldx}
\usepackage{mathpazo}


\usepackage{fancyhdr}
\usepackage{lastpage}
\usepackage{xstring}
\makeatletter
\ifx\pdfmdfivesum\undefined
  \let\pdfmdfivesum\mdfivesum
\fi
\edef\filesum{\pdfmdfivesum file {\jobname}}
\pagestyle{fancy}
\makeatletter
\let\runauthor\@author
\let\runtitle\@title
\makeatother
\fancyhf{}
\lhead{\footnotesize\runtitle}
\rhead{\footnotesize Version: \texttt{\StrMid{\filesum}{1}{8}}}
\cfoot{\small\thepage\ of \pageref*{LastPage}}


\crefname{section}{Section}{Sections}
\crefname{equation}{}{}


%% Shortcuts %%

\newcommand{\sh}{\mathscr}
\newcommand{\cat}{\mathcal}

\renewcommand{\geq}{\geqslant}
\renewcommand{\leq}{\leqslant}

\newcommand{\todo}{\textbf{ !TODO! }}
\newcommand{\oldpage}[1]{\marginpar{\footnotesize$\Big\vert$ \textit{p.~#1}}}


%% Document %%

\usepackage{embedall}
\begin{document}

\maketitle
\thispagestyle{fancy}

\renewcommand{\abstractname}{Translator's note.}

\begin{abstract}
  \renewcommand*{\thefootnote}{\fnsymbol{footnote}}
  \emph{This text is one of a series\footnote{\url{https://github.com/thosgood/translations}} of translations of various papers into English.}
  \emph{The translator takes full responsibility for any errors introduced in the passage from one language to another, and claims no rights to any of the mathematical content herein.}
  
  \emph{What follows is a translation (last updated \today) of the German paper:}

  \medskip\noindent
  \textsc{Hilbert, D.}
  ``Ueber die Darstellung definiter Formen als Summe von Formenquadraten''.
  \emph{Mathematische Annalen}, Volume~\textbf{32} (1888), pp.~342--350.
  {\footnotesize\url{http://eudml.org/doc/157385}}
\end{abstract}

\setcounter{footnote}{0}
\renewcommand{\thefootnote}{\fnsymbol{footnote}}


%% Content %%

\oldpage{342}

An algebraic form of even order $n$ with real coefficients and $m$ homogeneous variables is said to be \emph{definite} if it takes a positive value for every system of real values of the $m$ variables and, moreover, has a non-zero discriminant.
A form with real coefficients is also called a \emph{real form} for short.

It is known that \emph{any definite quadratic form} with $m$ variables can be expressed as the sum of $m$ squares of real linear forms.
In the same way, \emph{any definite binary form} can be represented as the sum of the squares of two real forms, as can be seen by a suitable factor decomposition of the form.
Since the representation in question reveals the definite character of the form in the simplest possible way, an investigation of the possibility of such a representation in general seems to be of interest.
For the case where $n=4$, $m=3$, we have the following theorem:

\emph{Every definite ternary quartic form can be written as the sum of the squares of three quadratic forms.}

To proof this, consider a ternary quartic form $F$ that can be written as the sum of the squares of three quadratic forms $\varphi$, $\psi$, and $\chi$.
If the form $F$ can also be written as the sum of the squares of three quadratic forms $\varphi+\varepsilon\varphi'$, $\psi+\varepsilon\psi'$, and $\chi+\varepsilon\chi'$, where $\varepsilon$ is an infinitely small constant, then the comparison of the two expressions leads to the relation
\[
\label{equation1}
  \varphi\varphi' + \psi\psi' + \chi\chi' = 0.
  \tag{1}
\]
The three equations
\[
\label{equation2}
  \varphi=0,
  \quad\psi=0,
  \quad\chi=0
  \tag{2}
\]
cannot have a common solution.
By \cref{equation1}, \todo
\oldpage{343}
the quadratic form $\varphi'$ must vanish;
thus
\[
  \varphi' = \alpha\psi+\gamma\chi,
\]
and, similarly,
\begin{align*}
  \psi' &= \beta\varphi+\zeta\chi,
\\\chi' &= \delta\varphi+\vartheta\psi.
\end{align*}
By substituting these equations into \cref{equation1}, we obtain the following relations for the introduced constants:
\[
  \alpha+\beta = 0,
  \quad\gamma+\delta = 0,
  \quad\zeta+\vartheta = 0.
\]
As soon as the resultant of the three equations \cref{equation2} is non-zero, the identity \cref{equation1} can only be satisfied by a linear combination of the three solutions:
\[
  \left\{
    \begin{array}{rcr}
      \varphi' &= &0
    \\\psi' &= &\chi
    \\\chi' &= &-\psi
    \end{array}
  \right\}
  \quad
  \left\{
    \begin{array}{rcr}
      \varphi' &= &-\chi
    \\\psi' &= &0
    \\\chi' &= &\varphi
    \end{array}
  \right\}
  \quad
  \left\{
    \begin{array}{rcr}
      \varphi' &= &\psi
    \\\psi' &= &-\varphi
    \\\chi' &= &0
    \end{array}
  \right\}
\]
i.e. all possible solution systems represent the same form $F$, and so all representations of $F$ as the sum of the squares of three forms consist only of $\varphi$, $\psi$, and $\chi$.
Since the system of these three quadratic forms has 18 coefficients, whereas the quartic form $F$ has 15 coefficients, it follows from the above remarks that every ternary quartic form can be expressed as the sum of the squares of three forms.
\footnote{The general principle on which this is based comes from L.~Kronecker, c.f. \emph{Mathematische Annalen} Bd.~\textbf{13}, p.~549.}


%% Bibliography %%

\nocite{*}
\bibliographystyle{acm}

\end{document}
